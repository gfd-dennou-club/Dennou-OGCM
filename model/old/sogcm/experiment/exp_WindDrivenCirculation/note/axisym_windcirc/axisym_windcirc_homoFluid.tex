\chapter{水惑星における風成循環}

本章では, 水惑星における風成循環の問題を定式化し, その近似解を導く. 
水惑星における風成循環の主な特徴は, 
%%
\begin{itemize}
 \item 内部領域における順圧的な東西流
 \item 弱い南北方向の輸送
\end{itemize}
%%
である. 
これらを完全に記述するためためには, 三次元静力学ブジネスク方程式を用いる必要がある. 
しかし, その基本的な特徴はより簡単な系を使って記述することができる. 
本章では, 最初に, 密度一様で軸対称な循環をもつ系において, 
水惑星の風成循環を定式化し, その力学の考察を行う. 
その後に, 密度成層の効果を取り入れることにする. 

\section{密度一様・軸対称設定における風成循環}
\subsection{モデルの定式化}

本節では, 
%%
\begin{itemize}
 \item 密度一様
 \item 軸対称(経度微分ゼロ)
 \item 循環のスケールは, 惑星半径$a$と同じオーダである. 
 \item 渦による混合の効果は, 一定の渦粘性係数を持つ拡散項によって簡単に表現する
\end{itemize}
%%
であることを仮定する. 
1, 2 番目は, 水惑星の海洋風成循環の基本的特徴を, 
できるだけ簡単に記述するために導入した仮定である. 
3 番目の仮定は, 地球でみられるような風成循環において当てはまる仮定である%
\footnote{
木星大気のように, 東西ジェットが南北に多数存在する場合には, 
その東西ジェットによって駆動される海洋大循環のスケールは, $a$よりずっと小さくなるだろう. 
その場合は, 3 番目の仮定は当てはまらない. 
}.

\subsubsection*{支配方程式系}
%%
循環のスケールが惑星半径と同じオーダーである場合には, 
その循環を惑星地衡流方程式系%%
\footnote{
詳細は, 別ノート「海洋大循環を記述するための方程式系」における, 惑星地衡流方程式系の導出を参考されたい. 
}
を使って記述するのが適切である. 
今,  密度一様と軸対称の仮定を考慮しよう. 
また, 長さのスケールを惑星半径$a$, 速度のスケールを$U$, 時間スケールを移流時間スケールでとるとき, 
無次元化した支配方程式系は,  
%%
\begin{subequations} %16:32の式群
  \begin{align}
  R_o \left( \DD{u}{t} - uv\tan{\theta}\right) - \sin{\theta} v 
     &= - \dfrac{E_H}{2} \DP{}{\theta} \mathscr{D}[u] + \dfrac{E_V}{2} \DP[2]{u}{z}
   ,\\
  R_o \left( \DD{v}{t} - u^2\tan{\theta}\right) - \sin{\theta} u 
     &= - \DP{p}{\theta} + \dfrac{E_H}{2} \DP{}{\theta} \mathscr{D}[v] + \dfrac{E_V}{2} \DP[2]{v}{z}
   , \\
   \mathscr{D}[v] + \DP{w}{z} &= 0
  \end{align}
\label{eq:PGE1_nondim_axisym}
\end{subequations}
%%
と書かれる. 
今, 各無次元数を次のように定義した. 
%%
\begin{subequations} %12:17の式群
  \begin{equation}
     \textrm{ロスビー数:} \;\;\; 
     R_o = \dfrac{U}{2\Omega a},
  \end{equation}
  \begin{equation}
     \textrm{水平エクマン数:} \;\;\; 
     E_H = \dfrac{2A_h}{2\Omega a^2}, 
  \end{equation}
  \begin{equation}
     \textrm{鉛直エクマン数:} \;\;\; 
     E_V = \dfrac{2A_v}{2\Omega D^2}.
  \end{equation}
\end{subequations}
%%
ここで, $\Omega$は自転角速度, $D$は流体層の厚さ, $A_h, A_v$はそれぞれ水平・鉛直渦粘性係数である. 
また, 発散演算子と関係した微分演算子を, 
%%
\begin{equation}
  \mathscr{D}[(\;)] = \dfrac{1}{\cos{\theta}} \DP{((\;)\cos{\theta})}{\theta}
\end{equation}
%%
と定義した. 
さらに, 今, ラグランジュ微分は, 
%%
\begin{equation}
  \DD{}{t} = \DP{}{t} + v\DP{}{\theta} + w\DP{}{z}
\end{equation}
%%
と書かれる. 


\subsubsection*{境界条件}
%%
\eqref{eq:PGE1_nondim_axisym}に対して, 上下端における境界条件を以下のように課す. 
下端境界は剛体壁であり, 
%%
\begin{equation}
  w = 0 \;\;\;\;\; {\rm at} \;\;\; z=0
\end{equation}
%%
を課す. 
また, 水平速度に対して, 滑り無し条件
%%
\begin{equation}
  u, v = 0 \;\;\;\;\; {\rm at} \;\;\; z=0
\end{equation}
%%
を課す. 

一方, 上端境界は自由表面であり, 海面応力として, 
%%
\begin{equation}
  \DP{u}{z} = \dfrac{D \tau_0}{A_V \rho_0 U} \tau_\lambda(\theta), \;\;\; 
  \DP{v}{z} = 0  \;\;\;\;\; {\rm at} \;\; z=1
\end{equation}
%%
を与える. 
ここでは, 上端境界において, rigid-lid 近似を適用し, 
%%
\begin{equation}
  w = 0 \;\;\;\;\; {\rm at} \;\;\; z=1
\end{equation}
%%
を課す. 

%%%%%%%%%%%%%%%%%%%%%%%%%%%%%%%%%%%%%%%%%%%%%%%
\subsection{ポテンシャル渦度方程式の導出}
%%%%%%%%%%%%%%%
始めに, 各無次元変数をロスビー数によって展開する.  
つまり, 
%%
\begin{equation}
  u = u_{(0)} + R_o u_{(1)} + \dots
\end{equation}
%%
のように展開する. 
このとき, 水平・鉛直渦粘性項の係数は, それぞれ, 
%%
\begin{equation}
  E_H = R_o \dfrac{2}{R_e}, \;\; E_V = R_o (\sqrt{E_V} r)
\end{equation}
%%
と書き直す. 
ここで, $R_e(=LU/A_h))$は水平レイノルズ数である. 
また, $r=\sqrt{E_V}/{R_o}$はエクマンパンピングの時間スケールと移流時間スケールの比を表す. 
海洋大循環において, 多くの場合, $R_e^{-1}$や$r$は 1 より十分小さいため, 
ここではそのような場合を考える. 

\subsubsection*{O(1) のバランス:}
%%%%%%%
\eqref{eq:PGE1_nondim_axisym}において, O(1) の項だけ取り出すと, 
%%
\begin{subequations} %16:32の式群
  \begin{align}
   - \sin{\theta} \; v_{(0)} &= 0
   ,\\
   - \sin{\theta} \; u_{(0)}
     &= - \DP{p_{(0)}}{\theta}
   , \\
   \DP{p_{(0)}}{z} &= 0
   , \\ 
   \mathscr{D}[v_{(0)}] + \DP{w_{(0)}}{z} &= 0
  \end{align}
\label{eq:PGE1_nondim_axisym_order1}
\end{subequations}
%%
を得る. 
したがって, ロスビー数の最低次に対する速度場は, 
%%
\begin{equation}
  v_{(0)} = 0, \;\;\; - \sin{\theta} \; u_{(0)} = - \DP{p_{(0)}}{\theta}, \;\; w_{(0)} = 0
\end{equation}
%%
と決まる. 
ここで, 上端・下端で鉛直速度はゼロであることを用いた. 
また, 
%%
$$
  - \sin{\theta} \; \DP{u_{(0)}}{z} = \DP{}{z}\DP{p_{(0)}}{\theta} = 0 
$$
%%
より, $u_{(0)}$は, (エクマン層内を除いて)鉛直方向に変化しないことに注意されたい. 

\subsubsection*{O($R_o$) のバランス:}
%%
\eqref{eq:PGE1_nondim_axisym}において, O($R_o$) の項だけ取り出すと, 
%%
\begin{subequations} %16:32の式群
  \begin{align}
  \DP{u_{(0)}}{t} - \sin{\theta} v_{(1)} 
     &= - \dfrac{1}{R_e} \mathscr{D}[u_{(0)}]
   ,\\
   - u_{(0)}^2\tan{\theta} - \sin{\theta} u_{(1)}
     &= - \DP{p_{(1)}}{\theta} 
   , \\
   \mathscr{D}[v_{(1)}] + \DP{w_{(1)}}{z} &= 0
  \end{align}
\label{eq:PGE1_nondim_axisym_Ro1}
\end{subequations}
%%
を得る. 

次に, \eqref{eq:PGE1_nondim_axisym_Ro1}の一式目に$-\mathscr{D}$を作用させて, 
\eqref{eq:PGE1_nondim_axisym_Ro1}を用いて$\partial v_1/\partial \theta$を消去すれば, 
渦度方程式
%%
\begin{equation}
  \DP{\zeta_{(0)}}{t} - \cos{\theta} v_{(1)} = \sin{\theta} \DP{w_{(1)}}{z} + \dfrac{1}{R_e} \mathscr{L}[\zeta_{(0)}]
\end{equation}
%%
を得る. 
ここで, $\zeta_{(0)} = - \mathscr{D} [u_{(0)}]$である.   
また, $\mathscr{L}[\;]=\mathscr{D}[\DP{}{\theta}[\;]]$は水平ラプラシアンである. 

さらに, $u_{(0)}, v_{(1)}$が$z$に依存しないことに注意して, この両辺を$z=0$から$z=1$まで積分すれば, 
%%
\begin{equation}
    \DP{\zeta_{(0)}}{t} - \cos{\theta} v_{(1)} 
  = \sin{\theta} \Big[ w_{(1)} \Big]_{z=0}^{z=1} + \dfrac{1}{R_e} \mathscr{L}[\zeta_{(0)}]
\end{equation}
%%
となる. 
今, $z=0,1$における$w_1$として, エクマン層の線形理論から導かれるエクマン・パンピング速度
%%
\begin{align*}
  w_* &= {\rm curl}_* \left(\dfrac{\Dvect{\tau}_*}{\rho_0 f_*}\right) \;\;\;\; {\rm at \;\;}z=1, \\
  w_* &= {\rm curl}_* \left( \dfrac{\delta_E(\theta)}{2} \Dvect{u}_* \right) \;\;\;\; {\rm at \;\;}z=0 
\end{align*}
%%
を与えることにする.  
ここで, $\delta_E=(2A_V/f_*)^{1/2}$は, 緯度に依存するエクマン層の厚さである. 
このとき, 均質流体対する惑星地衡流ポテンシャル渦度方程式
%%
\begin{equation}
\boxed{
  \DP{\zeta_{(0)}}{t} - \cos{\theta} v_{(1)}
  = \sin{\theta} \left\{ \dfrac{\tau_0/\rho_0}{2\Omega U D R_o} \mathscr{D}[\dfrac{-\tau_x}{\sin{\theta}}] 
                   - \dfrac{r}{2} \mathscr{D}[\dfrac{-u_{(0)}}{\sqrt{\sin{\theta}}}] \right\}
    + \dfrac{1}{R_e} \mathscr{L}[\zeta_{(0)}]
}
\label{eq:PGE1_nondim_axisym_PV}
\end{equation}
%%
を得る. 

\subsection{定常状態における漸近級数解の導出}
\eqref{eq:PGE1_nondim_axisym_PV}の定常解を, 
$r_{H,V}\ll 1$の場合の漸近級数解として得ることを考えよう. 

定常状態において, \eqref{eq:PGE1_nondim_axisym_Ro1}の一式目は
%%
\begin{equation}
  - \sin{\theta} v_{(1)} = - \dfrac{1}{R_e}\DP{}{\theta} \mathscr{D}[u_{(0)}]
\label{eq:PGE1_nondim_axisym_Ro1_xmomentum_static}
\end{equation}
%%
であるので, これを用いて, 
\eqref{eq:PGE1_nondim_axisym_PV} の$v_1$を消去し, 
少し変形を行えば, 定常状態において, 
%%
\begin{equation}
 0 = \left\{ \dfrac{\tau_0}{\rho_0 2\Omega U D R_o} \mathscr{D}[\dfrac{-\tau_x}{\sin{\theta}}] 
                   - \dfrac{r}{2} \mathscr{D}[\dfrac{-u_{(0)}}{\sqrt{\sin{\theta}}}] \right\}
    + r_{H,V} \mathscr{D}[\dfrac{1}{\sin{\theta}}\DP{\zeta_{(0)}}{\theta}]
\label{eq:PGE1_nondim_axisym_PV_static}
\end{equation}
%%
を得る. 

今, 速度スケールを, 
%%
\begin{equation}
  U = \dfrac{2\tau_0}{\rho_0 2\Omega E_V^{1/2} D}
\end{equation}
%%
ととれば, \eqref{eq:PGE1_nondim_axisym_PV_static}は, 
%%
\begin{equation}
 0 = \left\{ \mathscr{D}[\dfrac{-\tau_x}{\sin{\theta}}] 
                   - \mathscr{D}[\dfrac{-u_{(0)}}{\sqrt{\sin{\theta}}}] \right\}
    + r_{H,V} \mathscr{D}[\dfrac{1}{\sin{\theta}}\DP{\zeta_{(0)}}{\theta}]
\label{eq:PGE1_nondim_axisym_PV_static2}
\end{equation}
%%
となる. 
ここで, 
%%
\begin{equation}
  r_{H,V} = \dfrac{E_H}{E_V^{1/2}}
\end{equation}
%%
とおいた. 
最後に, 極において$\tau_x=0, u_{(0)}=0$であるとし, 
\eqref{eq:PGE1_nondim_axisym_PV_static2}を極から緯度$\theta$まで積分すれば, 
$u_{(0)}$に対する方程式
%%
\begin{equation}
%\boxed{
  0 = \dfrac{\tau_x}{\sin{\theta}} - \dfrac{u_{(0)}}{\sqrt{\sin{\theta}}} 
     + r_{H,V} \dfrac{1}{\sin{\theta}}\DP{}{\theta} \mathscr{D}[u_{(0)}]
%}
\label{eq:ode_forU0}
\end{equation}
%%
を得る. 

次に, $r_{H,V} \ll 1$の場合を考え, \eqref{eq:ode_forU0}の漸近級数解を導こう. 
今, $u_{(0)}$を$r_{H,V}$によって, 
$$
 u_{(0)} = u_{(0,0)} + r_{H,V} u_{(0,1)} + \cdots
$$
と展開し, \eqref{eq:ode_forU0}に代入する. 
このとき, $r_{H,V}$の最低次の項を集めれば, 
%%
\begin{equation}
  u_{(0,0)} = \dfrac{\tau_x}{\sqrt{\sin\theta}}
\end{equation}
%%
を得る. 
さらに, $(r_{H,V})^m \;(m=1,2, \dots)$の項を集めれば, 
%%
\begin{equation}
  u_{(0,m)} = \dfrac{1}{\sqrt{\sin\theta}}\DP{}{\theta}\mathscr{D}[u_{(0,m-1)}]
\end{equation}
%%
を得る. 
したがって, 最終的な\eqref{eq:ode_forU0}の漸近級数解の形式は, 
%%
\begin{equation}
\boxed{
  u_{(0)} = \sum_{m=0}^\infty 
              \left(\dfrac{r_{H,V}}{\sqrt{\sin\theta}} \DP{}{\theta} \mathscr{L} \right)^m 
                \left[\dfrac{\tau_x(\theta)}{\sqrt{\sin\theta}}\right]
}
\label{eq:u_RoOder1_asympt}
\end{equation}
%%
と導かれる. 

$u_{(0)}$以外の解は, 上で得た\eqref{eq:u_RoOder1_asympt}を用いて, 以下のように表される. 
$v_{(1)}$は\eqref{eq:PGE1_nondim_axisym_Ro1_xmomentum_static}から,
%%
\begin{equation}
\boxed{
  v_{(1)} (\theta) = - E_H \dfrac{1}{\sin\theta} \DP{}{\theta} \mathscr{D}[u_{(0)}(\theta)]
}
\end{equation}
%%
と書かれる. 
また, \eqref{eq:PGE1_nondim_axisym_Ro1}の三式目から, $w_{(1)}$は$v_{(1)}$を使って, 
%%
\begin{equation}
\boxed{
  w_{(1)}(\theta,z) = w_{(1)}|_{z=1}(\theta) + (1-z) \mathscr{L}[v_{(1)}(\theta)] 
}
\end{equation}
%%
と書かれる. 
