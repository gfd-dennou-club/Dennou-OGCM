\chapter{支配方程式系の近似解: 赤道近傍}
\markright{\arabic{chapter} 支配方程式系の近似解: 赤道近傍}

本章では, $\varepsilon_{eq}$が 1 より十分小さい場合を考えて, 
\eqref{eq:x_momem_nondim_eq} - \eqref{eq:continuity_nondim_eq} の線形解の導出を, 
\citet{gill1971421,dowden1972equatorial}に従って行う. 
彼らによれば, 海面応力の詳細な分布は赤道境界層の力学に対して重要でない.  
そのため, 本ノートにおいても, 簡単化のために南北方向に一様な海面応力を与える. 
すなわち, 
%%
\begin{equation}
  \tau_x(y) = -1
\end{equation}
%%
の場合を考えることにする. 

%%%%%%%%%%%%%%%%%%%%%%%%%%%%%%%%%%%%%%%%%%%%%%%%%%
%
\section{線形近似解(赤道近傍)}
%%
\eqref{eq:x_momem_nondim_eq} - \eqref{eq:continuity_nondim_eq} において, 
非線形項を無視すれば, 
%%
\begin{subequations}
 \begin{align}
   - yv &= \DP[2]{u}{y} + \delta_E^2 \DP[2]{u}{z}, \\
     yu &= - \DP{p}{y} + \DP[2]{v}{y} + \delta_E^2 \DP[2]{u}{z}, \\
   \DP{u}{y} + \DP{w}{z} &= 0
 \end{align}
\label{eq:EBLgoverneq_linear}
\end{subequations}
%%
となる. 
以下では, 流れを順圧成分と傾圧成分に分割して, \eqref{eq:EBLgoverneq_linear} の解を求める. 
初めに, 流れの順圧成分$(\bar{u}, \bar{v})$を
%%
\begin{align}
  (\bar{u}, \bar{v}) = \int_0^1 (u, v) \; dz
\end{align}
%%
と定義する. 
一方で, 流れの傾圧成分$(\hat{u}, \hat{v})$は, 流れの順圧成分からのずれによって定義する. 
すなわち, 
%%
\begin{align}
  (\hat{u}, \hat{v}) = (u, v) - (\bar{u}, \bar{v}). 
\end{align}
%%

%%%%%%%%%%%%%%%%%%%%%%%%%%%%%%%%%%%%%%%%%%%%%%%%%%%%
\subsection{流れの順圧成分}
%%
流れの順圧成分に対する方程式を得るために, 
\eqref{eq:EBLgoverneq_linear}を$z$について 0 から 1 まで積分し, 
上下端における境界条件を適用すれば, 
%%
\begin{subequations}
 \begin{align}
   - y\bar{v} &=  \DP[2]{\bar{u}}{y} + \delta_E \left[ -1 - \delta_E \left(\DP{\hat{u}}{z}\right)_{z=0} \; \right], \\
     y\bar{u} &= - \DP{p}{y} + \DP[2]{\bar{v}}{y} + \delta_E \left[ \;\;\;\;\;  - \delta_E \left(\DP{\hat{v}}{z}\right)_{z=0} \; \right], \\
   \DP{\bar{v}}{y}  &= 0
 \end{align}
\label{eq:EBLgoverneq_linear_barotropic}
\end{subequations}
%%
が得られる. 

\eqref{eq:EBLgoverneq_linear_barotropic}を満たす解について考える. 
例えば, $|y|=y_\infty$において, $\bar{u} = \bar{u}_\infty$となる, 
\eqref{eq:EBLgoverneq_linear_barotropic}の解を求めることにしよう. 
O($\delta_E$) の項を無視することにすれば, 
%%
\begin{align}
  \bar{v} &= 0, \;\;\;\;\;\; \bar{u} = \bar{u}_\infty, \\
  p &= - \dfrac{\bar{u}}{2} y^2 + ({\rm const}). 
\end{align}
%%
が得られる. 

%%%%%%%%%%%%%%%%%%%%%%%%%%%%%%%%%%%%%%%%%%%%%%%%%%%%
\subsection{流れの傾圧成分}
%%
流れの傾圧成分に対する方程式を得るために, 
\eqref{eq:EBLgoverneq_linear}を\eqref{eq:EBLgoverneq_linear_barotropic}で引けば, 
%%
%%
\begin{subequations}
 \begin{align}
   - y\hat{v} &=  \DP[2]{\hat{u}}{y} + \delta_E^2 \DP[2]{\hat{u}}{z} + \delta_E \left[ 1 + \delta_E \left(\DP{\hat{u}}{z}\right)_{z=0} \; \right], \\
     y\hat{u} &=  \DP[2]{\hat{v}}{y} + \delta_E^2 \DP[2]{\hat{v}}{z} + \delta_E \left[ \;\;\;\;\;\;  \delta_E \left(\DP{\hat{v}}{z}\right)_{z=0} \; \right], \\
   \DP{\hat{v}}{y} + \DP{w}{z} &= 0
 \end{align}
\label{eq:EBLgoverneq_linear_baroc}
\end{subequations}
%%
を得る. 
今, 
%%
\begin{equation}
  P = \hat{u} + i \hat{v}
\end{equation}
%%
とおけば, \eqref{eq:EBLgoverneq_linear_baroc}は, 
%%
\begin{equation}
\boxed{
  \delta_E^2 \DP[2]{P}{z} + \DP[2]{P}{y} - iyP = \delta_E \left[ 1 + \delta_E \left(\DP{P}{z}\right)_{z=0} \right]
}
\label{eq:EBLgoverneq_linear_baroc_P}
\end{equation}
%%
と書ける. 

\subsubsection*{流れの傾圧成分の分解}
次に, 流れの傾圧成分に対する線形の方程式\eqref{eq:EBLgoverneq_linear_baroc}の解を, 
それぞれ内部領域, 上側エクマン層, 下側エクマン層内の寄与に分割して考えるために, $P$を 
%%
\begin{equation}
  P = P_I + P_{\rm e1} + P_{\rm e2}
\end{equation}
%%
と分解する. 
ここで, $P_I, P_{\rm e1}, P_{\rm e2}$を以下のように定義する. 
%%
\begin{itemize}
 \item $P_I$: 内部領域の流れの速度. ただし, $P_I$に関する鉛直拡散項はゼロする.  
 \item $P_{e1}$: 上側エクマン層内の流れの速度. ただし, その外側ではゼロとする. 
 \item $P_{e1}$: 下側エクマン層内の流れの速度. ただし, その外側ではゼロとする. 
\end{itemize}

\subsubsection*{流れの傾圧成分の決定: (i) 内部領域}
%%
$P_I$に対する方程式は, \eqref{eq:EBLgoverneq_linear_baroc_P}から, 
%%
\begin{equation}
 \DP[2]{P_I}{y} - iyP_I = \delta_E \left[ 1 + \delta_E \left(\DP{P}{z}\right)_{z=0} \right]
\end{equation}
%%
と得られる. 
ここで, O($\delta_E$) 以下の項を無視することにすれば, 
%%
\begin{equation}
 \DP[2]{P_I}{y} - iyP_I = 0
\end{equation}
%%
となる. 
今, $|y| \to \infty$において, 
%%
\begin{equation}
  - iy P_I = 0
\end{equation}
%%
に漸近的に収束する条件を課せば,
%%
\begin{equation}
  P_I = 0
\end{equation}
%%
を得る. 
したがって, O($\delta_E$) までに対して, 
内部領域における流れの傾圧成分$P_I$は\textbf{ゼロ}である. 

\subsubsection*{流れの傾圧成分の決定: (ii) 上側エクマン層}
%%
今, 上端エクマン層内の流れの表現に適した座標
%%
\begin{equation}
  \tilde{z} = \dfrac{z-1}{\delta_E}
\end{equation}
%%
を導入する. 
この$\tilde{z}$を用いて, \eqref{eq:EBLgoverneq_linear_baroc_P}を書き直し, 
その結果を\eqref{eq:EBLgoverneq_linear_barotropic}で引けば, 
上側エクマン層内の流れ$P_{e1}$に対する方程式
%%
\begin{equation}
  \DP[2]{P_{\rm e1}}{\tilde{z}} + \DP[2]{P_{\rm e1}}{y} - iy P_{\rm e1} = 0
\label{eq:EBLgoverneq_linear_Pe1}
\end{equation}
%%
を得る. 

\eqref{eq:EBLgoverneq_linear_Pe1}に対して, 以下の境界条件を課す.  
%%
\begin{subequations}
 \begin{align}
    \DP{P_{\rm e1}}{\tilde{z}} \to -1  \;\;\;\;\;\;\;\; &{\rm as} \;\;\;\; \tilde{z} \to -0, \\
    P_{\rm e1} \to 0 \;\;\;\;\;\;\;\; &{\rm as} \;\;\;\; \tilde{z} \to - \infty, \\
    P_{\rm e1} \sim \dfrac{-1 \pm i}{2 \sqrt{|y|}} \exp{\left[ \tilde{z} (1+i) \sqrt{|y|/2} \right]}
       \;\;\;\; &{\rm as} \;\;\;\; |y| \to  \infty. 
 \end{align}
\label{eq:BC_for_Pe1}
\end{subequations}
%%
最後の境界条件により, 赤道から十分に遠方では, $P_{e1}$はエクマン解と接続しなければならない. 

境界条件\eqref{eq:BC_for_Pe1}を満たす\eqref{eq:EBLgoverneq_linear_Pe1}の解は, 最終的には以下のように得られる. 
その導出過程については, 付録を参照されたい. 
%%
\begin{align}
\boxed{
  P_{e1} = - \dfrac{1}{\sqrt{\pi}} \int_0^\infty \dfrac{1}{\sqrt{l}} 
             \exp{\left( -\dfrac{l^3}{3} - ily - \dfrac{\tilde{z}^2}{4l} \right)} \; dl
}
\end{align}
%%
また, 上の$P_{e1}$と連続の式を用いて, 上側エクマン層内の鉛直速度を求めれば, 
%%
\begin{align}
\boxed{
  w_{e1} = \dfrac{\delta_E}{\sqrt{\pi}} \int_0^\infty \sqrt{l} \exp \left(-\dfrac{l^3}{3} \right) \cos (ly) 
              \int_0^{\tilde{z}} \exp \left(-\dfrac{\tilde{z}^2}{4l} \right) \;\; d\tilde{z} \; dl
}
\end{align}
%%
と得られる. 
特に, $\tilde{z} \to -\infty$に対する鉛直速度は, 
%%
\begin{align}
 w_{\rm e1} |_{\tilde{z} \to - \infty} = \delta_E \int_0^\infty l \exp \left(-\dfrac{l^3}{3}\right)  \cos(ly) \; dl
\end{align}
%%
と書ける. 

\subsubsection*{流れの傾圧成分の決定: (iii) 下側エクマン層}
%%
上側エクマン層の場合と同様な方法によって解を導出できる. 

上側エクマン層と同様の手順により, 下側エクマン層の流れ$P_{\rm e2}$に対する方程式は, 
%%
\begin{equation}
  \DP[2]{P_{\rm e2}}{\tilde{z}} + \DP[2]{P_{\rm e2}}{y} - iy P_{\rm e2} = 0
\label{eq:EBLgoverneq_linear_Pe2}
\end{equation}
%%
と得られる. 
ただし, 伸縮座標$\tilde{z}$は
\begin{align}
  \tilde{z} = \dfrac{z}{\delta_E}
\end{align}
%%
と導入した. 

\eqref{eq:EBLgoverneq_linear_Pe2} に対して, 以下の境界条件を課すことにする. 
%%
\begin{subequations}
 \begin{align}
    P_{e2} = - (P_I + \bar{u} + i\bar{v}), \;\;\;\;\;\;\;\; &{\rm as} \;\;\;\; \tilde{z} = 0, \\
    P_{\rm e2} \to 0 \;\;\;\;\;\;\;\; &{\rm as} \;\;\;\; \tilde{z} \to  \infty, \\
    P_{\rm e2} \sim - (P_I + \bar{u}_\infty + i\bar{v}_\infty) \exp{\left[ \tilde{z} (1+i) \sqrt{|y|/2} \right]}
       \;\;\;\; &{\rm as} \;\;\;\; |y| \to  \infty. 
 \end{align}
\label{eq:BC_for_Pe2}
\end{subequations}
%%
最初の境界条件は, 下端境界における滑り無し条件である. 
また, 最後の境界条件によって, 赤道から十分遠方では, $P_{e2}$と接続しなければならない. 

境界条件\eqref{eq:BC_for_Pe2}を満たす\eqref{eq:EBLgoverneq_linear_Pe2}の解は, 最終的には以下のように得られる. 
その導出過程については, 付録を参照されたい. 
%%
\begin{align}
\boxed{
  P_{\rm e2} = - \dfrac{\bar{u}_\infty}{2\sqrt{\pi}} \int_0^\infty \dfrac{\tilde{z}}{l^{3/2}} 
             \exp{\left( -\dfrac{l^3}{3} - ily - \dfrac{\tilde{z}^2}{4l} \right)} \; dl
}
\end{align}
%%
また, 上の$P_{e1}$と連続の式を用いて, 上側エクマン層内の鉛直速度を求めれば, 
%%
\begin{align}
\boxed{
  w_{\rm e2} = \delta_E \dfrac{\bar{u}_\infty}{\sqrt{\pi}} \int_0^\infty \sqrt{l} \exp \left(-\dfrac{l^3}{3} \right) \cos (ly) 
              \left[ 1 - \exp \left(-\dfrac{z^2}{4l} \right)\right] \; dl
}
\end{align}
%%
と得られる. 
特に, $\tilde{z} \to \infty$に対する鉛直速度は, 
%%
\begin{align}
 w_{\rm e2} |_{\tilde{z} \to - \infty} = \delta_E \dfrac{\bar{u}_\infty}{\sqrt{\pi}}
              \int_0^\infty l \exp \left(-\dfrac{l^3}{3}\right)  \cos(ly) \; dl
\end{align}
%%
と書ける. 

