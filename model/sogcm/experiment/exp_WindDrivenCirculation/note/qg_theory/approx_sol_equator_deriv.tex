\subsection*{付録 B: 赤道近傍における上側・下側エクマン層内の解の導出}
%%%%%%%%%%%%%%%%%%%%%%%%%%%%%%%%%%%%
\renewcommand{\theequation}{B.\arabic{equation}}
\setcounter{equation}{0}
%%%%%%%%%%%%%%%%%%%%%%%%%%%%%%%%%%%%%%
\subsubsection{上側エクマン層内の解$P_{\rm e1}$の導出}
%%
%%
\begin{equation*}
  \DP[2]{P_{\rm e1}}{\tilde{z}} + \DP[2]{P_{\rm e1}}{y} - iy P_{\rm e1} = 0
\tag{\ref{eq:EBLgoverneq_linear_Pe1}}
\end{equation*}
%%
の解を導出する. 
ただし, 求める解は次の境界条件, 
%%
 \begin{align*}
    \DP{P_{\rm e1}}{\tilde{z}} \to -1  \;\;\;\;\;\;\;\; &{\rm as} \;\;\;\; \tilde{z} \to -0, \\
    P_{\rm e1} \to 0 \;\;\;\;\;\;\;\; &{\rm as} \;\;\;\; \tilde{z} \to - \infty, \\
    P_{\rm e1} \sim \dfrac{-1 \pm i}{2 \sqrt{|y|}} \exp{\left[ \tilde{z} (1+i) \sqrt{|y|/2} \right]}
       \;\;\;\; &{\rm as} \;\;\;\; |y| \to  \infty. 
  \tag{\ref{eq:BC_for_Pe1}}
 \end{align*}
%%
を満たすものとする. 

今, 境界条件\eqref{eq:BC_for_Pe1}の 2 つ目を満たすような解として, 
%%
\begin{equation}
  P_{\rm e1} = \int_{C} F(y, k) e^{k\tilde{z}} \; dk
\label{eq:P_e1_solution_form}
\end{equation}
%%
を考える. 
ただし, $C$は複素$k$平面上の経路であり, 後ほど決定する. 
また, 
%%
\begin{equation}
  {\rm Re} (k) \geq 0 \;\;\;\;\; {\rm on} \;\;\; C
\end{equation}
%%
でなければならない. 

次に, 核$F$が満たす方程式を求める. 
\eqref{eq:P_e1_solution_form}を\eqref{eq:EBLgoverneq_linear_Pe1}に代入すれば, 
%%
\begin{equation}
  \DP[2]{F}{y} + (k^2 - iy) F = B(k,y)
\label{eq:ODE_for_F}
\end{equation}
%%
を得る. 
ここで, 
\begin{equation}
  \int_C B(k,y) \; dk = 0.
\end{equation}
%%

さらに, 
%%
\begin{equation}
  F(y,k) = \int_D \tilde{F} (l,k) e^{-ily} \; dl
\label{eq;F_solution_form}
\end{equation}
%%
なる形式の$F(y,k)$を求める. 
ここで, $D$は複素$l$面上の経路であり後ほど決定される. 
\eqref{eq::F_solution_form}を\eqref{eq:ODE_for_F}に代入し, 
部分積分を実行すれば, 
%%
\begin{equation}
  \int_D \big[ (-l^2 + k^2) \tilde{F} + - \DP{\tilde{F}}{l} \big] e^{-ily} \; dl
  = - \left[\tilde{F} e^{-ily} \right]_D + B(k,y)
\label{eq:F_fourierCoef_1}
\end{equation}
%%
を得る. 
今, 右辺がゼロとなるように, 経路$D$を取ることにすれば, 
%%
\begin{equation}
 \DP{\tilde{F}}{l} = (k^2 - l^2) \tilde{F}
\label{eq:F_fourierCoef_ODE}
\end{equation}
%%
を得る. 
ただし, 
%%
\begin{equation}
  \left[ \tilde{F} e^{-ily} \right]_D = B(k,y)
\label{eq:relation_with_contour_D_and_B}
\end{equation}
%%
でなければならない. 

\eqref{eq:F_fourierCoef_ODE}を解けば, 
%%
\begin{equation}
  \tilde{F} = C(k) \exp \big[ -\dfrac{l^3}{3} + k^2 l \big]. 
\end{equation}
%%
ここで, $C(k)$は$k$の任意関数である. 
これを, \ref{eq:relation_with_contour_D_and_B}に代入すれば, 
%%
\begin{equation}
  C(k) \left[ \exp \big( -\dfrac{l^3}{3} + k^2 l - ily \big) \right]_D = B(k,y)
\end{equation}
%%
を得る. 
さらに, 
%%
\begin{equation}
  \left[ \exp \big( -\dfrac{l^3}{3} + k^2 l - ily \big) \right]_D = 1
\label{eq:contour_D_constraint}
\end{equation}
%%
を満たすように, 経路$D$を選ぶことによって, 任意関数$C(k)$は, 
%%
\begin{equation}
  C(k) = B
\end{equation}
%%
と決まる. 
したがって, $B$は$y$には依存しない. 
なお, \eqref{eq:contour_D_constraint}を満たす$D$は正の実軸である. 
以上より, 
%%
\begin{equation}
  P_{\rm e1} = - \int_C \int_{l=0}^\infty B(k) 
                  \exp \big( -\dfrac{l^3}{3} + k^2 l - ily + k\tilde{z} \big) \; dl dk
 \label{eq:P_e1_solution_applyBC2}
\end{equation}
%%
を得る. 

次に, 境界条件\eqref{eq:BC_for_Pe1}の 3 つ目を考慮することを考える. 
\eqref{eq:P_e1_solution_applyBC2}に対して, $\tilde{z}$の一階微分をとり, 
さらに部分積分を実行すれば, 
%%
\begin{equation}
\begin{split}
  \DP{P_{\rm e1}}{\tilde{z}} &= 
                  - \int_C \int_{l=0}^\infty k B(k) 
                    \exp \big( -\dfrac{l^3}{3} + k^2 l - ily + k\tilde{z} \big) \; dl dk \\
  &= \int_C \left\{ \left[ 
              \dfrac{k B(k) e^{k\tilde{z}}}{iy - k^2 + l^2} \exp \big( -\dfrac{l^3}{3} + k^2l -ily \big)
             \right]_{l=0}^\infty \right\} dk + {\rm O}(y^{-2})
\end{split} 
\label{eq:dP_e1dz_form1}
\end{equation}
%%
を得る. 
今, $|y| \to \infty$かつ$C$上で${\rm Re}(k^2) < 0$の場合を考えるとき, 
\eqref{eq:dP_e1dz_form1}は, 
%%
\begin{equation}
  - \int_C \dfrac{k B(k) e^{k\tilde{z}}}{iy - k^2} \; dk
\label{eq:dP_e1_dz_form1_yinfty}
\end{equation}
%%
と漸近的に等しくなる. 
\eqref{eq:dP_e1_dz_form1_yinfty}をさらに変形すれば, 
%%
\begin{equation}
  - \int_C \dfrac{k B(k) e^{k\tilde{z}}}{iy - k^2} \; dk
 = \int_C \left[ \dfrac{1}{\sqrt{iy} + k} - \dfrac{1}{\sqrt{iy} - k}\right] \dfrac{B(k)e^{k\tilde{z}}}{2} \; dk  
\end{equation}
%%
である. 
したがって, 
今, 経路$C$は${\rm Re}(k^2) < 0$かつ${\rm Re}(k) \geq 0$



