\section{パラメータの設定例}\label{model_ogcm_parameters}
%%
海洋モデルのパラメータ設定の例として, リスト\cite{dogcm_Pl64L60_conf}に, 
\cite{ykawai2018_Dthesis}で行なった海惑星の気候実験における設定を示す. 
なお, 海洋海氷モデルに入力する設定ファイルの中で, 海洋モデルと関係する部分のみを以下では示している.
%%
\begin{lstlisting}[caption={
海洋モデルの設定ファイルのテンプレート.
ただし, \#var\#のように書かれた部分は実験ケース等ごとに適切な値が入る.
},label={dogcm_Pl64L60_conf},
basicstyle=\ttfamily\footnotesize,
frame=single]
&DOGCM_nml
  OCN_do  = .true.,
  SICE_do = .true.,
  exp_name = '#DOGCM_nml_EXP_NAME#',
/
!= 各過程で使用するスキームの選択 -----------------------------
&GovernEq_nml
  DynEqTypeName  = 'HydroBoussinesq',  ! 力学
  EOSTypeName    = 'EOS_JM95',         ! 海水の状態方程式
  LPhysNames     = 'LMixMOM, LMixTRC, RediGM',  ! 水平方向の物理
  VPhysNames     = 'VMixMOM, VMixTRC, Convect', ! 鉛直方向の物理 
  SolverTypeName = 'HSPM-VFVM',        ! 空間スキーム
/
!= 境界条件の設定 ----------------------------------------
&BoundaryCondition_nml
  KinBCSurface   = 'Rigid',
  DynBCSurface   = 'SpecStress',
  DynBCBottom    = 'NoSlip',
  ThermBCSurface = '#BoundaryCondition_nml_ThermBCSurface#',
    ! 結合 run では PrescFlux, 
    ! 海洋海氷モデル単体 run では PrescFlux_Han1984
  ThermBCBottom  = 'Adiabat', 
  SaltBCSurface  = 'PrescFlux',
  SaltBCBottom   = 'Adiabat', 
/
!= 物理定数の設定 ----------------------------------------
&Constants_nml
  RPlanet     = 6.37d6,
  Omega       = 7.29d-5, 
  Grav        = 9.8d0, 
  hViscCoef   = 3d5,
  vViscCoef   = 1d-3,
  hDiffCoef   = 0d3,
  vDiffCoef   = 3d-5, 
  albedoOcean = 0d0,
  LatentHeat  = 2.4253d6,
  RefSalt     = 35d0, 
/
!= 格子の設定 ----------------------------------------
&Grid_nml
 IM = 1,   ! 経度格子点数.
 JM = 64,  ! 緯度格子点数. 
 KM = 60,  ! 鉛直層数.
/
&Grid_Spm_nml
 NM = 63,  ! 最大全波数.
/
!= 周期的同期結合と関連した処理や実験ケースの設定 -----------
&Exp_APECoupleClimate_nml
 RunCycle            =  #Exp_APECPLClim_nml_RunCycle#, 
 RunTypeName         = '#Exp_APECPLClim_nml_RunTypeName#', 
     ! 結合 run では Coupled, 
     ! 海洋海氷モデル単体 run では Standalone
 SfcBCDataDir        = '#Exp_APECPLClim_nml_SfcBCDataDir#', 
 SfcBCMeanInitTime   = #Exp_APECPLClim_nml_SfcBCMeanIniTime#, 
 SfcBCMeanEndTime    = #Exp_APECPLClim_nml_SfcBCMeanEndTime#,
 RestartDataDir      = '#Exp_APECPLClim_nml_RestartDataDir#', 
 RestartMeanInitTime =  #Exp_APECPLClim_nml_RestartMeanInitTime#, 
 RestartMeanEndTime  =  #Exp_APECPLClim_nml_RestartMeanEndTime#, 
 OcnInitSalt         = 35d0, [psu]
/
!= 鉛直混合の設定 -----------
&SGSPhys_VMixing_nml
  VMixCoef_scheme_name = 'Simple'  
/
!= 水平混合の設定 -----------
&RediGM_nml   ! Redi, GM スキーム
  DiffCoefRedi    = 800.0,
  DiffCoefGM      = 800.0,
  InteriorTaperingName = 'DM95'
  SlopeMax        = 4d-3,
  Sd              = 1d-3, 
  PBLTaperingName = 'LDD97'
/
&LPhys_DIFF_nml  ! 水平超粘性
  NumDiffOrdH     = 4,     ! 超粘性の次数. 
  NumDiffTimeVal  = 50.0,  ! 最大波数に対する e-folding time
  NumDiffTimeUnit = 'day', ! e-folding time の単位 
/
!= 時間スキーム・時間管理の設定 -----------
&TemporalInteg_nml
  cal_type             = 'noleap', 
  barocTimeIntModeName = 'TimeIntMode_LFAM3', ! 時間スキーム
  DelTimeVal           = #TemporalInteg_nml_DelTimeHour#,
       ! 基本的に, 結合runでは 4 h, 海洋海氷モデル単体runでは 12 h 
  DelTimeUnit          = 'hour',
  ReStartTimeVal       = #TemporalInteg_nml_RestartTimeVal#, 
  ReStartTimeUnit      = 'day', 
  InitYear = #TemporalInteg_nml_InitYear#, InitMonth=1, InitDay=1, InitHour=0,
  InitMin  = 0, EndYear =#TemporalInteg_nml_EndYear#, EndMonth =1,
  EndDay   = #TemporalInteg_nml_EndDay#, EndHour=0, EndMin=0,
/
!* セミ・インプリシットスキームに含まれる係数の設定 -----------
&SemiImplicitScheme_nml
  VDiffTermACoef    = 0.5d0,
  CoriolisTermACoef = 0.5d0, 
/
!= ファイル関連の設定 ----------------------------------------
!* ヒストリデータ出力の全体設定 
&gtool_historyauto_nml
  IntValue      = #gtool_historyauto_nml_IntValue#,  ! 出力間隔の数値
  IntUnit       = 'day',                             ! 出力間隔の単位
  OriginValue   = #gtool_historyauto_nml_OriginValue#,  
  OriginUnit    = 'day',                             
  TerminusValue = #gtool_historyauto_nml_TerminusValue#,  
  TerminusUnit  = 'day',                             
  FilePrefix    = '',
/
!* リスタートデータの入力・出力の設定
&OGCM_IO_Restart_nml
 OutputFileName = '#OcnRestartFile_nml_OutputFileName#', 
 InputFileName  = '#OcnRestartFile_nml_InputFileName#', 
 IntValue       = #OcnRestartFile_nml_IntValue#, 
 IntUnit        = 'day'
/
!* ヒストリデータ出力の個別設定
&gtool_historyauto_nml
  Name = 'U, V, OMG, PTemp, Salt, H, HydPres, ConvIndex', 
  Precision='float'
/
&gtool_historyauto_nml
  Name = 'SfcPres, SfcHFlxO, SfcHFlx_ns, SfcHFlx_sr, DSfcHFlxDTs, FreshWtFlxS', 
  Precision='float', 
  TimeAverage = .true.
/
&gtool_historyauto_nml
  Name = 'a2o_WindStressX, a2o_WindStressY, a2o_SenHFlx, a2o_LatHFlx,a2o_LDwRFlx, a2o_LUwRFlx, a2o_SDwRFlx, a2o_SUwRFlx, a2o_DSfcHFlxDTs,a2o_RainFall, a2o_SnowFall, a2o_Evap, a2s_Evap, a2o_SfcHFlxMod', 
  Precision = 'double', 
  TimeAverage = .true.
/
\end{lstlisting}
