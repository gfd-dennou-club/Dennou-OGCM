\section{パラメータの設定例}
%%
海氷モデルのパラメータ設定の例として, リスト\cite{dogcm_Pl64L60_conf_sice}に, 
\cite{ykawai2018_Dthesis}で行なった海惑星の気候実験における設定を示す. 
なお, 海洋海氷モデルに入力する設定ファイルの中で, 海氷モデルと関係する部分のみを以下では示している.
%%%%%%
\begin{lstlisting}[caption={
海氷モデルの設定ファイルのテンプレート.
ただし, \#var\#のように書かれた部分は実験ケース等ごとに適切な値が入る.
},label={dogcm_Pl64L60_conf_sice},
basicstyle=\ttfamily\footnotesize,
frame=single]
&DOGCM_nml
  OCN_do  = .true.,
  SICE_do = .true.,
  exp_name = '#DOGCM_nml_EXP_NAME#',
/
!= 格子の設定 ----------------------------------------
&SeaIce_Grid_nml
 IM = 1,
 JM = 64
/
!= 物理定数の設定 ----------------------------------------
&SeaIce_Admin_Constants_nml
 AlbedoOcean   = 0d0, 
 EmissivOcean  = 1d0,
 EmissivSnow   = 1d0,
 EmissivIce    = 1d0,
 IceMaskMin    = 0.999999999999999999999999999999d0,
 IceThickMin   = 1d-2, 
 SIceHDiffCoef = #SeaIce_Admin_Constants_nml_SIceHDiffCoef#d0, 
                 ! 海氷厚さの水平拡散係数
/
!= 表面アルベドの計算方法の設定 ----------------------------------------
&SeaIce_Boundary_SfcAlbedo_nml
 SIceSfcAlbedoName = 'I07SGS',
 AlbedoSnowWarm    = 0d0,
 AlbedoSnowCold    = 0.5d0,
 TLVTempAlbedoSnow = -10d0, 
 SFCALBEDO_I07SGS_FlagIceAreaFrac = .true.,  
                       ! 表面アルベドの工夫を適用するかのフラグ
/
!= ファイル関連の設定 ----------------------------------------
!* ヒストリデータ出力の全体設定 
&SeaIce_IO_History_nml
  IntValue      = #gtool_historyauto_nml_IntValue#,  ! 出力間隔の数値
  IntUnit       = 'day',                             ! 出力間隔の単位
  FilePrefix    = '',
/
!* リスタートデータの入力・出力の設定
&SeaIce_IO_Restart_nml
 OutputFileName = '#SIceRestartFile_nml_OutputFileName#', 
 InputFileName  = '#SIceRestartFile_nml_InputFileName#', 
 IntValue       = #SIceRestartFile_nml_IntValue#, 
 IntUnit        = 'day'
/
!* ヒストリデータ出力の個別設定
&SeaIce_IO_History_nml
   Name = 'SIceEn, SIceV, SIceCon, SnowThick, IceThick, SIceTemp, SIceSfcTemp,SfcAlbedoAI', 
/
&SeaIce_IO_History_nml
  Name = 'Wice, SfcHFlxAI, DelSfcHFlxAI, SfcHFlxAI0_ns, SfcHFlxAI0_sr, DSfcHFlxAIDTs, SfcHFlxAO, DSfcHFlxAODTs', 
  TimeAverage = .true.
\end{lstlisting}
