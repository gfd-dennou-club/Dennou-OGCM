\documentclass[a4j,12pt,openbib,oneside]{jreport}
\input{version}
\title{{\LARGE Dennou-Clube Planetary Ocean Model \\
              (DCPOM version \version) \\
               支配方程式系とその離散化手法 }}
\author{河合 佑太}
\date{\today}

\usepackage[dvipdfmx]{graphicx}
\usepackage[dvipdfmx]{hyperref}
\usepackage{amsmath}
\usepackage{mathrsfs}
\usepackage{Dennou6}		% 電脳スタイル ver 6
\usepackage{siunitx}
\usepackage[round]{natbib}
\usepackage{bm}
\usepackage{url}
\usepackage{pxjahyper}

\hypersetup{% options for hyperref
 bookmarksopen=true,
 bookmarksnumbered=true,
 colorlinks=true,
% linkcolor=red,
 linkcolor=cyan,
 citecolor=cyan,
 urlcolor=cyan,
}
\usepackage{listings,jlisting} 
\renewcommand{\lstlistingname}{リスト}
\lstset{
  breaklines=true,
}
\usepackage{natbib}
\bibliographystyle{ametsoc}
\bibpunct[,]{(}{)}{;}{a}{}{,}

\def\thesection{\arabic{chapter}.\arabic{section}}
\def\thesubsection{\arabic{chapter}.\arabic{section}.\arabic{subsection}}
\def\theequation{\arabic{chapter}.\arabic{equation}}
\def\thepage{\arabic{page}}
\def\thefigure{\arabic{chapter}.\arabic{figure}}
\def\thetable{\arabic{chapter}.\arabic{table}}
\def\thefootnote{*\arabic{footnote}}



%%%%%%%%%%%%%%%%%%%%%%%%%%%%%%%%%%%%%%%%%%%%%%%%%%%%%%%%
%%%%%%%%   Set Counter (chapter, section etc. ) %%%%%%%%
\setcounter{chapter}{0}    % 章番号
\setcounter{section}{1}    % 節番号
\setcounter{subsection}{1}    % 節番号
\setcounter{equation}{0}   % 式番号
\setcounter{page}{0}     % 必ず開始ページは明記する
\setcounter{figure}{0}     % 図番号
\setcounter{table}{0}      % 表番号
%\setcounter{footnote}{0}

%%%%%%%%%%%%%%%%%%%%%%%%%%%%%%%%%%%%%%%%%%%%%%%%%%%%%%%%
%%%%%%%%        Dennou-Style Definition         %%%%%%%%

%% 改段落時の空行設定
\Dparskip      % 改段落時に一行空行を入れる
%\Dnoparskip    % 改段落時に一行空行を入れない

%% 改段落時のインデント設定
\Dparindent    % 改段落時にインデントする
%\Dnoparindent  % 改段落時にインデントしない

%%%%%%%%%%%%%%%%%%%%%%%%%%%%%%%%%%%%%%%%%%%%%%%%%%%%%%%%%%
%% Macro defined by author
\def\univec#1{ \hat{ \Dvect{\rm #1}} }


%%%%%%%%%%%%%%%%%%%%%%%%%%%%%%%%%%%%%%%%%%%%%%%%%%%%%%%%
%%%%%%%%             Text Start                 %%%%%%%%
%%
\begin{document}
\maketitle
\tableofcontents

\chapter{はじめに}
%!TEX root = ./dcpom_description.tex
\section{この文書について}
地球流体電脳倶楽部では, 海洋大循環を考慮した惑星気候の探索を目指して, 
海洋大循環モデルと海氷モデルを開発しており, 
Dennou-Club Planetary Ocean Model (DCPOM) として公開している%
\footnote{
\url{https://www.gfd-dennou.org/arch/index.htm}からソースコード等を取得できる. 
}. 
本書では, DCPOM に含まれる海洋モデルと海氷モデルの支配方程式系や離散化手法を記述する. 

現状, 本書の記述と実際のコードが一致しない所もあるので, ご了承頂きたい. 

\section{DCPOM の概要}
ここでは, DCPOM version \version \; (2018年10月2日版)の概要を記述する. 

本書で主に対象とするのは, \verb|${TOP_DIR}/model/dogcm|ディレクトリ中にある, 海洋大循環モデルと海氷モデルのコードである. 
この海洋大循環モデルで表現される過程と数値解法は, 以下にまとめられる。
\begin{itemize}
  \item 力学コア: 
  \begin{itemize}
    \item 浅い流体近似・静水圧近似をしたブジネスク方程式系
  \end{itemize}
  \item サブグリッド・スケールの過程のパラメタリゼーション
    \begin{itemize}
      \item メソスケールの渦による混合
      \begin{itemize}  
        \item 等密度面に沿った拡散スキーム \citep{redi1982oceanic} 
        \item skew flux に基づく GM スキーム \citep{gent1990isopycnal,griffies1998gent}
      \end{itemize} 
      \item 対流による鉛直混合
      \begin{itemize}  
        \item 対流調節スキーム \citep{marotzke1991influence,rahmstorf1993fast}
      \end{itemize}   
    \end{itemize} 
  \item 数値解法
  \begin{itemize}
    \item 水平離散化: 球面調和関数展開に基づくスペクトル法
    \item 鉛直離散化: 有限体積法
    \item 時間離散化
    \begin{itemize}
      \item 順圧モードと傾圧モードの分離 \citep{shchepetkin2005regional}
      \item 移流項: LF-AM3 スキーム \citep{shchepetkin2005regional}
  	  \item 鉛直拡散項, 伝播速度の速い波と関係した項: リープフロッグ法
  	  \item 水平拡散項: 前進オイラー法または後退オイラー法
    \end{itemize}                  
  \end{itemize}
\end{itemize}
また, 海氷モデルで表現される過程とその数値解法は, 以下にまとめられる。
\begin{itemize}
  \item 熱力学過程: \citep{winton2000reformulated}に基づく三層モデル
  \item 水平輸送のパラメタリゼーション: 海氷厚さの水平拡散で表現
  \item 数値解法
  \begin{itemize}
    \item 水平離散化: 有限体積法
    \item 鉛直離散化: 有限体積法
    \item 時間離散化
    \begin{itemize}  
      \item 移流項: LF-AM3 スキーム \citep{shchepetkin2005regional}
  	  \item 鉛直拡散項: リープフロッグ法
  	  \item 水平拡散項: 前進オイラー法
    \end{itemize}                  
  \end{itemize}
\end{itemize}

なお, \verb|${TOP_DIR}/model|内には, 上記のモデルのディレクトリ(dogcm)以外にも, 
「globalSWModel\_DG, globalSWModel\_FVM, ogcm」等のディレクトリが存在する. 
これらは, 以下のように岸が取り扱うことを念頭にした海洋モデルの試作品であるが, 
これらのモデルの詳細は, 本書においてまだ記述できていない.  

\section{本書の構成}
%%
第\ref{ocn_model}章において, 海洋大循環モデルの支配方程式系と, 
それらの離散化手法を記述する. 本章の最後に, 海洋モデルに与えるパラメータの設定例も示す. 
第\ref{sice_model}章において, 海氷モデルの支配方程式系と, 
それらの離散化手法を記述する. 本章の最後に, 海氷モデルに与えるパラメータの設定例も示す. 





\chapter{海洋大循環モデル}
%\section{海洋大循環モデルの詳細}
\label{ocn_model}
\markright{\arabic{chapter}.\arabic{section} 海洋大循環モデル} %  節の題名を書き込むこと
%%
本章では, 海洋大循環モデルの詳細を記述する. 
なお, \cite{ykawai2018_Dthesis}の数値実験では本海洋モデルを軸対称モデルとして使用したが, 
以下に記述するように, 実際には汎用性のために三次元モデルとして定式化を行い, 実装している. 

%% モデルの基礎
\section{モデルの基礎}
\markright{\arabic{chapter}.\arabic{section} モデルの基礎} %  節の題名を書き込むこと
%%
将来的にモデルを拡張することを念頭に, 水平方向は局所直交座標系, 
鉛直方向は一般化した鉛直座標系を用いてモデルの方程式系を記述する. 
はじめに, モデルの基礎方程式系をベクトル形式で示す. 
次に水平座標系が局所直交座標系, 鉛直座標が幾何的座標($z$座標)の場合の方程式系を記述する.  
その後, 一般化した鉛直座標系($s$座標)を導入して, $s$座標を用いた場合の表現を得ることにする. 

\subsection{基礎方程式系の導入}
%%
海洋大循環モデルの基礎方程式系は, 静力学ブジネスク方程式系(プリミティブ方程式系)であり, 
運動量方程式, 連続の式, 温位と塩分の保存式, 海水の状態方程式により構成される. 
今, 等ジオポテンシャル面と直交する単位ベクトルを$\Dvect{k}$, 
等ジオポテンシャル面と接する単位ベクトルを$(\Dvect{i},\Dvect{j})$として, 
局所直交座標系$(\Dvect{i}, \Dvect{j}, \Dvect{k})$を導入する. 
また, 三次元速度ベクトル$\Dvect{U}$は$\Dvect{U}=\Dvect{U}_h + w \Dvect{k}$のように水平成分と鉛直成分に分割する. 
このとき, 静力学ブジネスク方程式系のベクトル不変形式は以下のように書かれる. 
%%
\begin{subequations} 
\label{eq:OCN_primitive_eq_vector_form}
\begin{gather}
 \DP{\Dvect{U}_h}{t} = 
   - \left[ (\nabla \times \Dvect{U}) \times \Dvect{U} + \nabla \left(\dfrac{\Dvect{U}^2}{2}\right) \right]_h  
   - f \Dvect{k} \times \Dvect{U}_h 
   - \dfrac{1}{\rho_0} \nabla_h p  + \Dvect{\mathcal{\tilde{D}}}^{\bm{U}} , 
\label{eq:horizontal_mom_eq} \\
%%
 \DP{p}{z} = - \rho g, 
 \label{eq:hydrostatic_eq} \\
%%
 \nabla \cdot \Dvect{U} = 0, 
 \label{eq:continuous_eq} \\
%%
 \DP{\Theta}{t} = - \nabla\cdot(\Theta\Dvect{U}) + \mathcal{\tilde{D}}^\Theta , \\
%%
 \DP{S}{t} = - \nabla\cdot(S\Dvect{U}) + \mathcal{\tilde{D}}^S, \\
%%
 \rho = \rho(\Theta, S, p). 
%%
\end{gather}  
\end{subequations}
%%
ここで, $\nabla$は三次元勾配ベクトル, $t$は時刻, $z$は鉛直座標, $\rho$は海水の密度, 
$\rho_0$は基準密度, $p$は圧力, 
$f=2\Dvect{\Omega}\cdot\Dvect{k}$はコリオリパラメータ($\Dvect{\Omega}$は自転角速度ベクトル), 
$g$は重力加速度である. 
また, $\Dvect{\mathcal{\tilde{D}}}^{\bm{U}}, \mathcal{\tilde{D}}^\Theta, \mathcal{\tilde{D}}^S$は, 
中規模渦による水平混合, 小スケールの渦や対流による鉛直混合などの, 
サブグリッド・スケールの物理を表現するパラメタリゼーションの項を表す. 

海面における運動学的な境界条件は, 
%%
\begin{subequations}
\begin{equation}
  w = \DP{\zeta}{t} + \Dvect{U}_h \cdot \nabla_h \eta + (P - E) \;\;\; {\rm at} \; z = \eta, 
\label{eq:ocn_surface_bc}
\end{equation}
%%
である. 
ここで, $\eta$は海水表面の変位である. また, $(P-E)$は淡水フラックスであり, 降水量と蒸発量の差を表す. 
一方で, 海底面における運動学的な境界条件は, 
\begin{equation}
  w = - \Dvect{U}_h \cdot \nabla_h H \;\;\; {\rm at} \; z = -H
\label{eq:ocn_bottom_bc}
\end{equation}
\label{eq:ocn_vertical_bc}
\end{subequations}
%%
である. 
ここで, $H$は平均的な海面($z=0$)から海底までの距離である. 

\subsection{圧力の定式化}
%%
ある深さ$z$における圧力$p$は, 大気圧$p_a$, 
基準となる海水面($z=0$)に対する表面の変動と関係した圧力$p_s$, 
静水圧$p_h$の三つの寄与に分割され, 
%%
\begin{align*}
 p(i,j,z,t) 
 &= p_a + p_s(i,j,t) + p_h(i,j,z,t)  \\
 &= p_a + (\rho_0 g\eta) + (p_h - \rho_0 gz)
\end{align*}
%%
と書かれる. 
ここで, $p_h$は, 静水圧平衡の式\eqref{eq:hydrostatic_eq}から, 
%%
\begin{equation*}
  p_h(i,j,z,t) = \int^\eta_z g \left[\rho(\Theta,S, -\rho_0 g z') - \rho_0 \right] \Dd{z'}
\end{equation*}
%%
と求まる. 
ただし, ブジネスク方程式系においてエネルギー論の一貫性を保つには, 
状態方程式に現れる圧力は深さのみに依存する形式で書く必要があるため, 
$p \approx -\rho_0 gz$と近似している.
表面圧力$p_s$の計算には, 以下の 2 通りの方法がある. 
%%
\begin{enumerate}
  \setlength{\itemsep}{-0.5mm} % 項目の隙間
  \setlength{\parskip}{-0.5mm} % 段落の隙間
 \item 海水面の変位$\eta$を新たな予報変数として導入し, $\eta$(すなわち$p_s=\rho_0 g \eta$)の時間発展式を解く. 
 \item 海面において剛体蓋近似を適用し, $p_s$は診断的に決定する. 
\end{enumerate}
%%
前者の方法において, $\eta$の時間発展式は連続の式を鉛直積分することで得られる. 
海面の変位を許容する場合には表面波を表現できるが, その位相速度は海洋大循環
と比べるとずっと速い(数百 m/s)ために, 時間刻み幅に強い制約を与える. 
後者の場合では, 鉛直積分した水平速度が非発散であるという拘束条件を用いて, 
表面圧力に関する楕円方程式を導くことができる. 海面の変位は許されないために, 
表面波は解として含まれない. 剛体蓋近似の適用は, 表面波(外部重力波)だけでなく
順圧ロスビー波等の長波にも影響を与えることに注意が必要である. 

\subsubsection*{ \underline{方法 1 の場合: 表面変位$\eta$の時間発展式} }
%%
連続の式\eqref{eq:continuous_eq}を, 海底$z=-H$から海面$z=\eta$まで鉛直積分すると, 
%%
\begin{equation*}
  w_{z=\eta} = w_{z=-H} - \int_{z=-H}^{z=\eta} \nabla_h \cdot \Dvect{U}_h \; dz
\end{equation*}
%%
となる. 
海面および海底での運動学的境界条件\eqref{eq:ocn_vertical_bc}を適用すると, 
$\eta$の時間発展式として, 
%%
\begin{equation}
  \DP{\eta}{t} = - \nabla_h \cdot \int_{z=-H}^{z=\eta} \Dvect{U}_h \; dz
                 + (P - E)
\label{eq:free_surface_flux_eq}
\end{equation}
%%
を得る. 

\subsubsection*{ \underline{方法 2 の場合: 表面圧力$p_s$の診断方程式} }
%%
剛体蓋近似を適用する場合には, $w=D\eta/Dt=0$であることを課すので, 
$\eta$は常に定数(ここでは, ゼロとする)である. したがって, 
\eqref{eq:free_surface_flux_eq}は, 
%%
\begin{equation}
 \nabla_h \cdot \int_{z=-H}^{z=0} \Dvect{U}_h = 0
\label{eq:rigid_lid_constraintment_horimom}
\end{equation}
%%
となり, 鉛直積分した水平速度は非発散でなければならない. 
次に, 水平方向の運動量方程式\eqref{eq:horizontal_mom_eq}を$z=-H$から$z=0$まで鉛直積分し, 
水平発散をとった後に, \eqref{eq:rigid_lid_constraintment_horimom}を適用すれば, 
%%
\begin{equation}
 \nabla_h \cdot ( H \nabla_h \; p_s) = \int_{-H}^0 F_b \; dz
\label{eq:rigid_lid_pressure_poisson}
\end{equation}
%%
が得られる. 
ここで, $F_b$は, 表面圧力勾配の項(と局所時間微分の項)を除く全ての項の寄与を表す. 
この表面圧力に対する楕円方程式を解くことにより, 表面圧力を決定できる. 

\subsection{水平直交曲線座標・$z$座標における基礎方程式系の表現}
%%
水平直交曲線座標・鉛直$z$座標の座標変数$(i,j,z)$は, 
地理座標系の座標変数と
%%
\begin{equation*}
 i = i(\lambda,\phi), \;\; j = j(\lambda,\phi) \;\; z=z
\end{equation*}
%%
によって関係付けられるとする. 
ここで, $\lambda$は経度, $\phi$は緯度, $z$は平均海面水位からの高度である. 
それぞれの方向の単位ベクトルは前に定義した$\Dvect{i}, \Dvect{j}, \Dvect{k}$であり,  
これらは局所的に直交することに注意が必要である. 
このとき, 水平方向のスケール因子$e_1, e_2, e_3$は, 
%%
\begin{equation*}
\begin{split}
  e_1 &= a \left[ \left(\DP{\lambda}{i} \cos\phi \right)^2 + \left(\DP{\phi}{i}\right)^2 \right]^{1/2}, \\ 
  e_2 &= a \left[ \left(\DP{\lambda}{j} \cos\phi \right)^2 + \left(\DP{\phi}{j}\right)^2 \right]^{1/2}
\end{split}
\end{equation*}
%%
によって与えられる. 
惑星半径に対して海洋が十分に浅いモデルを考えるので, もとのスケール因子に含まれる$(a+z)$は$a$に置き換えた. 

導入した座標系$(i,j,k)$において, 
モデルの基礎方程式系\eqref{eq:OCN_primitive_eq_vector_form}は以下のように書かれる. 
%%
\begin{subequations} 
\label{eq:OCN_basic_equations_z_coord}
\begin{equation}
\begin{split}
  \DP{u}{t} 
  =& + (\zeta + f)v -  \dfrac{1}{e_1} \DP{}{i} \left( \dfrac{u^2 + v^2}{2} \right) 
     - w\DP{u}{z}            \\
   & - \dfrac{1}{e_1} \DP{}{i} \left( \dfrac{p_s + p_h}{\rho_0} \right) 
     + \mathcal{\tilde{D}}^u, 
\end{split}
\end{equation}
%%
\begin{equation}
\begin{split}
  \DP{v}{t} 
   =& - (\zeta + f)u -  \dfrac{1}{e_2} \DP{}{j} \left( \dfrac{u^2 + v^2}{2} \right) 
      - w\DP{v}{z}             \\
    & - \dfrac{1}{e_2} \DP{}{j} \left( \dfrac{p_s + p_h}{\rho_0} \right) 
      + \mathcal{\tilde{D}}^v, 
\end{split}
\end{equation}
%%
\begin{equation}
  \DP{p_h}{z} = - (\rho - \rho_0)  g, 
\end{equation}
%%
\begin{equation}
  \DP{w}{z} = - \chi \\
\end{equation}
%%
\begin{equation}
  \DP{\Theta}{t} 
  =   
    - \dfrac{1}{e_1 e_2}\left[ \DP{(e_2 u \Theta)}{i} + \DP{(e_1 v \Theta)}{j} \right]
    - \DP{(\Theta w)}{z}
    + \mathcal{\tilde{D}}^\Theta, 
\end{equation} 
%%
\begin{equation}
  \DP{S}{t} 
  =   
    - \dfrac{1}{e_1 e_2}\left[ \DP{(e_2 u S)}{i} + \DP{(e_1 v S)}{j} \right]
    - \DP{(S w)}{z}
    + \mathcal{\tilde{D}}^S, 
\end{equation}
%%
\begin{equation}
  \rho_o = \rho_o(\Theta,S,z). 
\end{equation}
%%
\end{subequations}
%%
ここで, $(u,v)$は水平速度ベクトル$\Dvect{U}_h$の各方向成分, 
$\Dvect{\mathcal{\tilde{D}}}^{\bm{U}}=(\mathcal{\tilde{D}}^u, \mathcal{\tilde{D}}^v)$である. 
また, $\zeta$は相対渦度の鉛直成分, $D$は速度の水平発散であり, $z$座標系において, 
%%
\begin{equation}
 \zeta = \dfrac{1}{e_1 e_2}\left[ \DP{(e_2 v)}{i} - \DP{(e_1 u)}{j} \right], \;\;\;
 D  = \dfrac{1}{e_1 e_2}\left[ \DP{(e_2 u)}{i} + \DP{(e_1 v)}{j} \right]
\label{eq:relvor_hdiv_zcoord}
\end{equation}
%
と書かれる. 
なお, 表面圧力$p_s$は, \eqref{eq:rigid_lid_pressure_poisson}あるいは\eqref{eq:free_surface_flux_eq}から決定され, 
また大気圧の影響はここでは無視した. 

\subsection{水平直交曲線座標・一般鉛直座標における基礎方程式系の表現}
%%
現業の海洋モデルでは, 海底地形や海面変位を考慮するために様々な鉛直座標が用いられる. 
対象とする問題に応じて適切な鉛直座標が選択できるように, しばしば一般化した鉛直座標($s$座標)を用いて
定式化がなされる. 
今, $z$座標系の座標変数を$(i^*,j^*,z^*)$と$s$座標系の座標変数を$(i,j,s)$とすると, 
$z$座標系から$s$座標系への変換は, 
%%
\begin{equation*}
  i = i^*, \;\; j = j^*, \;\; s = s(i^*,j^*,z^*,t^*), \;\; t=t^*
\end{equation*}
%%
によって関係付けられる. 

次に, $z$座標系の方程式系から$s$座標系の方程式系への変換に必要な幾つかの関係式を導く.
物理量$\Psi$が, $z$座標系では$\Psi=\psi^*(i^*,j^*,z^*,t^*)$,
$s$座標系では$\Psi=\psi(i,j,s,t)$と表されるとすると, 
物理量$\Psi$の偏微分は, 微分の連鎖率によって, 
%%
\begin{equation}
\begin{split}
 \DP{\psi^*}{i^*} &= \DP{\psi}{i} + \DP{\psi}{s}\DP{s}{i^*} 
                   = \DP{\psi}{i} - \dfrac{e_1}{e_3}\sigma_1 \DP{\psi}{s}, \\
 \DP{\psi^*}{j^*} &= \DP{\psi}{j} + \DP{\psi}{s}\DP{s}{j^*} 
                   = \DP{\psi}{j} - \dfrac{e_2}{e_3}\sigma_2 \DP{\psi}{s}, \\
 \DP{\psi^*}{z^*} &= \DP{\psi}{s} \DP{s}{z^*} 
                   = \dfrac{1}{e_3} \DP{\psi}{s}, \\
 \DP{\psi^*}{t^*} &= \DP{\psi}{t} + \DP{\psi}{s}\DP{s}{t^*} 
                   = \DP{\psi}{t} + \DP{\psi}{s}\DP{s}{t}                   
\end{split}
\label{eq:derivative_chain_rule_scoord}
\end{equation}
%%
と変換される. 
ここで, $e_3$は鉛直方向のスケール因子, $\sigma_1, \sigma_2$は水平方向の$s$面の傾斜であり, 
%%
\begin{equation*}
 e_3 = \DP{z^*}{s}, \;\; 
 \sigma_1 = \dfrac{1}{e_1} \left(\DP{z^*}{i^*}\right)_s, \;\;
 \sigma_2 = \dfrac{1}{e_2} \left(\DP{z^*}{j^*}\right)_s
\end{equation*} 
%%
と定義される. 
一方, $\Psi$のラグランジュ微分は, \eqref{eq:derivative_chain_rule_scoord}を用いて, 
%%
\begin{equation*}
\begin{split}
 \DD{\psi^*}{t^*} &=   \DP{\psi^*}{t^*} 
                     + \dfrac{u}{e_1}\DP{\psi^*}{i^*}
                     + \dfrac{v}{e_2}\DP{\psi^*}{j^*}
                     + w\DP{\psi^*}{z^*} \\
                 &= \DP{\psi}{t} 
                     + \dfrac{u}{e_1}\DP{\psi}{i}
                     + \dfrac{v}{e_2}\DP{\psi}{j}
                     + \left(w + e_3 \DP{s}{t} - \sigma_1 u - \sigma_2 v \right) \dfrac{1}{e_3} \DP{\psi}{s}
\end{split}
\end{equation*}
%%
と変換される. 
したがって, $s$座標系における鉛直速度$\omega$は, 
%%
\begin{equation*}
  \omega \equiv \DD{s}{t^*} 
=  w - w_s - \sigma_1 u - \sigma_2 v
\end{equation*}
%%
によって与えられる. ここで, 
$w_s \equiv (\partial z^*/\partial t)_s = - e_3 (\partial s/\partial t)_{z^*}$
である.
最終的に, $\Psi$のラグランジュ微分は, 
%%
\begin{equation}
 \DD{\psi^*}{t^*} = \DP{\psi}{t} 
                     + \dfrac{u}{e_1}\DP{\psi}{i}
                     + \dfrac{v}{e_2}\DP{\psi}{j}
                     + \omega \DP{\psi}{s}
\label{eq:lagrange_derivative_scoord}
\end{equation}
%%
と書ける. 

$z$座標系における方程式系\eqref{eq:OCN_basic_equations_z_coord}や
鉛直渦度や水平発散の表現\eqref{eq:relvor_hdiv_zcoord}に対して, 
鉛直座標変換のための関係式\eqref{eq:derivative_chain_rule_scoord}, \eqref{eq:lagrange_derivative_scoord}を適用し整理すると, 以下の$s$座標系における方程式系を導かれる. \\\\
%%
\noindent
\framebox[15cm][c]{
\begin{minipage}{13cm}
\begin{subequations} 
\label{eq:OCN_basic_equations_s_coord}
\begin{equation}
\begin{split}
 \DP{u}{t} 
  =& + (\zeta + f)v -  \dfrac{1}{e_1} \DP{}{i} \left( \dfrac{u^2 + v^2}{2} \right) 
     - \dfrac{\omega}{e_3} \DP{u}{s}            \\
   & - \dfrac{1}{e_1} \DP{}{i} \left( \dfrac{p_h+p_s}{\rho_0} \right) 
     + g\dfrac{\rho'}{\rho_0} \sigma_1 
     + \mathcal{D}^u, 
\end{split}
\end{equation}
%%
\begin{equation}
\begin{split}
  \DP{v}{t} 
   =& - (\zeta + f)u -  \dfrac{1}{e_2} \DP{}{j} \left( \dfrac{u^2 + v^2}{2} \right) 
      - \dfrac{\omega}{e_3}\DP{v}{s}             \\
    & - \dfrac{1}{e_2} \DP{}{j} \left( \dfrac{p_s + p_h}{\rho_0} \right) 
     + g\dfrac{\rho'}{\rho_0} \sigma_2
     + \mathcal{D}^v, 
\end{split}
\end{equation}
%%
\begin{equation}
  \dfrac{1}{e_3}\DP{p_h}{s} = - \dfrac{\rho'}{\rho_0}  g, 
\end{equation}
%%
\begin{equation}
  \varepsilon_1 \DP{e_3}{t} + e_3 \underline{D} + \DP{\omega}{s} = 0, \\
\end{equation}
%%
\begin{equation}
\begin{split}
 \varepsilon_1 \DP{\eta}{t} 
  =   
    &- \dfrac{1}{e_1 e_2}\left[ 
    		  \DP{\left(e_2 (\varepsilon_2 \eta + H) U_{barot}\right)}{i} 
    		+ \DP{\left(e_1 (\varepsilon_2 \eta + H) V_{barot}\right)}{j} 
    \right]  \\
    	&+ \varepsilon_1 (P-E) 
\end{split}
\end{equation} 
%%
\begin{equation}
 \dfrac{1}{e_3}\DP{(e_3 \Theta)}{t} 
  =   
    - \dfrac{1}{e_1 e_2 e_3}\left[ \DP{(e_2 e_3 u \Theta)}{i} + \DP{(e_1 e_3 v \Theta)}{j} \right]
    - \dfrac{1}{e_3} \DP{(\Theta \omega)}{s}
    + \mathcal{D}^\Theta, 
\end{equation} 
%%
\begin{equation}
  \dfrac{1}{e_3}\DP{(e_3 S)}{t} 
  =   
    - \dfrac{1}{e_1 e_2 e_3}\left[ \DP{(e_2 e_3 u S)}{i} + \DP{(e_1 e_3 v S)}{j} \right]
    - \dfrac{1}{e_3} \DP{(S \omega)}{s}
    + \mathcal{D}^S, 
\end{equation}
%%
\begin{equation}
  \rho_o = \rho_o(\Theta,S,z(i,j,s,t)). 
\end{equation}
%%
\end{subequations}
\end{minipage}
}\\\\
%%
ここで, $\rho'(=\rho - \rho_0)$は参照密度からの偏差である. 
また, $\mathcal{D}^u, \mathcal{D}^v, \mathcal{D}^\Theta, \mathcal{D}^S$はそれぞれ, 
$\mathcal{\tilde{D}}^u, \mathcal{\tilde{D}}^v, \mathcal{\tilde{D}}^\Theta, \mathcal{\tilde{D}}^S$
の$s$座標系での表現である. 
$s$座標系の鉛直渦度と水平発散の表現は, $s$座標系において, 
%%
\begin{equation}
 \zeta = \dfrac{1}{e_1 e_2}\left[ \DP{(e_2 v)}{i} - \DP{(e_1 u)}{j} \right], \;\;\;
 \underline{D}  = \dfrac{1}{e_1 e_2 e_3}\left[ \DP{(e_2 e_3 u)}{i} + \DP{(e_1 e_3 v)}{j} \right]
\label{eq:relvor_hdiv_generalvcoord}
\end{equation}
%
となる. 
$\underline{D}$は, $e_3=e_3(z)$の場合を除いて, 前に示した水平発散の表現$D$とは一致しないことに注意されたい. 
$U_{barot}, V_{barot}$は, 鉛直平均した水平速度(順圧成分)であり, 
例えば, $U_{barot}$は, 
%%
\begin{align*}
 U_{barot} = \dfrac{1}{\varepsilon_2 \eta + H}
  \int_{s(z=-H)}^{s(z=\varepsilon_2 \eta)} (U e_3) \; \Dd{s}
\end{align*}
%%
によって定義される. 
$\varepsilon_1, \varepsilon_2$は海面(故に表面圧力)の扱いにより決まる指標であり, 
(i) 剛体蓋近似を適用する場合は$\varepsilon_1=\varepsilon_2=0$, 
(ii) 線形化した自由表面の扱いの場合は$\varepsilon_1=1, \varepsilon_2=0$,
(iii) 海面の扱いに関して近似をしない場合は$\varepsilon_1=\varepsilon_2=1$である. 
表面圧力$p_s$は,
(i)の場合は\eqref{eq:rigid_lid_pressure_poisson},
(ii)または(iii)の場合は\eqref{eq:free_surface_flux_eq}の方法に基づいて決定される. 
任意の時刻・位置における$z$の値は, $\eta$が既知であれば, 定義した鉛直座標変換の関係式から計算することができ,  
同時に鉛直方向のスケール因子$e_3(=\partial z/\partial s)$も計算できる. 

%%%%%%%%%%%%%%%%%%%%%%%%%%%%%%%%%%%%%%%%%%%%%%%%%%%%%%%%%%%%%%%%%%%%%%%%%%%%%%%%%%%%%%%%%%%%%%%%%%%
\subsection{サブグリッド・スケールの物理の表現}
%%
重力の存在は, 水平運動と鉛直運動の間に強い非等方性を生む. 
そのため, \eqref{eq:OCN_basic_equations_s_coord}における, 
サブグリッド・スケールの物理による寄与$D^q$(ただし, $q=u,v,\Theta,S$)を, 
水平運動に伴う寄与$\mathcal{D}^{lq}$と鉛直運動に伴う寄与$\mathcal{D}^{vq}$に分ける. 
ここでは, これらの項の形式を簡潔にまとめることに主眼を置き,  
各パラメタリゼーション・スキームの詳細については参考文献を参照されたい.  

\subsubsection*{\underline{鉛直方向の混合}}
%%
鉛直乱流混合を生じさせるシア不安定や内部波の砕破等の過程は, 
海洋大循環モデルで典型的な格子スケールよりもはるかに小さいため, 
その効果を取り入れるにはパラメータ化する必要がある. 
鉛直乱流フラックスは, 格子スケールの変数の空間勾配に線形に比例すると仮定し, 
$\overline{q'w'}= -A^{vq} \partial{\overline{q}}/\partial {z}$のように表現する. 
このとき, 運動量, トレーサーの鉛直乱流混合を表現する項は, 二階の空間微分を用いて, 
%%
\begin{equation*}
\begin{split}
 \Dvect{\mathcal{D}}^{v{\bm U}} 
    &= \dfrac{1}{e_3} \DP{}{s}\left( \dfrac{A^{vm}}{e_3} \DP{\Dvect{U}_h}{s} \right), \\
 \mathcal{D}^{v\Theta}  &= \dfrac{1}{e_3}\DP{}{s}\left( \dfrac{A^{vT}}{e_3} \DP{\Theta}{s} \right), \;\;
 \mathcal{D}^{vS}  = \dfrac{1}{e_3}\DP{}{s}\left( \dfrac{A^{vT}}{e_3} \DP{S}{s} \right)
\end{split}
\end{equation*}
%%
と書かれる. 
ここで, $A^{vm}, A^{vT}$はそれぞれ, 鉛直渦粘性係数, 鉛直渦拡散係数である. 
これらの係数として, (a)定数, (b)位置の関数, (c)局所的な流体の特性(安定度やリチャードソン数など)の簡単な関数, 
の三種類の場合がモデルでは想定されている. 
鉛直乱流混合をより良く表現するために海洋モデル業界では, K-profile パラメタリゼーション(KPP)スキーム
\citep{large1994oceanic}等の乱流クロージャーが用いられるが, 本モデルには今の所実装できていない. 

%%%%%%%%%%%%%%%%%%%%%%%%%%%%%%%%%%%%%%%%%%%%%%%%%%%%%%%%%%%%%%%%%%%%%%%%%%%%%%%%%%%%%%%%%%%%%%%%%%%
\subsubsection*{\underline{水平方向の混合}}
%%
水平方向の乱流混合は, メソスケールの乱流(直径数百 km 程度)による混合と, 
サブメソスケールの乱流(1$\sim$50 km 程度)による混合に大きくは分けられる. 
メソスケールの乱流は, 解像度が十分に高ければ, 静力学平衡を仮定する海洋大循環モデルにおいても陽に表現できる. 
しかしながら, 全球領域で長時間積分を行う場合に, メソスケールの渦を十分に表現できる解像度で計算することは今でも容易でない.  
そのため, 伝統的には, 等密度面混合スキーム\citep{redi1982oceanic}と 
GM スキーム\citep{gent1990isopycnal}といったパラメタリゼーションを用いる.
一方, サブメソスケールの乱流は, 海洋大循環モデルにおいて現実的な格子スケールよりも遥かに小さく, 
また静力学近似の適用限界も存在するために, モデルの中で陽に表現することは難しい. 
そのため, サブメソスケールの渦の効果をモデルに取り入れるためにはパラメタリゼーションを要する.  
以下では, トレーサーに対する二次の水平拡散, 四次の水平拡散, メソスケール渦による移流のパラメタリゼーション, 
水平運動量に対する二次の水平粘性, 四次の水平粘性に伴う項の形式をまとめる. 

はじめに, トレーサー$T$に対する二次の水平拡散の表現は,  
%%
\begin{equation}
 \mathcal{D}^{lT}_{\rm ldiff} = \nabla\cdot (A^{lT} \mathcal{R} \nabla T), 
 \;\;\;\; 
 \mathcal{R} =
\begin{pmatrix}
  1 &0 &-r_1              \\
  0 &1 &-r_2               \\
  -r_1 &-r_2 &r_1^2 + r_2^2
\end{pmatrix}
\label{eq:lateral_diffusion_expression}
\end{equation}
%%
である. 
ここで, $r_1, r_2$は, モデル座標系の水平面と拡散演算子が作用する面との間の傾斜を表す. 
\eqref{eq:lateral_diffusion_expression}は, 等ジオポテンシャル面に沿う拡散を想定するならば厳密な表現である. 
一方で, 等密度面に沿った拡散を想定するならば, 
現実の海洋内部領域における等密度面の傾斜が最大でも$10^{-2}$程度であることを利用した近似的な表現である\citep{redi1982oceanic}. 
$r_1, r_2$は, 等ジオポテンシャル面に沿った拡散に対しては, 
$r_1=\sigma_1, r_2=\sigma_2$であり, $z$座標系では$r_1=r_2=0$となる. 
また, 等密度面に沿った拡散に対しては, 
%%
\begin{equation*}
  r_1 = - \dfrac{e_3}{e_1} \left(\DP{s}{i}\right)_\rho, \;\;\;
  r_2 = - \dfrac{e_3}{e_2} \left(\DP{s}{j}\right)_\rho
\end{equation*}
%%
である. 
%$z$座標系の場合には, $e_3=1, s=z$となる. 

トレーサー$T$に対する四次の水平拡散は, 
%%
\begin{equation*}
 \mathcal{D}^{lT}_{\rm ldiff,4th} = 
   \nabla \cdot \left\{  A^{lT,4th} \; \mathcal{R} \nabla
       \left[ \nabla \cdot  (\mathcal{R} \nabla T) \right]
    \right\}
\end{equation*}
%%
の形式で与えられる. 

メソスケール渦により生じる移流のパラメタリゼーション\citep{gent1990isopycnal}は, 
%%
\begin{equation*}
 \mathcal{D}^{lT}_{\rm GM} = \nabla \cdot \left(\Dvect{U}^* T \right)
\end{equation*}
%%
と書かれる. 
ここで, $\Dvect{U}^*=(u^*,v^*,w^*)$は渦により生じる, 非発散な移流速度である. 
この速度は, 
%%
\begin{equation*}
\begin{split}
  u^* &= \dfrac{1}{e_3}\DP{}{s} (A^{GM} \tilde{r}_1), \;\;
  v^* = \dfrac{1}{e_3}\DP{}{s} (A^{GM} \tilde{r}_2), \\
  w^* &= - \dfrac{1}{e_1 e_2} \left[ 
             \DP{}{i} (A^{GM} e_2 \tilde{r}_1)
           + \DP{}{i} (A^{GM} e_1 \tilde{r}_2)
          \right]
\end{split} 
\end{equation*}
%%
と定義される. 
ここで, $A^{GM}$は渦により生じる速度と関係する係数であり, しばしば等密度面混合における拡散係数と同じ値が設定される. 
$\tilde{r}_1, \tilde{r}_2$は, \textbf{ジオポテンシャル面}と等密度面の間の傾斜
$$
\tilde{r}_n = r_n + \sigma_n \;\; (n=1,2 )
$$
であり,  $z$座標系の場合は単に$\tilde{r}_n=r_n$となる. 
また, 境界において, $\Dvect{U}^*$の法線方向の成分はゼロに設定する. 
GM スキームや等密度面混合スキームを実際に用いる際には, $\tilde{r}_n$が非常に大きい場所(海面の境界層内など)において, 
計算不安定の軽減や他のパラメタリゼーションスキーム(境界層スキーム等)との競合を避けるための対処
\citep{danabasoglu1995sensitivity,large1997sensitivity,griffies1998gent}が必要である. 
%GM スキームや等密度面混合スキームの詳細については, 第??章を参照されたい. 

水平運動量$\Dvect{U}_h$に対する二次の水平粘性に対する表現は, 
%%
\begin{equation*}
\begin{split}
 \Dvect{\mathcal{D}}^{l {\bm U}}_{\rm lvisc}
 &= \nabla_h (A^{lm} \chi) - \nabla_h \times (A^{lm} \zeta \Dvect{k})  \\
 &= 
\begin{pmatrix}
 \dfrac{1}{e_1}\DP{(A^{lm}\chi)}{i} - \dfrac{1}{e_2 e_3}\DP{(A^{lm} e_3 \zeta)}{j} \\
 \dfrac{1}{e_2}\DP{(A^{lm}\chi)}{j} + \dfrac{1}{e_1 e_3}\DP{(A^{lm} e_3 \zeta)}{i}
\end{pmatrix}
\end{split}
\end{equation*}
%%
である
\footnote{
水平速度ベクトル$\Dvect{U}_h$に対するラプラシアンは, ベクトル恒等式を用いて, 
%%
\begin{equation*}
\begin{split}
  \nabla^2 \Dvect{U}_h
 &= \nabla (\nabla \cdot\Dvect{U}_h) -  \nabla_h \times(\nabla \times \Dvect{U}_h)  \\
 & \\
 &=  \nabla_h (\chi) - \nabla_h \times (\zeta\Dvect{k}) + \dfrac{1}{e_3}\DP{}{s}\left(\dfrac{1}{e_3}\DP{\Dvect{U}_h}{s} \right)
\end{split}
\end{equation*}
%%
と変形できる. 
}. 
この表現において, 水平運動量はモデル座標の水平面に沿って拡散される. 
スカラー場に対する拡散演算子の場合とは違い, モデル座標の水平面と異なる等値面に沿った
水平粘性の表現を一般に得ることは困難である. 
ただし, デカルト座標系の場合には, スカラー場の拡散演算子のように, 
%%
\begin{equation*}
  \mathcal{D}^{lu}_{\rm lvisc} = \nabla\cdot(\mathcal{R}\nabla u), \;\;
  \mathcal{D}^{lv}_{\rm lvisc} = \nabla\cdot(\mathcal{R}\nabla v)
\end{equation*}
%%
と書くことができる. 

最後に, 水平運動量$\Dvect{U}_h$に対する四次の水平粘性は, 
%%
\begin{equation*}
  \mathcal{D}^{l{\bm U}}_{\rm lvisc,4th} = 
       \nabla_h \left\{ \nabla_h \cdot \left[ A^{lm,4th} \;  \nabla_h (\chi) \right]\right\}
     + \nabla_h \times \{ \Dvect{k}\cdot\nabla\times\left[A^{lm,4th} \; \nabla_h\times(\zeta \Dvect{k}) \right]  \}
\end{equation*}
%%
の形式で与えられる. 
ただし, 4 次の水平粘性もまた, モデル座標の水平面に沿って水平運動量が拡散される場合のみを考えることにする. 


%% 時間離散化
\section{時間離散化}\label{sice_model_time_scheme}
\markright{\arabic{chapter}.\arabic{section} 時間離散化}
%%
\subsection{時間スキームの概要}\label{sice_model_time_scheme_brief}
%%
\subsubsection*{熱力学過程}\label{sice_model_time_scheme_brief_thermodyn}
%%
鉛直熱拡散項と関係した氷層のエンタルピーの時間変化を, 以下の形式で陰的に評価する. 
%%
\begin{equation*}
 \dfrac{\rho_I h_I}{2} \left(\DD{E_{I,k}}{t}\right)_{\rm thermodyn} 
  = \dfrac{\rho_I h_I}{2 \Delta t} 
  \int_{t^n}^{t^*} \left(\DP{E_{I,k}}{T_{I,k}}\right)  \DD{T_{I,k}}{t} \Dd{t}
  = \mathcal{D}_{\rm vdiff}^{vT_{I}} (T_{I,1}^*, T_{I,2}^*, T_s^*)
\end{equation*}
%%
ここで, $k=1,2$であり, 
また$\mathcal{D}_{\rm vdiff}^{vT_{I}} (T_I^*)$は, 鉛直離散化した鉛直熱拡散項を表す. 
この時に, 海氷表面での熱収支式を連立することによって, 海氷の表面温度$T_s^*$も決定される.

次に, 上で計算された鉛直熱伝導フラックスから, 海氷上端での融解量, 海氷下端での生成融解量が, 
\eqref{eq:sice_model_surface_energy_budget}と\eqref{eq:sice_3lyr_model_bottom_energy_budget}によって計算される.
この過程において, 海氷上端の積雪や昇華が考慮される. 

最後に, 海氷内部の厚さの調整が行われる. 
このとき, 雪層から氷層への変化や氷層の上下層の等分化が行われ, それに伴い氷層の温度も変化する.  

\subsubsection*{力学過程}\label{sice_model_time_scheme_brief_dynamics}
%%
熱力学過程後の変数の値を用いて, 移流項に伴う海氷場の時間変化を陽的に評価する. 
$q=h_S, h_I, \rho_I h_I E_{I,1}/2, \rho_I h_I E_{I,2}/2$とすると, 
%%
\begin{equation*}
 \dfrac{q^{n+1} - q^*}{\Delta t}
 = \left(\DD{q}{t}\right)_{\rm dyn} (h_S^*, h_I^*, q^*)
\end{equation*}
%%
と書かれる. 
力学過程において, 氷層のエンタルピーの時間発展式は, 保存形式であることに注意が必要である. 

\subsection{時間スキームの詳細}\label{sice_model_time_scheme_detail}
%%
\subsubsection*{鉛直熱拡散方程式の時間離散化}
%%
海氷の鉛直熱拡散と関係する方程式は, 
%%
\begin{subequations}
\begin{gather}
 \dfrac{\rho_I h_I^n}{2}\left( C + \dfrac{L\mu S_i}{T_{I,1}^* T_{I,1}^n}\right)
 \dfrac{T_{I,1}^* - T_{I,1}^n}{\Delta t} 
 = K_{s1}^n (T_s^* - T_{I,1}^*) - K_{12}^n (T_{I,1}^* - T_{I,2}^*) + I, \\
%%
 \dfrac{\rho_I h_I^n}{2}C \dfrac{T_{I,2}^* - T_{I,2}^n}{\Delta t} 
  = K_{12}^n (T^*_{I,1} - T^*_{I,2}) - 2K_{12}^n (T^*_{I,2} - T_f)
\end{gather}
\label{eq:sice_model_thermodyn_thermo_vdiff_fvm}
\end{subequations}
%%
と時間離散化される%
\footnote{
\eqref{eq:sice_model_thermodyn_thermo_vdiff_fvm}(a) の左辺括弧内の表現を得るために,  
$$
 \frac{1}{\Delta t} \int_{t^*}^{t^n} 
 \left(C + L_i \dfrac{\mu S_i}{T_{I,1}^2} \right) \; 
  \DD{T_{I,1}}{t} \;\Dd{t} 
 =  \left( C + \dfrac{L\mu S_i}{T_{I,1}^*
     T_{I,1}^n}\right)
     \dfrac{T_{I,1}^* - T_{I,1}^n}{\Delta t} 
$$
であることが用いられる. 
}. 
また, \eqref{eq:sice_model_thermodyn_thermo_vdiff_fvm}に加えて, 
時刻$t^n$周りで海面熱フラックスを線形化した, 海氷面で融解が起きない場合の熱収支式
%%
\begin{equation}
	A + B T_s^* = K_{s1} (T_{I,1}^* - T_s^*)
\label{eq:sice_model_thermodyn_sfc_heat_budget_fvm}
\end{equation}
%%
を連立させる. 
ここで, 
%%
\begin{align*}
  A 
  &= F_s(T_s^n) - T_s^n \left(\DP{F_s}{T_s}\right)^n, \;\;
  B
   = \left(\DP{F_s}{T_s}\right)^n
\end{align*} 
%%
とおいた. 
後に記述するように, この方程式系で決定される海表面温度が, 雪もしくは海氷の融点を超える場合には, 
\eqref{eq:sice_model_thermodyn_sfc_heat_budget_fvm}を連立せずに, 海氷面温度を固定した問題に変更する. 

\cite{winton2000reformulated}では,
以下に示すように, \eqref{eq:sice_model_thermodyn_thermo_vdiff_fvm}を解析的に解くことで,  
$T_{I,1}^*$の表現を陽に導いている. 
はじめに, \eqref{eq:sice_model_thermodyn_thermo_vdiff_fvm}(b) から, 
%%
\begin{equation}
 T_{I,2}^* 
 = \dfrac{  2 \Delta t K^n_{12} (T_{I,1}^* + 2 T_f) 
          + \rho_I h_I C_i T_{I,2}^n}
          {6\Delta t K^n_{12} + \rho_I h_I C_i} 
\label{eq:sice_model_thermodyn_icetemp2_new}
\end{equation}
が得られる. 
\eqref{eq:sice_model_thermodyn_sfc_heat_budget_fvm}と\eqref{eq:sice_model_thermodyn_icetemp2_new}を用いて, 
\eqref{eq:sice_model_thermodyn_thermo_vdiff_fvm}(a)
に現れる$T_s^*$と$T_{I,2}^*$を消去すると, 
二次方程式
$A_1 (T_{I,1}^*)^2 + B_1 T_{I,1}^* + C_1 = 0$
が得られ, $T_{I,1}^*$は,
%%
\begin{equation}
 T_{I,1}^* = - \dfrac{B_1 + (B_1^2 - 4 A_1 C_1)^{1/2}}{2 A_1}
\label{eq:sice_model_thermodyn_icetemp1_new}
\end{equation}
%%
と与えられる.  
ここで, 
%%
\begin{align*}
 A_1 
 &= 
   \dfrac{\rho_I h_I}{2\Delta t}C_i
 + K_{12}\dfrac{4\Delta t K^n_{12} + \rho_I h_I C_i}
               {6\Delta t K^n_{12} + \rho_I h_I C_i}
 + \dfrac{K^n_{s1} B}{K^n_{s1} + B}, \\
%%
 B_1
 &= 
   - \dfrac{\rho_I h_I}{2\Delta t} 
       \left( C_i T_{I,1}^n - \dfrac{L_i\mu S_I}{T_{I,1}^n} \right) 
   - I \nonumber \\
 &\;\;\;\; - K_{12}\dfrac{4\Delta t K^n_{12} T_f + \rho_i h_i C_i T_{I,2}^n}
                 {6\Delta t K^n_{12} + \rho_i h_i C_i}
   + \dfrac{K^n_{s1} A}{K^n_{s1} + B}, \\
%%
  C_1 
  &= - \dfrac{\rho_I h_I}{2\Delta t} L_i \mu S_I
\end{align*} 
%%
である. 
これらの係数の符号を考慮すれば, 二次方程式のもう一つの解は正であり, 物理的に不適切であることが分かる.  
以上から, 
はじめに\eqref{eq:sice_model_thermodyn_icetemp1_new}から$T_{I,1}^*$が得られれば, 
\eqref{eq:sice_model_thermodyn_icetemp2_new}から$T_{I,1}^*$, 
\eqref{eq:sice_model_thermodyn_sfc_heat_budget_fvm}から$T_s^*$が求まる. 
 
もし, \eqref{eq:sice_model_thermodyn_icetemp1_new}から得られた$T_s^*$が,
雪の融点(雪が存在する場合)あるいは海氷の融点(雪が存在しない場合)を超えているならば, 
海氷の表面温度を$T_s^*=0$(雪が存在する場合)あるいは$T_s^*=-\mu S_i$(雪が存在しない場合)と固定した問題に変更して, 
鉛直熱拡散方程式を解き直す. 
この場合の二次方程式の係数は, 
%%
\begin{align*}
 A_1 
 &= 
   \dfrac{\rho_I h_I}{2\Delta t}C 
 + K_{12}\dfrac{4\Delta t K^n_{12} + \rho_I h_I C_i}
               {6\Delta t K^n_{12} + \rho_I h_I C_i}
 + K_{s1}^n, \\
%%
 B_1
 &= 
   - \dfrac{\rho_I h_I}{2\Delta t} 
       \left( C_i T_{I,1}^n - \dfrac{L_i\mu S_I}{T_{I,1}^n} \right) 
   - I \nonumber \\
 &\;\;\;\; - K_{12}\dfrac{4\Delta t K^n_{12} T_f + \rho_I h_I C_i T_{I,2}^n}
                 {6\Delta t K^n_{12} + \rho_I h_I C_i}
   - K_{s1}^n T_s^*
%%
\end{align*}
%%
である. 
ただし, $C_1$は変わらない. 

ここで陰的に求められた熱伝導フラックスを用いて, 海氷上下端での生成融解量が計算される. 



%% 空間離散化
\section{空間離散化}\label{sice_model_space_scheme}
\markright{\arabic{chapter}.\arabic{section} 空間離散化}
%%
\subsection{鉛直離散化}\label{sice_model_vspace_scheme}
\eqref{eq:sice_model_basic_equations}に含まれる鉛直熱拡散項は, 
有限体積法を用いて, 
%%
\begin{subequations}
\begin{gather}
 \dfrac{\rho_I h_I}{2} \left(\DD{E_{I,1}}{t}\right)_{\rm thermodyn} 
 %= \dfrac{\rho_i h_i}{2}\left(C + \dfrac{L\mu S_i}{T_1^2}\right) \DD{T_1}{t} 
 = K_{s1} (T_s - T_{I,1}) - K_{12} (T_{I,1} - T_{I,2}) + I, \\
 %%
 \dfrac{\rho_I h_I}{2} \left(\DD{E_{I,2}}{t}\right)_{\rm thermodyn} 
 %\dfrac{\rho_i h_i}{2} C \DD{T_2}{t}
  = K_{12} (T_{I,1} - T_{I,2}) - 2K_{12} (T_{I,2} - T_f) 
\end{gather}
\end{subequations}
%%
と鉛直離散化する. 
ここで, 上式に含まれる$K_{s1}, K_{12}$は, 
%% 
\begin{align*}
  K_{s1} = \dfrac{4 k_I k_S}{k_S h_I + 4 k_I h_S}, \;\;
  K_{12} = \dfrac{2 k_I}{h_I}
\end{align*}
%%
によって与えられ, $k_S$は雪の熱伝導率, $k_I$は氷の熱伝導率である.  
$K_{s1}$の表現は, 氷層上端での熱伝導フラックスと, 
雪層下端での熱伝導フラックスが等しいと仮定することによって決定される%
\footnote{ 
氷層上端での熱伝導フラックスと, 雪層下端での熱伝導フラックスが等しい条件は, 
雪層と氷層の境界の温度を$T^*$とすると,  
$$
  K_s \dfrac{T_s - T^*}{h_S} = K_i \dfrac{T^* - T_{I,1}}{h_i/4}
$$
と書かれる. この式から, 
$T^* = (K_s h_i T_s + 4 K_i h_s T_1)/(4K_i h_s + K_s h_i)$が得られる. 
したがって, 氷層上層の上端から雪層を通過する熱伝導フラックスの表現として, 
$$
 K_s \dfrac{T_s - T^*}{h_s} 
 = \dfrac{4 K_i K_s}{K_s h_I + 4K_i h_S} (T_s - T_{I,1}) 
$$
が得られ, 右辺の因子を$K_{s1}$と置いていることが分かる.  
}. 

\subsection{水平離散化}\label{sice_model_hspace_scheme}
%%
海氷の方程式系における力学項(現状は水平移流項のみ)の水平離散化に, 有限体積法を適用する.   
$q=h_S, h_I, \rho_I h_I E_{I,1}/2, \rho_I h_I E_{I,2}/2$とすると, 
%%
\begin{subequations}
\begin{gather}
 \left(\DD{q}{t}\right)_{\rm dyn}
 =  
   - \left[ \dfrac{  \delta_i (e_2 u^* \overline{q}^{i,UP1}) 
                   + \delta_j (e_2 v^* \overline{q}^{j,UP1})} {e_1 e_2} 
      \right]_{i,j}
\end{gather}
\end{subequations}
%%
と書かれる. 
ここで, $\overline{q}^{i,UP1}, \overline{q}^{j,UP1}$は, 1次精度の風上フラックスを与える補間を表し,  
例えば, 前者は, 
%%
\begin{equation*}
  \overline{q}^{i,UP1} 
  = \dfrac{q_{i,j} + q_{i+1,j}}{2} 
    + |u^*_{i+\frac{1}{2},j}| (q_{i,j} - q_{i+1,j})
\end{equation*}
%%
と与えられる. 
海氷の水平速度のパラメータリゼーション\eqref{eq:sice_model_horivel_param_thickness_diff}は, 
海氷厚さ(厳密には質量)の式に対して水平拡散と等価となるように, 
%%
\begin{align}
 (u^*)_{i+\frac{1}{2},j} 
 = K_h^s 
 \left(
  \dfrac{\delta_i [m_{\rm sice} ]}{e_1 \; \tilde{m}_{\rm sice}}
 \right)_{i+\frac{1}{2},j}, \;\;\;
%% 
 (v^*)_{i,j+\frac{1}{2}} 
 = K_h^s 
 \left(
  \dfrac{\delta_j [m_{\rm sice} ]}{e_2 \; \tilde{m}_{\rm sice}}
 \right)_{i,j+\frac{1}{2}}
%%
\label{eq:sice_model_horivel_param_thickness_diff_fvm}
\end{align}
%%
によって与えられる. 
ここで, $\tilde{m}_{\rm sice}$は, 例えば$i$方向に対して, 
$$
 (\tilde{m}_{\rm sice})_{i+\frac{1}{2},j} 
 = {\rm max}\left[(\tilde{m}_{\rm sice})_{i,j}, \;\;
                  (\tilde{m}_{\rm sice})_{i+1,j}
  \right]
$$
のように定義される. 



%% 物理パラメタリゼーションの詳細

%% パラメータの設定例
\section{パラメータの設定例}\label{model_ogcm_parameters}
%%
海洋モデルのパラメータ設定の例として, リスト\cite{dogcm_Pl64L60_conf}に, 
\cite{ykawai2018_Dthesis}で行なった海惑星の気候実験における設定を示す. 
なお, 海洋海氷モデルに入力する設定ファイルの中で, 海洋モデルと関係する部分のみを以下では示している.
%%
\begin{lstlisting}[caption={
海洋モデルの設定ファイルのテンプレート.
ただし, \#var\#のように書かれた部分は実験ケース等ごとに適切な値が入る.
},label={dogcm_Pl64L60_conf},
basicstyle=\ttfamily\footnotesize,
frame=single]
&DOGCM_nml
  OCN_do  = .true.,
  SICE_do = .true.,
  exp_name = '#DOGCM_nml_EXP_NAME#',
/
!= 各過程で使用するスキームの選択 -----------------------------
&GovernEq_nml
  DynEqTypeName  = 'HydroBoussinesq',  ! 力学
  EOSTypeName    = 'EOS_JM95',         ! 海水の状態方程式
  LPhysNames     = 'LMixMOM, LMixTRC, RediGM',  ! 水平方向の物理
  VPhysNames     = 'VMixMOM, VMixTRC, Convect', ! 鉛直方向の物理 
  SolverTypeName = 'HSPM-VFVM',        ! 空間スキーム
/
!= 境界条件の設定 ----------------------------------------
&BoundaryCondition_nml
  KinBCSurface   = 'Rigid',
  DynBCSurface   = 'SpecStress',
  DynBCBottom    = 'NoSlip',
  ThermBCSurface = '#BoundaryCondition_nml_ThermBCSurface#',
    ! 結合 run では PrescFlux, 
    ! 海洋海氷モデル単体 run では PrescFlux_Han1984
  ThermBCBottom  = 'Adiabat', 
  SaltBCSurface  = 'PrescFlux',
  SaltBCBottom   = 'Adiabat', 
/
!= 物理定数の設定 ----------------------------------------
&Constants_nml
  RPlanet     = 6.37d6,
  Omega       = 7.29d-5, 
  Grav        = 9.8d0, 
  hViscCoef   = 3d5,
  vViscCoef   = 1d-3,
  hDiffCoef   = 0d3,
  vDiffCoef   = 3d-5, 
  albedoOcean = 0d0,
  LatentHeat  = 2.4253d6,
  RefSalt     = 35d0, 
/
!= 格子の設定 ----------------------------------------
&Grid_nml
 IM = 1,   ! 経度格子点数.
 JM = 64,  ! 緯度格子点数. 
 KM = 60,  ! 鉛直層数.
/
&Grid_Spm_nml
 NM = 63,  ! 最大全波数.
/
!= 周期的同期結合と関連した処理や実験ケースの設定 -----------
&Exp_APECoupleClimate_nml
 RunCycle            =  #Exp_APECPLClim_nml_RunCycle#, 
 RunTypeName         = '#Exp_APECPLClim_nml_RunTypeName#', 
     ! 結合 run では Coupled, 
     ! 海洋海氷モデル単体 run では Standalone
 SfcBCDataDir        = '#Exp_APECPLClim_nml_SfcBCDataDir#', 
 SfcBCMeanInitTime   = #Exp_APECPLClim_nml_SfcBCMeanIniTime#, 
 SfcBCMeanEndTime    = #Exp_APECPLClim_nml_SfcBCMeanEndTime#,
 RestartDataDir      = '#Exp_APECPLClim_nml_RestartDataDir#', 
 RestartMeanInitTime =  #Exp_APECPLClim_nml_RestartMeanInitTime#, 
 RestartMeanEndTime  =  #Exp_APECPLClim_nml_RestartMeanEndTime#, 
 OcnInitSalt         = 35d0, [psu]
/
!= 鉛直混合の設定 -----------
&SGSPhys_VMixing_nml
  VMixCoef_scheme_name = 'Simple'  
/
!= 水平混合の設定 -----------
&RediGM_nml   ! Redi, GM スキーム
  DiffCoefRedi    = 800.0,
  DiffCoefGM      = 800.0,
  InteriorTaperingName = 'DM95'
  SlopeMax        = 4d-3,
  Sd              = 1d-3, 
  PBLTaperingName = 'LDD97'
/
&LPhys_DIFF_nml  ! 水平超粘性
  NumDiffOrdH     = 4,     ! 超粘性の次数. 
  NumDiffTimeVal  = 50.0,  ! 最大波数に対する e-folding time
  NumDiffTimeUnit = 'day', ! e-folding time の単位 
/
!= 時間スキーム・時間管理の設定 -----------
&TemporalInteg_nml
  cal_type             = 'noleap', 
  barocTimeIntModeName = 'TimeIntMode_LFAM3', ! 時間スキーム
  DelTimeVal           = #TemporalInteg_nml_DelTimeHour#,
       ! 基本的に, 結合runでは 4 h, 海洋海氷モデル単体runでは 12 h 
  DelTimeUnit          = 'hour',
  ReStartTimeVal       = #TemporalInteg_nml_RestartTimeVal#, 
  ReStartTimeUnit      = 'day', 
  InitYear = #TemporalInteg_nml_InitYear#, InitMonth=1, InitDay=1, InitHour=0,
  InitMin  = 0, EndYear =#TemporalInteg_nml_EndYear#, EndMonth =1,
  EndDay   = #TemporalInteg_nml_EndDay#, EndHour=0, EndMin=0,
/
!* セミ・インプリシットスキームに含まれる係数の設定 -----------
&SemiImplicitScheme_nml
  VDiffTermACoef    = 0.5d0,
  CoriolisTermACoef = 0.5d0, 
/
!= ファイル関連の設定 ----------------------------------------
!* ヒストリデータ出力の全体設定 
&gtool_historyauto_nml
  IntValue      = #gtool_historyauto_nml_IntValue#,  ! 出力間隔の数値
  IntUnit       = 'day',                             ! 出力間隔の単位
  OriginValue   = #gtool_historyauto_nml_OriginValue#,  
  OriginUnit    = 'day',                             
  TerminusValue = #gtool_historyauto_nml_TerminusValue#,  
  TerminusUnit  = 'day',                             
  FilePrefix    = '',
/
!* リスタートデータの入力・出力の設定
&OGCM_IO_Restart_nml
 OutputFileName = '#OcnRestartFile_nml_OutputFileName#', 
 InputFileName  = '#OcnRestartFile_nml_InputFileName#', 
 IntValue       = #OcnRestartFile_nml_IntValue#, 
 IntUnit        = 'day'
/
!* ヒストリデータ出力の個別設定
&gtool_historyauto_nml
  Name = 'U, V, OMG, PTemp, Salt, H, HydPres, ConvIndex', 
  Precision='float'
/
&gtool_historyauto_nml
  Name = 'SfcPres, SfcHFlxO, SfcHFlx_ns, SfcHFlx_sr, DSfcHFlxDTs, FreshWtFlxS', 
  Precision='float', 
  TimeAverage = .true.
/
&gtool_historyauto_nml
  Name = 'a2o_WindStressX, a2o_WindStressY, a2o_SenHFlx, a2o_LatHFlx,a2o_LDwRFlx, a2o_LUwRFlx, a2o_SDwRFlx, a2o_SUwRFlx, a2o_DSfcHFlxDTs,a2o_RainFall, a2o_SnowFall, a2o_Evap, a2s_Evap, a2o_SfcHFlxMod', 
  Precision = 'double', 
  TimeAverage = .true.
/
\end{lstlisting}



\chapter{海氷モデル}
%\section{海氷モデルの詳細}
\label{sice_model}
\markright{\arabic{chapter}.\arabic{section} 海氷モデル} %  節の題名を書き込むこと
%%
本節では, 本研究で用いた海氷モデルの詳細を記述する. 

%% モデルの基礎
\section{モデルの基礎}
\markright{\arabic{chapter}.\arabic{section} モデルの基礎} %  節の題名を書き込むこと
%%
将来的にモデルを拡張することを念頭に, 水平方向は局所直交座標系, 
鉛直方向は一般化した鉛直座標系を用いてモデルの方程式系を記述する. 
はじめに, モデルの基礎方程式系をベクトル形式で示す. 
次に水平座標系が局所直交座標系, 鉛直座標が幾何的座標($z$座標)の場合の方程式系を記述する.  
その後, 一般化した鉛直座標系($s$座標)を導入して, $s$座標を用いた場合の表現を得ることにする. 

\subsection{基礎方程式系の導入}
%%
海洋大循環モデルの基礎方程式系は, 静力学ブジネスク方程式系(プリミティブ方程式系)であり, 
運動量方程式, 連続の式, 温位と塩分の保存式, 海水の状態方程式により構成される. 
今, 等ジオポテンシャル面と直交する単位ベクトルを$\Dvect{k}$, 
等ジオポテンシャル面と接する単位ベクトルを$(\Dvect{i},\Dvect{j})$として, 
局所直交座標系$(\Dvect{i}, \Dvect{j}, \Dvect{k})$を導入する. 
また, 三次元速度ベクトル$\Dvect{U}$は$\Dvect{U}=\Dvect{U}_h + w \Dvect{k}$のように水平成分と鉛直成分に分割する. 
このとき, 静力学ブジネスク方程式系のベクトル不変形式は以下のように書かれる. 
%%
\begin{subequations} 
\label{eq:OCN_primitive_eq_vector_form}
\begin{gather}
 \DP{\Dvect{U}_h}{t} = 
   - \left[ (\nabla \times \Dvect{U}) \times \Dvect{U} + \nabla \left(\dfrac{\Dvect{U}^2}{2}\right) \right]_h  
   - f \Dvect{k} \times \Dvect{U}_h 
   - \dfrac{1}{\rho_0} \nabla_h p  + \Dvect{\mathcal{\tilde{D}}}^{\bm{U}} , 
\label{eq:horizontal_mom_eq} \\
%%
 \DP{p}{z} = - \rho g, 
 \label{eq:hydrostatic_eq} \\
%%
 \nabla \cdot \Dvect{U} = 0, 
 \label{eq:continuous_eq} \\
%%
 \DP{\Theta}{t} = - \nabla\cdot(\Theta\Dvect{U}) + \mathcal{\tilde{D}}^\Theta , \\
%%
 \DP{S}{t} = - \nabla\cdot(S\Dvect{U}) + \mathcal{\tilde{D}}^S, \\
%%
 \rho = \rho(\Theta, S, p). 
%%
\end{gather}  
\end{subequations}
%%
ここで, $\nabla$は三次元勾配ベクトル, $t$は時刻, $z$は鉛直座標, $\rho$は海水の密度, 
$\rho_0$は基準密度, $p$は圧力, 
$f=2\Dvect{\Omega}\cdot\Dvect{k}$はコリオリパラメータ($\Dvect{\Omega}$は自転角速度ベクトル), 
$g$は重力加速度である. 
また, $\Dvect{\mathcal{\tilde{D}}}^{\bm{U}}, \mathcal{\tilde{D}}^\Theta, \mathcal{\tilde{D}}^S$は, 
中規模渦による水平混合, 小スケールの渦や対流による鉛直混合などの, 
サブグリッド・スケールの物理を表現するパラメタリゼーションの項を表す. 

海面における運動学的な境界条件は, 
%%
\begin{subequations}
\begin{equation}
  w = \DP{\zeta}{t} + \Dvect{U}_h \cdot \nabla_h \eta + (P - E) \;\;\; {\rm at} \; z = \eta, 
\label{eq:ocn_surface_bc}
\end{equation}
%%
である. 
ここで, $\eta$は海水表面の変位である. また, $(P-E)$は淡水フラックスであり, 降水量と蒸発量の差を表す. 
一方で, 海底面における運動学的な境界条件は, 
\begin{equation}
  w = - \Dvect{U}_h \cdot \nabla_h H \;\;\; {\rm at} \; z = -H
\label{eq:ocn_bottom_bc}
\end{equation}
\label{eq:ocn_vertical_bc}
\end{subequations}
%%
である. 
ここで, $H$は平均的な海面($z=0$)から海底までの距離である. 

\subsection{圧力の定式化}
%%
ある深さ$z$における圧力$p$は, 大気圧$p_a$, 
基準となる海水面($z=0$)に対する表面の変動と関係した圧力$p_s$, 
静水圧$p_h$の三つの寄与に分割され, 
%%
\begin{align*}
 p(i,j,z,t) 
 &= p_a + p_s(i,j,t) + p_h(i,j,z,t)  \\
 &= p_a + (\rho_0 g\eta) + (p_h - \rho_0 gz)
\end{align*}
%%
と書かれる. 
ここで, $p_h$は, 静水圧平衡の式\eqref{eq:hydrostatic_eq}から, 
%%
\begin{equation*}
  p_h(i,j,z,t) = \int^\eta_z g \left[\rho(\Theta,S, -\rho_0 g z') - \rho_0 \right] \Dd{z'}
\end{equation*}
%%
と求まる. 
ただし, ブジネスク方程式系においてエネルギー論の一貫性を保つには, 
状態方程式に現れる圧力は深さのみに依存する形式で書く必要があるため, 
$p \approx -\rho_0 gz$と近似している.
表面圧力$p_s$の計算には, 以下の 2 通りの方法がある. 
%%
\begin{enumerate}
  \setlength{\itemsep}{-0.5mm} % 項目の隙間
  \setlength{\parskip}{-0.5mm} % 段落の隙間
 \item 海水面の変位$\eta$を新たな予報変数として導入し, $\eta$(すなわち$p_s=\rho_0 g \eta$)の時間発展式を解く. 
 \item 海面において剛体蓋近似を適用し, $p_s$は診断的に決定する. 
\end{enumerate}
%%
前者の方法において, $\eta$の時間発展式は連続の式を鉛直積分することで得られる. 
海面の変位を許容する場合には表面波を表現できるが, その位相速度は海洋大循環
と比べるとずっと速い(数百 m/s)ために, 時間刻み幅に強い制約を与える. 
後者の場合では, 鉛直積分した水平速度が非発散であるという拘束条件を用いて, 
表面圧力に関する楕円方程式を導くことができる. 海面の変位は許されないために, 
表面波は解として含まれない. 剛体蓋近似の適用は, 表面波(外部重力波)だけでなく
順圧ロスビー波等の長波にも影響を与えることに注意が必要である. 

\subsubsection*{ \underline{方法 1 の場合: 表面変位$\eta$の時間発展式} }
%%
連続の式\eqref{eq:continuous_eq}を, 海底$z=-H$から海面$z=\eta$まで鉛直積分すると, 
%%
\begin{equation*}
  w_{z=\eta} = w_{z=-H} - \int_{z=-H}^{z=\eta} \nabla_h \cdot \Dvect{U}_h \; dz
\end{equation*}
%%
となる. 
海面および海底での運動学的境界条件\eqref{eq:ocn_vertical_bc}を適用すると, 
$\eta$の時間発展式として, 
%%
\begin{equation}
  \DP{\eta}{t} = - \nabla_h \cdot \int_{z=-H}^{z=\eta} \Dvect{U}_h \; dz
                 + (P - E)
\label{eq:free_surface_flux_eq}
\end{equation}
%%
を得る. 

\subsubsection*{ \underline{方法 2 の場合: 表面圧力$p_s$の診断方程式} }
%%
剛体蓋近似を適用する場合には, $w=D\eta/Dt=0$であることを課すので, 
$\eta$は常に定数(ここでは, ゼロとする)である. したがって, 
\eqref{eq:free_surface_flux_eq}は, 
%%
\begin{equation}
 \nabla_h \cdot \int_{z=-H}^{z=0} \Dvect{U}_h = 0
\label{eq:rigid_lid_constraintment_horimom}
\end{equation}
%%
となり, 鉛直積分した水平速度は非発散でなければならない. 
次に, 水平方向の運動量方程式\eqref{eq:horizontal_mom_eq}を$z=-H$から$z=0$まで鉛直積分し, 
水平発散をとった後に, \eqref{eq:rigid_lid_constraintment_horimom}を適用すれば, 
%%
\begin{equation}
 \nabla_h \cdot ( H \nabla_h \; p_s) = \int_{-H}^0 F_b \; dz
\label{eq:rigid_lid_pressure_poisson}
\end{equation}
%%
が得られる. 
ここで, $F_b$は, 表面圧力勾配の項(と局所時間微分の項)を除く全ての項の寄与を表す. 
この表面圧力に対する楕円方程式を解くことにより, 表面圧力を決定できる. 

\subsection{水平直交曲線座標・$z$座標における基礎方程式系の表現}
%%
水平直交曲線座標・鉛直$z$座標の座標変数$(i,j,z)$は, 
地理座標系の座標変数と
%%
\begin{equation*}
 i = i(\lambda,\phi), \;\; j = j(\lambda,\phi) \;\; z=z
\end{equation*}
%%
によって関係付けられるとする. 
ここで, $\lambda$は経度, $\phi$は緯度, $z$は平均海面水位からの高度である. 
それぞれの方向の単位ベクトルは前に定義した$\Dvect{i}, \Dvect{j}, \Dvect{k}$であり,  
これらは局所的に直交することに注意が必要である. 
このとき, 水平方向のスケール因子$e_1, e_2, e_3$は, 
%%
\begin{equation*}
\begin{split}
  e_1 &= a \left[ \left(\DP{\lambda}{i} \cos\phi \right)^2 + \left(\DP{\phi}{i}\right)^2 \right]^{1/2}, \\ 
  e_2 &= a \left[ \left(\DP{\lambda}{j} \cos\phi \right)^2 + \left(\DP{\phi}{j}\right)^2 \right]^{1/2}
\end{split}
\end{equation*}
%%
によって与えられる. 
惑星半径に対して海洋が十分に浅いモデルを考えるので, もとのスケール因子に含まれる$(a+z)$は$a$に置き換えた. 

導入した座標系$(i,j,k)$において, 
モデルの基礎方程式系\eqref{eq:OCN_primitive_eq_vector_form}は以下のように書かれる. 
%%
\begin{subequations} 
\label{eq:OCN_basic_equations_z_coord}
\begin{equation}
\begin{split}
  \DP{u}{t} 
  =& + (\zeta + f)v -  \dfrac{1}{e_1} \DP{}{i} \left( \dfrac{u^2 + v^2}{2} \right) 
     - w\DP{u}{z}            \\
   & - \dfrac{1}{e_1} \DP{}{i} \left( \dfrac{p_s + p_h}{\rho_0} \right) 
     + \mathcal{\tilde{D}}^u, 
\end{split}
\end{equation}
%%
\begin{equation}
\begin{split}
  \DP{v}{t} 
   =& - (\zeta + f)u -  \dfrac{1}{e_2} \DP{}{j} \left( \dfrac{u^2 + v^2}{2} \right) 
      - w\DP{v}{z}             \\
    & - \dfrac{1}{e_2} \DP{}{j} \left( \dfrac{p_s + p_h}{\rho_0} \right) 
      + \mathcal{\tilde{D}}^v, 
\end{split}
\end{equation}
%%
\begin{equation}
  \DP{p_h}{z} = - (\rho - \rho_0)  g, 
\end{equation}
%%
\begin{equation}
  \DP{w}{z} = - \chi \\
\end{equation}
%%
\begin{equation}
  \DP{\Theta}{t} 
  =   
    - \dfrac{1}{e_1 e_2}\left[ \DP{(e_2 u \Theta)}{i} + \DP{(e_1 v \Theta)}{j} \right]
    - \DP{(\Theta w)}{z}
    + \mathcal{\tilde{D}}^\Theta, 
\end{equation} 
%%
\begin{equation}
  \DP{S}{t} 
  =   
    - \dfrac{1}{e_1 e_2}\left[ \DP{(e_2 u S)}{i} + \DP{(e_1 v S)}{j} \right]
    - \DP{(S w)}{z}
    + \mathcal{\tilde{D}}^S, 
\end{equation}
%%
\begin{equation}
  \rho_o = \rho_o(\Theta,S,z). 
\end{equation}
%%
\end{subequations}
%%
ここで, $(u,v)$は水平速度ベクトル$\Dvect{U}_h$の各方向成分, 
$\Dvect{\mathcal{\tilde{D}}}^{\bm{U}}=(\mathcal{\tilde{D}}^u, \mathcal{\tilde{D}}^v)$である. 
また, $\zeta$は相対渦度の鉛直成分, $D$は速度の水平発散であり, $z$座標系において, 
%%
\begin{equation}
 \zeta = \dfrac{1}{e_1 e_2}\left[ \DP{(e_2 v)}{i} - \DP{(e_1 u)}{j} \right], \;\;\;
 D  = \dfrac{1}{e_1 e_2}\left[ \DP{(e_2 u)}{i} + \DP{(e_1 v)}{j} \right]
\label{eq:relvor_hdiv_zcoord}
\end{equation}
%
と書かれる. 
なお, 表面圧力$p_s$は, \eqref{eq:rigid_lid_pressure_poisson}あるいは\eqref{eq:free_surface_flux_eq}から決定され, 
また大気圧の影響はここでは無視した. 

\subsection{水平直交曲線座標・一般鉛直座標における基礎方程式系の表現}
%%
現業の海洋モデルでは, 海底地形や海面変位を考慮するために様々な鉛直座標が用いられる. 
対象とする問題に応じて適切な鉛直座標が選択できるように, しばしば一般化した鉛直座標($s$座標)を用いて
定式化がなされる. 
今, $z$座標系の座標変数を$(i^*,j^*,z^*)$と$s$座標系の座標変数を$(i,j,s)$とすると, 
$z$座標系から$s$座標系への変換は, 
%%
\begin{equation*}
  i = i^*, \;\; j = j^*, \;\; s = s(i^*,j^*,z^*,t^*), \;\; t=t^*
\end{equation*}
%%
によって関係付けられる. 

次に, $z$座標系の方程式系から$s$座標系の方程式系への変換に必要な幾つかの関係式を導く.
物理量$\Psi$が, $z$座標系では$\Psi=\psi^*(i^*,j^*,z^*,t^*)$,
$s$座標系では$\Psi=\psi(i,j,s,t)$と表されるとすると, 
物理量$\Psi$の偏微分は, 微分の連鎖率によって, 
%%
\begin{equation}
\begin{split}
 \DP{\psi^*}{i^*} &= \DP{\psi}{i} + \DP{\psi}{s}\DP{s}{i^*} 
                   = \DP{\psi}{i} - \dfrac{e_1}{e_3}\sigma_1 \DP{\psi}{s}, \\
 \DP{\psi^*}{j^*} &= \DP{\psi}{j} + \DP{\psi}{s}\DP{s}{j^*} 
                   = \DP{\psi}{j} - \dfrac{e_2}{e_3}\sigma_2 \DP{\psi}{s}, \\
 \DP{\psi^*}{z^*} &= \DP{\psi}{s} \DP{s}{z^*} 
                   = \dfrac{1}{e_3} \DP{\psi}{s}, \\
 \DP{\psi^*}{t^*} &= \DP{\psi}{t} + \DP{\psi}{s}\DP{s}{t^*} 
                   = \DP{\psi}{t} + \DP{\psi}{s}\DP{s}{t}                   
\end{split}
\label{eq:derivative_chain_rule_scoord}
\end{equation}
%%
と変換される. 
ここで, $e_3$は鉛直方向のスケール因子, $\sigma_1, \sigma_2$は水平方向の$s$面の傾斜であり, 
%%
\begin{equation*}
 e_3 = \DP{z^*}{s}, \;\; 
 \sigma_1 = \dfrac{1}{e_1} \left(\DP{z^*}{i^*}\right)_s, \;\;
 \sigma_2 = \dfrac{1}{e_2} \left(\DP{z^*}{j^*}\right)_s
\end{equation*} 
%%
と定義される. 
一方, $\Psi$のラグランジュ微分は, \eqref{eq:derivative_chain_rule_scoord}を用いて, 
%%
\begin{equation*}
\begin{split}
 \DD{\psi^*}{t^*} &=   \DP{\psi^*}{t^*} 
                     + \dfrac{u}{e_1}\DP{\psi^*}{i^*}
                     + \dfrac{v}{e_2}\DP{\psi^*}{j^*}
                     + w\DP{\psi^*}{z^*} \\
                 &= \DP{\psi}{t} 
                     + \dfrac{u}{e_1}\DP{\psi}{i}
                     + \dfrac{v}{e_2}\DP{\psi}{j}
                     + \left(w + e_3 \DP{s}{t} - \sigma_1 u - \sigma_2 v \right) \dfrac{1}{e_3} \DP{\psi}{s}
\end{split}
\end{equation*}
%%
と変換される. 
したがって, $s$座標系における鉛直速度$\omega$は, 
%%
\begin{equation*}
  \omega \equiv \DD{s}{t^*} 
=  w - w_s - \sigma_1 u - \sigma_2 v
\end{equation*}
%%
によって与えられる. ここで, 
$w_s \equiv (\partial z^*/\partial t)_s = - e_3 (\partial s/\partial t)_{z^*}$
である.
最終的に, $\Psi$のラグランジュ微分は, 
%%
\begin{equation}
 \DD{\psi^*}{t^*} = \DP{\psi}{t} 
                     + \dfrac{u}{e_1}\DP{\psi}{i}
                     + \dfrac{v}{e_2}\DP{\psi}{j}
                     + \omega \DP{\psi}{s}
\label{eq:lagrange_derivative_scoord}
\end{equation}
%%
と書ける. 

$z$座標系における方程式系\eqref{eq:OCN_basic_equations_z_coord}や
鉛直渦度や水平発散の表現\eqref{eq:relvor_hdiv_zcoord}に対して, 
鉛直座標変換のための関係式\eqref{eq:derivative_chain_rule_scoord}, \eqref{eq:lagrange_derivative_scoord}を適用し整理すると, 以下の$s$座標系における方程式系を導かれる. \\\\
%%
\noindent
\framebox[15cm][c]{
\begin{minipage}{13cm}
\begin{subequations} 
\label{eq:OCN_basic_equations_s_coord}
\begin{equation}
\begin{split}
 \DP{u}{t} 
  =& + (\zeta + f)v -  \dfrac{1}{e_1} \DP{}{i} \left( \dfrac{u^2 + v^2}{2} \right) 
     - \dfrac{\omega}{e_3} \DP{u}{s}            \\
   & - \dfrac{1}{e_1} \DP{}{i} \left( \dfrac{p_h+p_s}{\rho_0} \right) 
     + g\dfrac{\rho'}{\rho_0} \sigma_1 
     + \mathcal{D}^u, 
\end{split}
\end{equation}
%%
\begin{equation}
\begin{split}
  \DP{v}{t} 
   =& - (\zeta + f)u -  \dfrac{1}{e_2} \DP{}{j} \left( \dfrac{u^2 + v^2}{2} \right) 
      - \dfrac{\omega}{e_3}\DP{v}{s}             \\
    & - \dfrac{1}{e_2} \DP{}{j} \left( \dfrac{p_s + p_h}{\rho_0} \right) 
     + g\dfrac{\rho'}{\rho_0} \sigma_2
     + \mathcal{D}^v, 
\end{split}
\end{equation}
%%
\begin{equation}
  \dfrac{1}{e_3}\DP{p_h}{s} = - \dfrac{\rho'}{\rho_0}  g, 
\end{equation}
%%
\begin{equation}
  \varepsilon_1 \DP{e_3}{t} + e_3 \underline{D} + \DP{\omega}{s} = 0, \\
\end{equation}
%%
\begin{equation}
\begin{split}
 \varepsilon_1 \DP{\eta}{t} 
  =   
    &- \dfrac{1}{e_1 e_2}\left[ 
    		  \DP{\left(e_2 (\varepsilon_2 \eta + H) U_{barot}\right)}{i} 
    		+ \DP{\left(e_1 (\varepsilon_2 \eta + H) V_{barot}\right)}{j} 
    \right]  \\
    	&+ \varepsilon_1 (P-E) 
\end{split}
\end{equation} 
%%
\begin{equation}
 \dfrac{1}{e_3}\DP{(e_3 \Theta)}{t} 
  =   
    - \dfrac{1}{e_1 e_2 e_3}\left[ \DP{(e_2 e_3 u \Theta)}{i} + \DP{(e_1 e_3 v \Theta)}{j} \right]
    - \dfrac{1}{e_3} \DP{(\Theta \omega)}{s}
    + \mathcal{D}^\Theta, 
\end{equation} 
%%
\begin{equation}
  \dfrac{1}{e_3}\DP{(e_3 S)}{t} 
  =   
    - \dfrac{1}{e_1 e_2 e_3}\left[ \DP{(e_2 e_3 u S)}{i} + \DP{(e_1 e_3 v S)}{j} \right]
    - \dfrac{1}{e_3} \DP{(S \omega)}{s}
    + \mathcal{D}^S, 
\end{equation}
%%
\begin{equation}
  \rho_o = \rho_o(\Theta,S,z(i,j,s,t)). 
\end{equation}
%%
\end{subequations}
\end{minipage}
}\\\\
%%
ここで, $\rho'(=\rho - \rho_0)$は参照密度からの偏差である. 
また, $\mathcal{D}^u, \mathcal{D}^v, \mathcal{D}^\Theta, \mathcal{D}^S$はそれぞれ, 
$\mathcal{\tilde{D}}^u, \mathcal{\tilde{D}}^v, \mathcal{\tilde{D}}^\Theta, \mathcal{\tilde{D}}^S$
の$s$座標系での表現である. 
$s$座標系の鉛直渦度と水平発散の表現は, $s$座標系において, 
%%
\begin{equation}
 \zeta = \dfrac{1}{e_1 e_2}\left[ \DP{(e_2 v)}{i} - \DP{(e_1 u)}{j} \right], \;\;\;
 \underline{D}  = \dfrac{1}{e_1 e_2 e_3}\left[ \DP{(e_2 e_3 u)}{i} + \DP{(e_1 e_3 v)}{j} \right]
\label{eq:relvor_hdiv_generalvcoord}
\end{equation}
%
となる. 
$\underline{D}$は, $e_3=e_3(z)$の場合を除いて, 前に示した水平発散の表現$D$とは一致しないことに注意されたい. 
$U_{barot}, V_{barot}$は, 鉛直平均した水平速度(順圧成分)であり, 
例えば, $U_{barot}$は, 
%%
\begin{align*}
 U_{barot} = \dfrac{1}{\varepsilon_2 \eta + H}
  \int_{s(z=-H)}^{s(z=\varepsilon_2 \eta)} (U e_3) \; \Dd{s}
\end{align*}
%%
によって定義される. 
$\varepsilon_1, \varepsilon_2$は海面(故に表面圧力)の扱いにより決まる指標であり, 
(i) 剛体蓋近似を適用する場合は$\varepsilon_1=\varepsilon_2=0$, 
(ii) 線形化した自由表面の扱いの場合は$\varepsilon_1=1, \varepsilon_2=0$,
(iii) 海面の扱いに関して近似をしない場合は$\varepsilon_1=\varepsilon_2=1$である. 
表面圧力$p_s$は,
(i)の場合は\eqref{eq:rigid_lid_pressure_poisson},
(ii)または(iii)の場合は\eqref{eq:free_surface_flux_eq}の方法に基づいて決定される. 
任意の時刻・位置における$z$の値は, $\eta$が既知であれば, 定義した鉛直座標変換の関係式から計算することができ,  
同時に鉛直方向のスケール因子$e_3(=\partial z/\partial s)$も計算できる. 

%%%%%%%%%%%%%%%%%%%%%%%%%%%%%%%%%%%%%%%%%%%%%%%%%%%%%%%%%%%%%%%%%%%%%%%%%%%%%%%%%%%%%%%%%%%%%%%%%%%
\subsection{サブグリッド・スケールの物理の表現}
%%
重力の存在は, 水平運動と鉛直運動の間に強い非等方性を生む. 
そのため, \eqref{eq:OCN_basic_equations_s_coord}における, 
サブグリッド・スケールの物理による寄与$D^q$(ただし, $q=u,v,\Theta,S$)を, 
水平運動に伴う寄与$\mathcal{D}^{lq}$と鉛直運動に伴う寄与$\mathcal{D}^{vq}$に分ける. 
ここでは, これらの項の形式を簡潔にまとめることに主眼を置き,  
各パラメタリゼーション・スキームの詳細については参考文献を参照されたい.  

\subsubsection*{\underline{鉛直方向の混合}}
%%
鉛直乱流混合を生じさせるシア不安定や内部波の砕破等の過程は, 
海洋大循環モデルで典型的な格子スケールよりもはるかに小さいため, 
その効果を取り入れるにはパラメータ化する必要がある. 
鉛直乱流フラックスは, 格子スケールの変数の空間勾配に線形に比例すると仮定し, 
$\overline{q'w'}= -A^{vq} \partial{\overline{q}}/\partial {z}$のように表現する. 
このとき, 運動量, トレーサーの鉛直乱流混合を表現する項は, 二階の空間微分を用いて, 
%%
\begin{equation*}
\begin{split}
 \Dvect{\mathcal{D}}^{v{\bm U}} 
    &= \dfrac{1}{e_3} \DP{}{s}\left( \dfrac{A^{vm}}{e_3} \DP{\Dvect{U}_h}{s} \right), \\
 \mathcal{D}^{v\Theta}  &= \dfrac{1}{e_3}\DP{}{s}\left( \dfrac{A^{vT}}{e_3} \DP{\Theta}{s} \right), \;\;
 \mathcal{D}^{vS}  = \dfrac{1}{e_3}\DP{}{s}\left( \dfrac{A^{vT}}{e_3} \DP{S}{s} \right)
\end{split}
\end{equation*}
%%
と書かれる. 
ここで, $A^{vm}, A^{vT}$はそれぞれ, 鉛直渦粘性係数, 鉛直渦拡散係数である. 
これらの係数として, (a)定数, (b)位置の関数, (c)局所的な流体の特性(安定度やリチャードソン数など)の簡単な関数, 
の三種類の場合がモデルでは想定されている. 
鉛直乱流混合をより良く表現するために海洋モデル業界では, K-profile パラメタリゼーション(KPP)スキーム
\citep{large1994oceanic}等の乱流クロージャーが用いられるが, 本モデルには今の所実装できていない. 

%%%%%%%%%%%%%%%%%%%%%%%%%%%%%%%%%%%%%%%%%%%%%%%%%%%%%%%%%%%%%%%%%%%%%%%%%%%%%%%%%%%%%%%%%%%%%%%%%%%
\subsubsection*{\underline{水平方向の混合}}
%%
水平方向の乱流混合は, メソスケールの乱流(直径数百 km 程度)による混合と, 
サブメソスケールの乱流(1$\sim$50 km 程度)による混合に大きくは分けられる. 
メソスケールの乱流は, 解像度が十分に高ければ, 静力学平衡を仮定する海洋大循環モデルにおいても陽に表現できる. 
しかしながら, 全球領域で長時間積分を行う場合に, メソスケールの渦を十分に表現できる解像度で計算することは今でも容易でない.  
そのため, 伝統的には, 等密度面混合スキーム\citep{redi1982oceanic}と 
GM スキーム\citep{gent1990isopycnal}といったパラメタリゼーションを用いる.
一方, サブメソスケールの乱流は, 海洋大循環モデルにおいて現実的な格子スケールよりも遥かに小さく, 
また静力学近似の適用限界も存在するために, モデルの中で陽に表現することは難しい. 
そのため, サブメソスケールの渦の効果をモデルに取り入れるためにはパラメタリゼーションを要する.  
以下では, トレーサーに対する二次の水平拡散, 四次の水平拡散, メソスケール渦による移流のパラメタリゼーション, 
水平運動量に対する二次の水平粘性, 四次の水平粘性に伴う項の形式をまとめる. 

はじめに, トレーサー$T$に対する二次の水平拡散の表現は,  
%%
\begin{equation}
 \mathcal{D}^{lT}_{\rm ldiff} = \nabla\cdot (A^{lT} \mathcal{R} \nabla T), 
 \;\;\;\; 
 \mathcal{R} =
\begin{pmatrix}
  1 &0 &-r_1              \\
  0 &1 &-r_2               \\
  -r_1 &-r_2 &r_1^2 + r_2^2
\end{pmatrix}
\label{eq:lateral_diffusion_expression}
\end{equation}
%%
である. 
ここで, $r_1, r_2$は, モデル座標系の水平面と拡散演算子が作用する面との間の傾斜を表す. 
\eqref{eq:lateral_diffusion_expression}は, 等ジオポテンシャル面に沿う拡散を想定するならば厳密な表現である. 
一方で, 等密度面に沿った拡散を想定するならば, 
現実の海洋内部領域における等密度面の傾斜が最大でも$10^{-2}$程度であることを利用した近似的な表現である\citep{redi1982oceanic}. 
$r_1, r_2$は, 等ジオポテンシャル面に沿った拡散に対しては, 
$r_1=\sigma_1, r_2=\sigma_2$であり, $z$座標系では$r_1=r_2=0$となる. 
また, 等密度面に沿った拡散に対しては, 
%%
\begin{equation*}
  r_1 = - \dfrac{e_3}{e_1} \left(\DP{s}{i}\right)_\rho, \;\;\;
  r_2 = - \dfrac{e_3}{e_2} \left(\DP{s}{j}\right)_\rho
\end{equation*}
%%
である. 
%$z$座標系の場合には, $e_3=1, s=z$となる. 

トレーサー$T$に対する四次の水平拡散は, 
%%
\begin{equation*}
 \mathcal{D}^{lT}_{\rm ldiff,4th} = 
   \nabla \cdot \left\{  A^{lT,4th} \; \mathcal{R} \nabla
       \left[ \nabla \cdot  (\mathcal{R} \nabla T) \right]
    \right\}
\end{equation*}
%%
の形式で与えられる. 

メソスケール渦により生じる移流のパラメタリゼーション\citep{gent1990isopycnal}は, 
%%
\begin{equation*}
 \mathcal{D}^{lT}_{\rm GM} = \nabla \cdot \left(\Dvect{U}^* T \right)
\end{equation*}
%%
と書かれる. 
ここで, $\Dvect{U}^*=(u^*,v^*,w^*)$は渦により生じる, 非発散な移流速度である. 
この速度は, 
%%
\begin{equation*}
\begin{split}
  u^* &= \dfrac{1}{e_3}\DP{}{s} (A^{GM} \tilde{r}_1), \;\;
  v^* = \dfrac{1}{e_3}\DP{}{s} (A^{GM} \tilde{r}_2), \\
  w^* &= - \dfrac{1}{e_1 e_2} \left[ 
             \DP{}{i} (A^{GM} e_2 \tilde{r}_1)
           + \DP{}{i} (A^{GM} e_1 \tilde{r}_2)
          \right]
\end{split} 
\end{equation*}
%%
と定義される. 
ここで, $A^{GM}$は渦により生じる速度と関係する係数であり, しばしば等密度面混合における拡散係数と同じ値が設定される. 
$\tilde{r}_1, \tilde{r}_2$は, \textbf{ジオポテンシャル面}と等密度面の間の傾斜
$$
\tilde{r}_n = r_n + \sigma_n \;\; (n=1,2 )
$$
であり,  $z$座標系の場合は単に$\tilde{r}_n=r_n$となる. 
また, 境界において, $\Dvect{U}^*$の法線方向の成分はゼロに設定する. 
GM スキームや等密度面混合スキームを実際に用いる際には, $\tilde{r}_n$が非常に大きい場所(海面の境界層内など)において, 
計算不安定の軽減や他のパラメタリゼーションスキーム(境界層スキーム等)との競合を避けるための対処
\citep{danabasoglu1995sensitivity,large1997sensitivity,griffies1998gent}が必要である. 
%GM スキームや等密度面混合スキームの詳細については, 第??章を参照されたい. 

水平運動量$\Dvect{U}_h$に対する二次の水平粘性に対する表現は, 
%%
\begin{equation*}
\begin{split}
 \Dvect{\mathcal{D}}^{l {\bm U}}_{\rm lvisc}
 &= \nabla_h (A^{lm} \chi) - \nabla_h \times (A^{lm} \zeta \Dvect{k})  \\
 &= 
\begin{pmatrix}
 \dfrac{1}{e_1}\DP{(A^{lm}\chi)}{i} - \dfrac{1}{e_2 e_3}\DP{(A^{lm} e_3 \zeta)}{j} \\
 \dfrac{1}{e_2}\DP{(A^{lm}\chi)}{j} + \dfrac{1}{e_1 e_3}\DP{(A^{lm} e_3 \zeta)}{i}
\end{pmatrix}
\end{split}
\end{equation*}
%%
である
\footnote{
水平速度ベクトル$\Dvect{U}_h$に対するラプラシアンは, ベクトル恒等式を用いて, 
%%
\begin{equation*}
\begin{split}
  \nabla^2 \Dvect{U}_h
 &= \nabla (\nabla \cdot\Dvect{U}_h) -  \nabla_h \times(\nabla \times \Dvect{U}_h)  \\
 & \\
 &=  \nabla_h (\chi) - \nabla_h \times (\zeta\Dvect{k}) + \dfrac{1}{e_3}\DP{}{s}\left(\dfrac{1}{e_3}\DP{\Dvect{U}_h}{s} \right)
\end{split}
\end{equation*}
%%
と変形できる. 
}. 
この表現において, 水平運動量はモデル座標の水平面に沿って拡散される. 
スカラー場に対する拡散演算子の場合とは違い, モデル座標の水平面と異なる等値面に沿った
水平粘性の表現を一般に得ることは困難である. 
ただし, デカルト座標系の場合には, スカラー場の拡散演算子のように, 
%%
\begin{equation*}
  \mathcal{D}^{lu}_{\rm lvisc} = \nabla\cdot(\mathcal{R}\nabla u), \;\;
  \mathcal{D}^{lv}_{\rm lvisc} = \nabla\cdot(\mathcal{R}\nabla v)
\end{equation*}
%%
と書くことができる. 

最後に, 水平運動量$\Dvect{U}_h$に対する四次の水平粘性は, 
%%
\begin{equation*}
  \mathcal{D}^{l{\bm U}}_{\rm lvisc,4th} = 
       \nabla_h \left\{ \nabla_h \cdot \left[ A^{lm,4th} \;  \nabla_h (\chi) \right]\right\}
     + \nabla_h \times \{ \Dvect{k}\cdot\nabla\times\left[A^{lm,4th} \; \nabla_h\times(\zeta \Dvect{k}) \right]  \}
\end{equation*}
%%
の形式で与えられる. 
ただし, 4 次の水平粘性もまた, モデル座標の水平面に沿って水平運動量が拡散される場合のみを考えることにする. 


%%
\section{空間離散化}\label{sice_model_space_scheme}
\markright{\arabic{chapter}.\arabic{section} 空間離散化}
%%
\subsection{鉛直離散化}\label{sice_model_vspace_scheme}
\eqref{eq:sice_model_basic_equations}に含まれる鉛直熱拡散項は, 
有限体積法を用いて, 
%%
\begin{subequations}
\begin{gather}
 \dfrac{\rho_I h_I}{2} \left(\DD{E_{I,1}}{t}\right)_{\rm thermodyn} 
 %= \dfrac{\rho_i h_i}{2}\left(C + \dfrac{L\mu S_i}{T_1^2}\right) \DD{T_1}{t} 
 = K_{s1} (T_s - T_{I,1}) - K_{12} (T_{I,1} - T_{I,2}) + I, \\
 %%
 \dfrac{\rho_I h_I}{2} \left(\DD{E_{I,2}}{t}\right)_{\rm thermodyn} 
 %\dfrac{\rho_i h_i}{2} C \DD{T_2}{t}
  = K_{12} (T_{I,1} - T_{I,2}) - 2K_{12} (T_{I,2} - T_f) 
\end{gather}
\end{subequations}
%%
と鉛直離散化する. 
ここで, 上式に含まれる$K_{s1}, K_{12}$は, 
%% 
\begin{align*}
  K_{s1} = \dfrac{4 k_I k_S}{k_S h_I + 4 k_I h_S}, \;\;
  K_{12} = \dfrac{2 k_I}{h_I}
\end{align*}
%%
によって与えられ, $k_S$は雪の熱伝導率, $k_I$は氷の熱伝導率である.  
$K_{s1}$の表現は, 氷層上端での熱伝導フラックスと, 
雪層下端での熱伝導フラックスが等しいと仮定することによって決定される%
\footnote{ 
氷層上端での熱伝導フラックスと, 雪層下端での熱伝導フラックスが等しい条件は, 
雪層と氷層の境界の温度を$T^*$とすると,  
$$
  K_s \dfrac{T_s - T^*}{h_S} = K_i \dfrac{T^* - T_{I,1}}{h_i/4}
$$
と書かれる. この式から, 
$T^* = (K_s h_i T_s + 4 K_i h_s T_1)/(4K_i h_s + K_s h_i)$が得られる. 
したがって, 氷層上層の上端から雪層を通過する熱伝導フラックスの表現として, 
$$
 K_s \dfrac{T_s - T^*}{h_s} 
 = \dfrac{4 K_i K_s}{K_s h_I + 4K_i h_S} (T_s - T_{I,1}) 
$$
が得られ, 右辺の因子を$K_{s1}$と置いていることが分かる.  
}. 

\subsection{水平離散化}\label{sice_model_hspace_scheme}
%%
海氷の方程式系における力学項(現状は水平移流項のみ)の水平離散化に, 有限体積法を適用する.   
$q=h_S, h_I, \rho_I h_I E_{I,1}/2, \rho_I h_I E_{I,2}/2$とすると, 
%%
\begin{subequations}
\begin{gather}
 \left(\DD{q}{t}\right)_{\rm dyn}
 =  
   - \left[ \dfrac{  \delta_i (e_2 u^* \overline{q}^{i,UP1}) 
                   + \delta_j (e_2 v^* \overline{q}^{j,UP1})} {e_1 e_2} 
      \right]_{i,j}
\end{gather}
\end{subequations}
%%
と書かれる. 
ここで, $\overline{q}^{i,UP1}, \overline{q}^{j,UP1}$は, 1次精度の風上フラックスを与える補間を表し,  
例えば, 前者は, 
%%
\begin{equation*}
  \overline{q}^{i,UP1} 
  = \dfrac{q_{i,j} + q_{i+1,j}}{2} 
    + |u^*_{i+\frac{1}{2},j}| (q_{i,j} - q_{i+1,j})
\end{equation*}
%%
と与えられる. 
海氷の水平速度のパラメータリゼーション\eqref{eq:sice_model_horivel_param_thickness_diff}は, 
海氷厚さ(厳密には質量)の式に対して水平拡散と等価となるように, 
%%
\begin{align}
 (u^*)_{i+\frac{1}{2},j} 
 = K_h^s 
 \left(
  \dfrac{\delta_i [m_{\rm sice} ]}{e_1 \; \tilde{m}_{\rm sice}}
 \right)_{i+\frac{1}{2},j}, \;\;\;
%% 
 (v^*)_{i,j+\frac{1}{2}} 
 = K_h^s 
 \left(
  \dfrac{\delta_j [m_{\rm sice} ]}{e_2 \; \tilde{m}_{\rm sice}}
 \right)_{i,j+\frac{1}{2}}
%%
\label{eq:sice_model_horivel_param_thickness_diff_fvm}
\end{align}
%%
によって与えられる. 
ここで, $\tilde{m}_{\rm sice}$は, 例えば$i$方向に対して, 
$$
 (\tilde{m}_{\rm sice})_{i+\frac{1}{2},j} 
 = {\rm max}\left[(\tilde{m}_{\rm sice})_{i,j}, \;\;
                  (\tilde{m}_{\rm sice})_{i+1,j}
  \right]
$$
のように定義される. 



%% 時間離散化
\section{時間離散化}\label{sice_model_time_scheme}
\markright{\arabic{chapter}.\arabic{section} 時間離散化}
%%
\subsection{時間スキームの概要}\label{sice_model_time_scheme_brief}
%%
\subsubsection*{熱力学過程}\label{sice_model_time_scheme_brief_thermodyn}
%%
鉛直熱拡散項と関係した氷層のエンタルピーの時間変化を, 以下の形式で陰的に評価する. 
%%
\begin{equation*}
 \dfrac{\rho_I h_I}{2} \left(\DD{E_{I,k}}{t}\right)_{\rm thermodyn} 
  = \dfrac{\rho_I h_I}{2 \Delta t} 
  \int_{t^n}^{t^*} \left(\DP{E_{I,k}}{T_{I,k}}\right)  \DD{T_{I,k}}{t} \Dd{t}
  = \mathcal{D}_{\rm vdiff}^{vT_{I}} (T_{I,1}^*, T_{I,2}^*, T_s^*)
\end{equation*}
%%
ここで, $k=1,2$であり, 
また$\mathcal{D}_{\rm vdiff}^{vT_{I}} (T_I^*)$は, 鉛直離散化した鉛直熱拡散項を表す. 
この時に, 海氷表面での熱収支式を連立することによって, 海氷の表面温度$T_s^*$も決定される.

次に, 上で計算された鉛直熱伝導フラックスから, 海氷上端での融解量, 海氷下端での生成融解量が, 
\eqref{eq:sice_model_surface_energy_budget}と\eqref{eq:sice_3lyr_model_bottom_energy_budget}によって計算される.
この過程において, 海氷上端の積雪や昇華が考慮される. 

最後に, 海氷内部の厚さの調整が行われる. 
このとき, 雪層から氷層への変化や氷層の上下層の等分化が行われ, それに伴い氷層の温度も変化する.  

\subsubsection*{力学過程}\label{sice_model_time_scheme_brief_dynamics}
%%
熱力学過程後の変数の値を用いて, 移流項に伴う海氷場の時間変化を陽的に評価する. 
$q=h_S, h_I, \rho_I h_I E_{I,1}/2, \rho_I h_I E_{I,2}/2$とすると, 
%%
\begin{equation*}
 \dfrac{q^{n+1} - q^*}{\Delta t}
 = \left(\DD{q}{t}\right)_{\rm dyn} (h_S^*, h_I^*, q^*)
\end{equation*}
%%
と書かれる. 
力学過程において, 氷層のエンタルピーの時間発展式は, 保存形式であることに注意が必要である. 

\subsection{時間スキームの詳細}\label{sice_model_time_scheme_detail}
%%
\subsubsection*{鉛直熱拡散方程式の時間離散化}
%%
海氷の鉛直熱拡散と関係する方程式は, 
%%
\begin{subequations}
\begin{gather}
 \dfrac{\rho_I h_I^n}{2}\left( C + \dfrac{L\mu S_i}{T_{I,1}^* T_{I,1}^n}\right)
 \dfrac{T_{I,1}^* - T_{I,1}^n}{\Delta t} 
 = K_{s1}^n (T_s^* - T_{I,1}^*) - K_{12}^n (T_{I,1}^* - T_{I,2}^*) + I, \\
%%
 \dfrac{\rho_I h_I^n}{2}C \dfrac{T_{I,2}^* - T_{I,2}^n}{\Delta t} 
  = K_{12}^n (T^*_{I,1} - T^*_{I,2}) - 2K_{12}^n (T^*_{I,2} - T_f)
\end{gather}
\label{eq:sice_model_thermodyn_thermo_vdiff_fvm}
\end{subequations}
%%
と時間離散化される%
\footnote{
\eqref{eq:sice_model_thermodyn_thermo_vdiff_fvm}(a) の左辺括弧内の表現を得るために,  
$$
 \frac{1}{\Delta t} \int_{t^*}^{t^n} 
 \left(C + L_i \dfrac{\mu S_i}{T_{I,1}^2} \right) \; 
  \DD{T_{I,1}}{t} \;\Dd{t} 
 =  \left( C + \dfrac{L\mu S_i}{T_{I,1}^*
     T_{I,1}^n}\right)
     \dfrac{T_{I,1}^* - T_{I,1}^n}{\Delta t} 
$$
であることが用いられる. 
}. 
また, \eqref{eq:sice_model_thermodyn_thermo_vdiff_fvm}に加えて, 
時刻$t^n$周りで海面熱フラックスを線形化した, 海氷面で融解が起きない場合の熱収支式
%%
\begin{equation}
	A + B T_s^* = K_{s1} (T_{I,1}^* - T_s^*)
\label{eq:sice_model_thermodyn_sfc_heat_budget_fvm}
\end{equation}
%%
を連立させる. 
ここで, 
%%
\begin{align*}
  A 
  &= F_s(T_s^n) - T_s^n \left(\DP{F_s}{T_s}\right)^n, \;\;
  B
   = \left(\DP{F_s}{T_s}\right)^n
\end{align*} 
%%
とおいた. 
後に記述するように, この方程式系で決定される海表面温度が, 雪もしくは海氷の融点を超える場合には, 
\eqref{eq:sice_model_thermodyn_sfc_heat_budget_fvm}を連立せずに, 海氷面温度を固定した問題に変更する. 

\cite{winton2000reformulated}では,
以下に示すように, \eqref{eq:sice_model_thermodyn_thermo_vdiff_fvm}を解析的に解くことで,  
$T_{I,1}^*$の表現を陽に導いている. 
はじめに, \eqref{eq:sice_model_thermodyn_thermo_vdiff_fvm}(b) から, 
%%
\begin{equation}
 T_{I,2}^* 
 = \dfrac{  2 \Delta t K^n_{12} (T_{I,1}^* + 2 T_f) 
          + \rho_I h_I C_i T_{I,2}^n}
          {6\Delta t K^n_{12} + \rho_I h_I C_i} 
\label{eq:sice_model_thermodyn_icetemp2_new}
\end{equation}
が得られる. 
\eqref{eq:sice_model_thermodyn_sfc_heat_budget_fvm}と\eqref{eq:sice_model_thermodyn_icetemp2_new}を用いて, 
\eqref{eq:sice_model_thermodyn_thermo_vdiff_fvm}(a)
に現れる$T_s^*$と$T_{I,2}^*$を消去すると, 
二次方程式
$A_1 (T_{I,1}^*)^2 + B_1 T_{I,1}^* + C_1 = 0$
が得られ, $T_{I,1}^*$は,
%%
\begin{equation}
 T_{I,1}^* = - \dfrac{B_1 + (B_1^2 - 4 A_1 C_1)^{1/2}}{2 A_1}
\label{eq:sice_model_thermodyn_icetemp1_new}
\end{equation}
%%
と与えられる.  
ここで, 
%%
\begin{align*}
 A_1 
 &= 
   \dfrac{\rho_I h_I}{2\Delta t}C_i
 + K_{12}\dfrac{4\Delta t K^n_{12} + \rho_I h_I C_i}
               {6\Delta t K^n_{12} + \rho_I h_I C_i}
 + \dfrac{K^n_{s1} B}{K^n_{s1} + B}, \\
%%
 B_1
 &= 
   - \dfrac{\rho_I h_I}{2\Delta t} 
       \left( C_i T_{I,1}^n - \dfrac{L_i\mu S_I}{T_{I,1}^n} \right) 
   - I \nonumber \\
 &\;\;\;\; - K_{12}\dfrac{4\Delta t K^n_{12} T_f + \rho_i h_i C_i T_{I,2}^n}
                 {6\Delta t K^n_{12} + \rho_i h_i C_i}
   + \dfrac{K^n_{s1} A}{K^n_{s1} + B}, \\
%%
  C_1 
  &= - \dfrac{\rho_I h_I}{2\Delta t} L_i \mu S_I
\end{align*} 
%%
である. 
これらの係数の符号を考慮すれば, 二次方程式のもう一つの解は正であり, 物理的に不適切であることが分かる.  
以上から, 
はじめに\eqref{eq:sice_model_thermodyn_icetemp1_new}から$T_{I,1}^*$が得られれば, 
\eqref{eq:sice_model_thermodyn_icetemp2_new}から$T_{I,1}^*$, 
\eqref{eq:sice_model_thermodyn_sfc_heat_budget_fvm}から$T_s^*$が求まる. 
 
もし, \eqref{eq:sice_model_thermodyn_icetemp1_new}から得られた$T_s^*$が,
雪の融点(雪が存在する場合)あるいは海氷の融点(雪が存在しない場合)を超えているならば, 
海氷の表面温度を$T_s^*=0$(雪が存在する場合)あるいは$T_s^*=-\mu S_i$(雪が存在しない場合)と固定した問題に変更して, 
鉛直熱拡散方程式を解き直す. 
この場合の二次方程式の係数は, 
%%
\begin{align*}
 A_1 
 &= 
   \dfrac{\rho_I h_I}{2\Delta t}C 
 + K_{12}\dfrac{4\Delta t K^n_{12} + \rho_I h_I C_i}
               {6\Delta t K^n_{12} + \rho_I h_I C_i}
 + K_{s1}^n, \\
%%
 B_1
 &= 
   - \dfrac{\rho_I h_I}{2\Delta t} 
       \left( C_i T_{I,1}^n - \dfrac{L_i\mu S_I}{T_{I,1}^n} \right) 
   - I \nonumber \\
 &\;\;\;\; - K_{12}\dfrac{4\Delta t K^n_{12} T_f + \rho_I h_I C_i T_{I,2}^n}
                 {6\Delta t K^n_{12} + \rho_I h_I C_i}
   - K_{s1}^n T_s^*
%%
\end{align*}
%%
である. 
ただし, $C_1$は変わらない. 

ここで陰的に求められた熱伝導フラックスを用いて, 海氷上下端での生成融解量が計算される. 



%% パラメータの設定例
\section{パラメータの設定例}
%%
海氷モデルのパラメータ設定の例として, リスト\cite{dogcm_Pl64L60_conf_sice}に, 
\cite{ykawai2018_Dthesis}で行なった海惑星の気候実験における設定を示す. 
なお, 海洋海氷モデルに入力する設定ファイルの中で, 海氷モデルと関係する部分のみを以下では示している.
%%%%%%
\begin{lstlisting}[caption={
海氷モデルの設定ファイルのテンプレート.
ただし, \#var\#のように書かれた部分は実験ケース等ごとに適切な値が入る.
},label={dogcm_Pl64L60_conf_sice},
basicstyle=\ttfamily\footnotesize,
frame=single]
&DOGCM_nml
  OCN_do  = .true.,
  SICE_do = .true.,
  exp_name = '#DOGCM_nml_EXP_NAME#',
/
!= 格子の設定 ----------------------------------------
&SeaIce_Grid_nml
 IM = 1,
 JM = 64
/
!= 物理定数の設定 ----------------------------------------
&SeaIce_Admin_Constants_nml
 AlbedoOcean   = 0d0, 
 EmissivOcean  = 1d0,
 EmissivSnow   = 1d0,
 EmissivIce    = 1d0,
 IceMaskMin    = 0.999999999999999999999999999999d0,
 IceThickMin   = 1d-2, 
 SIceHDiffCoef = #SeaIce_Admin_Constants_nml_SIceHDiffCoef#d0, 
                 ! 海氷厚さの水平拡散係数
/
!= 表面アルベドの計算方法の設定 ----------------------------------------
&SeaIce_Boundary_SfcAlbedo_nml
 SIceSfcAlbedoName = 'I07SGS',
 AlbedoSnowWarm    = 0d0,
 AlbedoSnowCold    = 0.5d0,
 TLVTempAlbedoSnow = -10d0, 
 SFCALBEDO_I07SGS_FlagIceAreaFrac = .true.,  
                       ! 表面アルベドの工夫を適用するかのフラグ
/
!= ファイル関連の設定 ----------------------------------------
!* ヒストリデータ出力の全体設定 
&SeaIce_IO_History_nml
  IntValue      = #gtool_historyauto_nml_IntValue#,  ! 出力間隔の数値
  IntUnit       = 'day',                             ! 出力間隔の単位
  FilePrefix    = '',
/
!* リスタートデータの入力・出力の設定
&SeaIce_IO_Restart_nml
 OutputFileName = '#SIceRestartFile_nml_OutputFileName#', 
 InputFileName  = '#SIceRestartFile_nml_InputFileName#', 
 IntValue       = #SIceRestartFile_nml_IntValue#, 
 IntUnit        = 'day'
/
!* ヒストリデータ出力の個別設定
&SeaIce_IO_History_nml
   Name = 'SIceEn, SIceV, SIceCon, SnowThick, IceThick, SIceTemp, SIceSfcTemp,SfcAlbedoAI', 
/
&SeaIce_IO_History_nml
  Name = 'Wice, SfcHFlxAI, DelSfcHFlxAI, SfcHFlxAI0_ns, SfcHFlxAI0_sr, DSfcHFlxAIDTs, SfcHFlxAO, DSfcHFlxAODTs', 
  TimeAverage = .true.
\end{lstlisting}




%% 参考文献
\renewcommand{\bibname}{参考文献}
\markright{参考文献} %  節の題名を書
\bibliography{reference}

\end{document}

