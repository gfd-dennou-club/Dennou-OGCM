\chapter*{付録 A: 海氷の比熱の式の導出}
ここでは, 海氷熱力学モデルで用いられる海氷の比熱の近似式
%%
\begin{equation*}
 c_{\rm si} = c_o + \dfrac{L \mu S}{T^2}
  \tag{A.1}
\end{equation*}
%%
の導出を\citep{ono1967specific}に従って行う.
ただし, $L$は 0$\circ$C における純氷の融解の潜熱,
$\mu$は経験的な定数(以下の導出を参照), $S$は海氷の塩分量である. 
本文と同様に以下でも, 温度の単位として摂氏を用いることに注意されたい. 

今, 海氷は純氷と濃縮された塩類溶液(ブライン)から成るとし, 気泡などの存在無視すれば, 
海氷 1 g の温度を$\Delta T$ $^\circ$C 上昇させるための熱量$C_{si}\Delta T$は,
%%
\begin{equation*}
 C_{si} \Delta T = C_i m_i \Delta T + C_b m_b \Delta T - \lambda_T \DD{m_i}{T} \Delta T
 \tag{A.2}
\end{equation*}
%%
と書ける.
ここで, $m_i, m_b$はそれぞれ海氷 1 g 中の純氷およびブラインの質量である.
また, $C_i$は純氷の比熱, $C_b$はブラインの比熱である.
$\lambda_T$は$T$ $^\circ$C における氷の融解の潜熱であり,
%%
\begin{equation*}
 \lambda_T = \lambda_0 + (C_w - C_i)T
 \tag{A.3}
\end{equation*}
%%
表される.
すなわち, $\lambda_T$は, $T$$^\circ$C の純氷 1 g をまず 0$^\circ$C まで暖め,
0 $^\circ$C で溶かし, 溶けた水の温度をまた $T$$^\circ$C まで下げるのに必要な熱量に等しい.
比熱は 1 g の物質の温度を 1$^\circ$C 上昇させるのに必要な熱量で定義されるので,
(A.2) から海水の比熱は,
%%
\begin{equation*}
 C_{si} = C_i m_i + C_b m_b - \lambda_T \DD{m_i}{T}
 \tag{A.4}
\end{equation*}
%%
と書ける.

海氷 1 g 中のブライン$m_b$ gは, $m_w$ g の水に$m_s$ g の塩類%
\footnote{
$m_s$は海氷中の塩類の質量分率であり, 塩分量$S$(単位はパーミル)とは,
%%
$$
 S = m_s \times 1000 
$$
%%
の関係がある.  
}
が溶けてできているとする.
このとき,
%
\begin{equation*}
 m_i + m_b = m_i + m_w + m_s = 1
 \tag{A.5}
\end{equation*}
%%
である.
比熱の中に現れるブライン中の水の質量$m_w$を$m_s$を使って書くために,
塩分の質量比
%%
\begin{equation*}
 \gamma = \dfrac{m_s}{m_w}
 \tag{A.6}
\end{equation*}
%%
を導入すれば, $m_w$は,
%%
\begin{equation*}
 m_w = \dfrac{m_s}{\gamma}
 \tag{A.7}  
\end{equation*}
%%
と書ける.
同様にして, 純粋の質量$m_i$は,
%%
\begin{equation*}
 m_i = 1 - m_s - \dfrac{m_s}{\gamma}
 \tag{A.8}  
\end{equation*}
%%
と書ける.

ブラインの比熱$C_b$は, 塩分量が増加すると小さくなり,
温度が下がるとわずかに大きくなる. したがって, ブラインの比熱を
%%
\begin{equation*}
 C_b = C_w  - \beta(S,T)
 \tag{A.9}  
\end{equation*}
%%
の形式で書くことができる.

(A.4) に (A.5),(A.7),(A.8),(A.9) を代入すれば,
%%
\begin{equation*}
 C_{\rm si} =
  C_i - \lambda_0 \DD{m_i}{T}
  + (C_w - C_i)\left(\dfrac{m_s}{\gamma} - T\DD{m_i}{T} \right)
  + (C_w - C_i - \beta)m_s - \beta\dfrac{m_s}{\gamma}
  \tag{A.10}
\end{equation*}
%%
を得る.
さらに,
$m_s$が$T$に依存しないことに注意すれば, 純氷の変化量は,
%%
$$
 \DD{m_i}{T} = - \DD{m_w}{T} = - \DD{}{T}\left(\dfrac{m_s}{\gamma}\right)
 = \dfrac{m_s}{\gamma^2}\DD{\gamma}{T}
 $$
 と$m_s$と$\gamma$を用いて書き表されるので, (A.10) は,
 %%
 \begin{equation*}
 C_{\rm si} =
  C_i - \lambda_0 \dfrac{m_s}{\gamma^2}\DD{\gamma}{T}
  + (C_w - C_i) \dfrac{m_s}{\gamma} \left(1 - \dfrac{T}{\gamma}\DD{\gamma}{T} \right)
  + (C_w - C_i - \beta)m_s - \beta\dfrac{m_s}{\gamma}
  \tag{A.10}  
 \end{equation*}
 %%
 となる.

 温度 $T ^\circ$C を結氷温度とするような平衡濃度にあるブラインの質量比$\gamma$と
 $T$には,
 %%
 \begin{equation*}
  T = - \alpha \gamma
  \tag{A.11}
 \end{equation*}
 %%
 のような線形の関係が成り立つことが知られている.
 -8.2 $^\circ$C 以上の温度範囲を考えるとき, 
 比例定数$\alpha$は 54.11 に取られる.
 (A.11)を(A.10)に代入すれば,
 %%
 \begin{equation*}
 C_{\rm si} =
  C_i + \lambda_0 \alpha \dfrac{m_s}{T^2}
  + (C_w - C_i - \beta)m_s + \alpha \beta\dfrac{m_s}{T}
  \tag{A.10}  
 \end{equation*}
 %%
 を得る.
 今考える温度範囲では, 右辺 3,4 項目の寄与は 2 項目に比べて小さい\citep{ono1967specific}ので
 無視すれば, 海氷の比熱は,
 %%
 \begin{equation*}
 C_{\rm si} =
  C_i + \lambda_0 \mu \dfrac{S}{T^2}
  \tag{A.10}  
 \end{equation*}
 %%
 と書ける.
 ただし, $m_s$を塩分量$S$で書き直し, $\mu = \alpha \times 1000$と置き直した. 