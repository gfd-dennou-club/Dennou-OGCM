%#BIBTEX bibtex main
%
% Dennou-SOGCM
%
%   2013/11/12  電脳海洋モデル 定式化
%               河合 佑太
%
% style  Setting             %%%%%%%%
% フォント: 12point (最大), 片面印刷
\documentclass[a4j,12pt,openbib,oneside]{jreport}

%%%%%%%%%%%%%%%%%%%%%%%%%%%%%%%%%%%%%%%%%%%%%%%%%%%%%%%%
%%%%%%%%             Package Include            %%%%%%%%

\usepackage{ascmac}
\usepackage{tabularx}
\usepackage{graphicx}
\usepackage{amssymb}
\usepackage{amsmath}
\usepackage{mathrsfs}
\usepackage{mathtools}
\usepackage{Dennou6}		% 電脳スタイル ver 6
\usepackage[round]{natbib}

%%%%%%%%%%%%%%%%%%%%%%%%%%%%%%%%%%%%%%%%%%%%%%%%%%%%%%%%
%%%%%%%%            PageStyle Setting           %%%%%%%%
\pagestyle{DAmyheadings}

%%%%%%%%%%%%%%%%%%%%%%%%%%%%%%%%%%%%%%%%%%%%%%%%%%%%%%%%
%%%%%%%%        Title and Auther Setting        %%%%%%%%
%%
%%  [ ] はヘッダに書き出される.
%%  { } は表題 (\maketitle) に書き出される.

\Dtitle{Dennou-SOGCM}   % 変更不可
\Dauthor{河合佑太}        % 筆者
\Ddate{2015/9/30}        % 変更日時 (毎回変更すること)
\Dfile{formulatiom.tex}

%%%%%%%%%%%%%%%%%%%%%%%%%%%%%%%%%%%%%%%%%%%%%%%%%%%%%%%%
%%%%%%%%   Set Counter (chapter, section etc. ) %%%%%%%%
\setcounter{chapter}{0}    % 章番号
\setcounter{section}{0}    % 節番号
\setcounter{subsection}{0}    % 節番号
\setcounter{equation}{0}   % 式番号
\setcounter{page}{1}     % 必ず開始ページは明記する
\setcounter{figure}{0}     % 図番号
\setcounter{table}{0}      % 表番号
%\setcounter{footnote}{0}


%%%%%%%%%%%%%%%%%%%%%%%%%%%%%%%%%%%%%%%%%%%%%%%%%%%%%%%%
%%%%%%%%        Counter Output Format           %%%%%%%%
\def\thechapter{\arabic{chapter}}
\def\thesection{\arabic{chapter}.\arabic{section}}
\def\thesubsection{\arabic{chapter}.\arabic{section}.\alph{subsection}}
\def\theequation{\arabic{chapter}.\arabic{section}.\arabic{equation}}
\def\thepage{\arabic{page}}
\def\thefigure{\arabic{chapter}.\arabic{section}.\arabic{figure}}
\def\thetable{\arabic{chapter}.\arabic{section}.\arabic{table}}
\def\thefootnote{*\arabic{footnote}}

%%%%%%%%%%%%%%%%%%%%%%%%%%%%%%%%%%%%%%%%%%%%%%%%%%%%%%%%
%%%%%%%%        Dennou-Style Definition         %%%%%%%%

%% 改段落時の空行設定
\Dparskip      % 改段落時に一行空行を入れる
%\Dnoparskip    % 改段落時に一行空行を入れない

%% 改段落時のインデント設定
\Dparindent    % 改段落時にインデントする
%\Dnoparindent  % 改段落時にインデントしない

%%%%%%%%%%%%%%%%%%%%%%%%%%%%%%%%%%%%%%%%%%%%%%%%%%%%%%%%%%
%% Macro defined by author
\def\univec#1{ \hat{ \Dvect{\rm #1}} }
%\definecolor{shadecolor}{gray}{0.88}
\def\DD#1#2{\frac{{\rm D} #1}{{\rm D} #2}}
\def\dd#1#2{\frac{{\rm d} #1}{{\rm d} #2}}
\def\Dd#1{\; {\mathrm d} #1}

%%%%%%%%%%%%%%%%%%%%%%%%%%%%%%%%%%%%%%%%%%%%%%%%%%%%%%%
%%%%%%%             Text Start                 %%%%%%%%

\begin{document}
\chapter{定式化(数理表現)} %%%%%%%%%%%%%%%%%%
% 基礎方程式系の提示
\section{基礎方程式系}
海洋大循環を支配する方程式はブジネスクプリミティブ方程式であり,
以下のように書かれる.
%%
\begin{subequations}
\begin{align}
 &   \DD{\Dvect{u}}{t}  +  f\Dvect{k} \times \Dvect{u}
 =
   -  \dfrac{1}{\rho_0} \nabla p
   + \nabla_h \cdot  (A_h \nabla_h \Dvect{u})
   + \DP{}{z} \left(A_v \DP{\Dvect{u}}{z}\right)
   + \Dvect{\mathscr{F}}_{\Dvect{u}},
 \label{eq:HydroBoussinesqPrimitive_MomUV}  
\\%%&&&&&&&&&&&&&&&&&&&&&&&&&&&&&&&&&&&&&&&&&&&&&&&&&&&&&&&&& 
 &   \DP{p}{z}
 = - g \dfrac{\rho}{\rho_0},
 \label{eq:HydroBoussinesqPrimitive_MomW} 
\\%%&&&&&&&&&&&&&&&&&&&&&&&&&&&&&&&&&&&&&&&&&&&&&&&&&&&&&&&&& 
 & \nabla \cdot \Dvect{u} + \DP{w}{z}                 
 = 0,
 \label{eq:HydroBoussinesqPrimitive_Continu} 
\\%%&&&&&&&&&&&&&&&&&&&&&&&&&&&&&&&&&&&&&&&&&&&&&&&&&&&&&&&&&
 &   \DD{\Theta}{t}
 =   \nabla_h \cdot (K_h \nabla_h \Theta)
   + \DP{}{z} \left(K_v\DP{\Theta}{z}\right)
   + {\mathscr{F}}_{\Theta},
 \label{eq:HydroBoussinesqPrimitive_PotTemp} 
\\%%&&&&&&&&&&&&&&&&&&&&&&&&&&&&&&&&&&&&&&&&&&&&&&&&&&&&&&&&&
 &   \DD{S}{t}
 =   \nabla_h \cdot (K_h \nabla_h S)
   + \DP{}{z} \left(K_v\DP{S}{z}\right)
   + {\mathscr{F}}_{S},
 \label{eq:HydroBoussinesqPrimitive_Salinity} 
\\%%&&&&&&&&&&&&&&&&&&&&&&&&&&&&&&&&&&&&&&&&&&&&&&&&&&&&&&&&&
 &   \rho = \tilde{\rho}(\Theta,S,z).
 \label{eq:HydroBoussinesqPrimitive_EOS}
\end{align}
\label{eq:HydroBoussinesqPrimitive}
\end{subequations}
%%
ここで,
$\Dvect{u}$ は水平速度ベクトル,
$w$ は鉛直速度, 
$f$ はコリオリパラメータ,
$\Dvect{k}$ は鉛直方向の単位ベクトル,
$g$ は重力加速度,
$\rho_0$ は基準密度(定数),
$\rho$ は密度,
$p$ は圧力,
$\Theta$ は温位,
$S$ は塩分である.
$A_h, A_v$ はそれぞれ水平粘性係数, 鉛直粘性係数,
$K_h, K_v$ はそれぞれ水平拡散係数, 鉛直拡散係数である.
また, $\Dvect{\mathscr{F}}_{\Dvect{u}}, \mathscr{F}_\Theta, \mathscr{F}_S$ は強制項を表す. 
最後に,
$\nabla_h$ は水平勾配演算子, 
$d/dt$ は物質微分
%%
\begin{align}
   \DD{}{t}
 = \DP{}{t} + \Dvect{u}\cdot\nabla_h + w\DP{}{z}
\end{align}
%%
である.
状態方程式\eqref{eq:HydroBoussinesqPrimitive_EOS}の左辺の飾り無しの$\rho$は
時間・空間上の一点での密度$\rho(\Dvect{x},t)$を表す.
一方, 右辺の$\tilde{\rho}$は$(\Theta,S,p)$を従属変数とする熱力学関数であることを表す.
なお, 状態方程式の圧力依存性は, 海水のブジネスク近似の一貫性のために$z$依存性に置き換える
(すなわち, 状態方程式から密度を求めるとき, 圧力は$-\rho_0 gz$とする). 


\section{境界条件}
\subsection*{速度に対する境界条件}
海洋の上端(海面)$z=\eta$と下端$z=-H(\Dvect{x}_h)$における速度の境界条件は,
運動学的境界条件(境界を横切る流れは無い)
%%
\begin{subequations} %17:12の式群
\begin{align}
 &    \DP{\eta}{t}  +  \Dvect{u} \cdot \nabla \eta
    + E - P
 =  w
 \;\;\; {\rm on} \;\; z = \eta,
 \label{eq:KinematicBCTop}
\\%%%%%%%%%%%%%%%%%%%%%%%%%%%%%%%%%%%%%%%%%%%%%%%
 &  \Dvect{u} \cdot \nabla H
 = - w
 \;\;\; {\rm on} \;\; z = - H,
 \label{eq:KinematicBCBtm} 
\end{align}
\end{subequations}
%%
および力学的境界条件
%%
\begin{subequations} %17:11の式群
\begin{align}
 {\rm (応力指定の場合)}&
 \nonumber
\\%%%%%%%%%%%%%%%%%%%%%%%%%%%%%%%%%%%%%%%%%%%%%%%%%%%%%%%%%%%%%%%%%%%%%%%%%%%%%%%%%
 &  \left(A_h \nabla \Dvect{u}^T + A_v \DP{\Dvect{u}^T}{z} \right) \cdot \Dvect{n}
 =  \dfrac{\Dvect{\tau}_s}{\rho_0}
 \;\; {\rm on} \;\; z = \eta, 
\label{eq:DynamicBCTop}
\\%%%%%%%%%%%%%%%%%%%%%%%%%%%%%%%%%%%%%%%%%%%%%%%%%%%%%%%%%%%%%%%%%%%%%%%%%%%%%%%%%
 &  \left(A_h \nabla \Dvect{u}^T + A_v \DP{\Dvect{u}^T}{z} \right) \cdot \Dvect{n}
 =  \dfrac{\Dvect{\tau}_b}{\rho_0}
 \;\;\; {\rm on} \;\; z = - H 
\label{eq:DynaimcBCBtm}
\\%%%%%%%%%%%%%%%%%%%%%%%%%%%%%%%%%%%%%%%%%%%%%%%%%%%%%%%%%%%%%%%%%%%%%%%%%%%%%%%%%
  {\rm (値指定の場合)}&
  \nonumber
\\%%%%%%%%%%%%%%%%%%%%%%%%%%%%%%%%%%%%%%%%%%%%%%%%%%%%
  & \Dvect{u} = \Dvect{u}_s
\;\; {\rm on} \;\; z = \eta, 
\\%%%%%%%%%%%%%%%%%%%%%%%%%%%%%%%%%%%%%%%%%%%%%%%%%%%%%%%%%%%%%%%%%%%%%%%%%%%%%%%%%
  & \Dvect{u} = \Dvect{u}_b
  \;\;\; {\rm on} \;\; z = - H 
\end{align}
\end{subequations}
%%
である.
ここで,
$E$ は海面における蒸発量,
$P$ は海面における降水量, 
$\Dvect{\tau}_s, \Dvect{\tau}_b$ はそれぞれ上端・下端で指定する(風)応力,
$\Dvect{u}_s, \Dvect{u}_b$ はそれぞれ上端・下端で指定する流速, 
$\Dvect{n}$ は境界の法線ベクトル,
$h(\Dvect{x})$ は海洋が静止しているときの深さである. 
$\Dvect{u}^T$ は$\Dvect{u}$の転置を表す.
また, 水平境界における速度の境界条件は,
%%
\begin{subequations} %17:33の式群
 \begin{align}
  {\rm (応力なし条件の場合)}&
  \nonumber
\\%%%%%%%%%%%%%%%%%%%%%%%%%%%%%%%%%%%%%%%%
  &\Dvect{u} \cdot \Dvect{n} = 0 \;\; {\rm かつ} \;\;
   \nabla (\Dvect{u} \cdot \Dvect{t}) \cdot \Dvect{n} = 0, 
\\%%%%%%%%%%%%%%%%%%%%%%%%%%%%%%%%%%%%%%%%  
  {\rm (滑りなし条件の場合)}&
  \nonumber
\\%%%%%%%%%%%%%%%%%%%%%%%%%%%%%%%%%%%%%%%%
  &\Dvect{u} = 0
 \end{align}
\end{subequations}
%%
を課す. 
ここで, $\Dvect{t}$ は境界の水平方向の接ベクトルである.
今, 水平境界面における$w$は連続の式から診断的に求められるので,
そこで$w$の境界条件は必要でない. 

一方, トレーサー$T_r$(温位や塩分)に対する境界条件は, 
海洋の境界において, 
%%
\begin{subequations} %17:11の式群
\begin{align}
 {\rm (フラックス指定の場合)}&
 \nonumber
\\%%%%%%%%%%%%%%%%%%%%%%%%%%%%%%%%%%%%%%%%%%%%%%%%%%%%%%%%%%%%%%%%%%%%%%%%%%%%%%%%%
 &  \left(K_h \nabla T_r + K_v \DP{T_r}{z} \Dvect{k} \right) \cdot \Dvect{n}
 =  F_{T_r}
\\%%%%%%%%%%%%%%%%%%%%%%%%%%%%%%%%%%%%%%%%%%%%%%%%%%%%%%%%%%%%%%%%%%%%%%%%%%%%%%%%%
  {\rm (値指定の場合)}&
  \nonumber
\\%%%%%%%%%%%%%%%%%%%%%%%%%%%%%%%%%%%%%%%%%%%%%%%%%%%%
  & T_r  = T_{r, {\rm boundary}}
\end{align}
\end{subequations}
%%
を課す.
ここで, $F_{T_r}$は境界で指定するトレーサーのフラックス,
$T_{r, {\rm boundary}}$は境界で指定するトレーサーの値である.


% 基礎方程式系の変形1
\section{基礎方程式系の変形 1}
上で挙げた方程式系を, 本海洋モデルにおいて解かれる式の形式に変形する.
本章では, 水平座標は陽に指定せず水平方向の微分演算子はベクトル形式で記述し, 
鉛直座標に関しては$z$座標系で方程式系を記述する.   


密度$\rho$は,
%%
\begin{align}
 \rho(\Dvect{x},t) = \rho_r(z,t)    +  \delta \rho(\Dvect{x},t), \;\;
\end{align}
のように参照密度$\rho_r$とそれからのずれに分解する.
また, 圧力も同様に,  
\begin{align}
 p(\Dvect{x},t)    = p_r(z,t) + p_s(\Dvect{x}_h,t)  + \delta p(\Dvect{x},t)  
\end{align}
%%
のように, 参照圧力$p_r$, 海面水位の寄与$p_s$, そしてそれらからのずれに分解する. 
ただし, 参照圧力と参照密度は, 静水圧平衡
%%
\begin{align}
  \DP{p_r}{z} = - \rho_r g
\end{align}
%%
を満たす. 
以下では, 簡潔に記述するために, 
$\delta \rho, \delta p$をそれぞれ$\rho, p$と書くことにし,  
密度・圧力の全体はそれぞれ$\rho_{\rm tot}, p_{\rm tot}$と書く. 

水平方向の運動量方程式\eqref{eq:HydroBoussinesqPrimitive_MomUV}は,
ベクトル不変形
%%
\begin{align}
 \DP{\Dvect{u}}{t}
 =&  - (f + \zeta) \Dvect{k} \times \Dvect{u}
     - g \nabla_h \eta
     -  \dfrac{1}{\rho_0} \nabla_h \; p
     -  \nabla_h \left(\dfrac{\Dvect{u}^2}{2} \right)
  \nonumber \\
  &  +   \nabla_h \cdot  (A_h \nabla_h \Dvect{u})
     +   \DP{}{z} \left(A_v \DP{\Dvect{u}}{z}\right). 
 \label{eq:HydroBoussinesqPrimitive_MomUV_invariantForm}   
\end{align}
%%
に書き直す.
ここで, $\zeta=\Dvect{k}\cdot\nabla_h\times\Dvect{u}$であり,
渦度の鉛直成分である. 
特に, 岸がない場合(水惑星設定の場合)には, 水平方向の運動量の時間発展は,
スペクトル変換法に基づく大気大循環モデルに習って, 
渦度方程式と発散方程式から求める%
\footnote{
(memo) 引用
}.
これらの方程式は以下のように導かれる. 
\eqref{eq:HydroBoussinesqPrimitive_MomUV_invariantForm}の両辺に$\Dvect{k}\times\nabla_h$を
作用させれば, 渦度方程式
%%
\begin{align}
 \DP{\zeta}{t}
 =
 & - \nabla_h \cdot \left[ (\zeta + f) \Dvect{u} \right] \\
 & + \Dvect{k} \cdot \nabla_h \times \left[
       -  w \DP{\Dvect{u}}{z}
       + \Dvect{\mathscr{D}}_{\Dvect{u}} + \Dvect{\mathscr{F}}_{\Dvect{u}}
    \right]
\label{eq:HydroBoussinesqPrimitive_Vor}
\end{align}
%%
を得る.
また, \eqref{eq:HydroBoussinesqPrimitive_MomUV_invariantForm}の両辺に$\nabla_h \cdot \;$を作用させれば, 発散方程式
%%
\begin{align}
    \DP{D}{t}
 =
 &   \Dvect{k} \cdot \nabla_h \times \left[ (\zeta + f) \Dvect{u} \right] \\
 & - \Delta_h \left( g\eta + \dfrac{p}{\rho_0} + \dfrac{\Dvect{u}^2}{2}  \right) \\
 & + \nabla_h \cdot \left[
       -  w \DP{\Dvect{u}}{z}
       + \Dvect{\mathscr{D}}_{\Dvect{u}} + \Dvect{\mathscr{F}}_{\Dvect{u}}
    \right]
\label{eq:HydroBoussinesqPrimitive_Div}
\end{align}
%%
を得る.
ここで, $D=\nabla_h \cdot \Dvect{u}$は水平発散である. 
なお, 水平方向の運動量方程式における粘性項は, $\Dvect{\mathscr{D}}_{\Dvect{u}}$で表した. 


鉛直速度は, 連続の式\eqref{eq:HydroBoussinesqPrimitive_Continu}から診断され,
%%
\begin{align}
  w(\Dvect{x}_h,z,t) = - \nabla_h \cdot \int_{-H}^z \Dvect{u}(\Dvect{x}_h,z',t) \Dd{z'} 
\label{eq:W_dianogstic}
\end{align}
%%
によって求められる%
\footnote{
連続の式\eqref{eq:HydroBoussinesqPrimitive_Continu}を$-H$から$z$まで積分すれば,
$$
    w(\Dvect{x}_h,z,t) - w(\Dvect{x}_h,-H,t)
  = - \int_{-H}^z \nabla_h \cdot \Dvect{u} (\Dvect{x}_h,z',t) \; \Dd{z'}
$$
を得る.
ここで, ライプニッツの公式を用いれば, 上の式の右辺は, 
$$
     - \int_{-H}^z \nabla_h \cdot \Dvect{u} \Dd{z'}
   = - \left(
         \nabla_h \cdot \int_{-H}^z  \Dvect{u} \Dd{z'}
       +  \Dvect{u} \cdot \nabla_h (-H)
     \right)
$$
と変形される.
また, $w(\Dvect{x}_h,-H,t)$は海底における境界条件\eqref{eq:KinematicBCBtm}から与えられるので,
これらを考慮すれば\eqref{eq:W_dianogstic}が得られる. 
}. 
静水圧平衡の式\eqref{eq:HydroBoussinesqPrimitive_MomW}は, $\rho, p$を用いて, 
%%
\begin{align}
  \DP{p}{z} = - \rho g
\end{align}
%%
と書け形式は変わらない.
$p$はこの静水圧平衡の式を鉛直方向に積分することで求められる.
すなわち,
%%
\begin{align}
  p(\Dvect{x}_h,z,t) = - \int_z^0 \rho(\Dvect{x}_h,z',t) g \; \Dd{z'}. 
\end{align}
%%

本モデルの海面水位の時間発展の求め方は, 線形化した海面の運動学的境界条件%
\footnote{
(Memo)Marshall(??)を引用
}%
に基づく.
すなわち, 
%%
\begin{align}
 \DP{\zeta}{t} =  P - E + w(\Dvect{x}_h,\zeta,t)
\end{align}
%%
右辺の$w$を\eqref{eq:W_dianogstic}から求めれば, 
線形化した海面水位の時間発展の式は, 
%%
\begin{align}
 \DP{\zeta}{t}
 = - \nabla_h \int_{-H}^\eta \Dvect{v} \Dd{z}
   + P - E
\label{eq:LinearizedSeaSurfElev}
\end{align}
%%
と書かれる. 

rigid-lid 近似を適用する場合には,
海面$z=\eta$において$w=0$を課す. 
このとき, 水平速度$\Dvect{u}$は, 
%%
\begin{align}
 0 = \nabla_h \cdot \int_{-H}^0 \Dvect{u} \Dd{z}
\label{eq:RigidLidApprox_Constraint}
\end{align}
%%
を満たさなければならない. 
rigid-lid 近似を用いる場合には, 海面水位の時間発展が陽に扱われなくなる. 
そのため, 海面水位の偏差と関係した圧力$p_s$は, \eqref{eq:RigidLidApprox_Constraint}が満たされるように決定される.

最後に, トレーサー(温位や塩分)の方程式\eqref{eq:HydroBoussinesqPrimitive_PotTemp},\eqref{eq:HydroBoussinesqPrimitive_Salinity}は, 以下のように変形したものを使う.
\begin{align}
 \DP{\Theta}{t}
 = &- \nabla_h \cdot (\Theta \Dvect{u}) + \Theta D + w\DP{\Theta}{z} \nonumber \\
   &+ \mathscr{D}_{\Theta}
    + \mathscr{F}_{\Theta}  
\label{eq:HydroBoussinesqPrimitive_PotTemp_2}
\end{align}
\begin{align}
 \DP{S}{t}
 = &- \nabla_h \cdot (S \Dvect{u}) + \Theta D + w\DP{S}{z} \nonumber \\
   &+ \mathscr{D}_{S}
    + \mathscr{F}_{S}  
\label{eq:HydroBoussinesqPrimitive_Salt_2}
\end{align}
ただし, トレーサーの方程式における拡散項は, $\mathscr{D}_\Theta, \mathscr{D}_S$によって表した. 

% 基礎方程式系の変形2 (座標系の導入(水平座標:局所直交座標, 鉛直座標:一般鉛直座標系))
\section{基礎方程式系の変形 2}
\subsection{鉛直座標系に対する一般鉛直座標系の導入}
本節では, 海面や地形に沿った座標系($\sigma$座標系)などにおける支配方程式を導くために,
一般鉛直座標系を用いた海洋モデルの支配方程式系を記述しておく. 海洋モデルにおける一般鉛直座標系に関しては,
付録?? や\cite{}を参照されたい.

\subsection{水平座標に対する局所直交座標系の導入}

%%%%%%%%%%%%%%%%%%%%%%%%%%%%%%%%%%%%%%%%%%%%%%%%%%%%%%%%%
\chapter{定式化(離散表現)} %%%%%%%%%%%%%%%%%%
%%
% Dennou-SOGCM
%
%   2013/05/30  海洋モデル 定式化
%               河合 佑太
%
% style  Setting             %%%%%%%%
% フォント: 12point (最大), 片面印刷
\documentclass[a4j,12pt,openbib,oneside]{jreport}

%%%%%%%%%%%%%%%%%%%%%%%%%%%%%%%%%%%%%%%%%%%%%%%%%%%%%%%%
%%%%%%%%             Package Include            %%%%%%%%

\usepackage{ascmac}
\usepackage{tabularx}
\usepackage{graphicx}
\usepackage{amssymb}
\usepackage{amsmath}
\usepackage{mathrsfs}
\usepackage{mathtools}
\usepackage{Dennou6}		% 電脳スタイル ver 6
%%%%%%%%%%%%%%%%%%%%%%%%%%%%%%%%%%%%%%%%%%%%%%%%%%%%%%%%
%%%%%%%%            PageStyle Setting           %%%%%%%%
\pagestyle{DAmyheadings}

%%%%%%%%%%%%%%%%%%%%%%%%%%%%%%%%%%%%%%%%%%%%%%%%%%%%%%%%
%%%%%%%%        Title and Auther Setting        %%%%%%%%
%%
%%  [ ] はヘッダに書き出される.
%%  { } は表題 (\maketitle) に書き出される.

\Dtitle{Dennou-SOGCM}   % 変更不可
\Dauthor{河合佑太}        % 担当者の名前
\Ddate{2013/11/14}        % 変更日時 (毎回変更すること)
\Dfile{formulatiom.tex}

%%%%%%%%%%%%%%%%%%%%%%%%%%%%%%%%%%%%%%%%%%%%%%%%%%%%%%%%
%%%%%%%%   Set Counter (chapter, section etc. ) %%%%%%%%
\setcounter{chapter}{1}    % 章番号
\setcounter{section}{1}    % 節番号
\setcounter{subsection}{1}    % 節番号
\setcounter{equation}{1}   % 式番号
\setcounter{page}{1}     % 必ず開始ページは明記する
\setcounter{figure}{0}     % 図番号
\setcounter{table}{0}      % 表番号
%\setcounter{footnote}{0}


%%%%%%%%%%%%%%%%%%%%%%%%%%%%%%%%%%%%%%%%%%%%%%%%%%%%%%%%
%%%%%%%%        Counter Output Format           %%%%%%%%
\def\thechapter{\arabic{chapter}}
\def\thesection{\arabic{chapter}.\arabic{section}}
\def\thesubsection{\arabic{chapter}.\arabic{section}.\alph{subsection}}
\def\theequation{\arabic{chapter}.\arabic{section}.\arabic{equation}}
\def\thepage{\arabic{page}}
\def\thefigure{\arabic{chapter}.\arabic{section}.\arabic{figure}}
\def\thetable{\arabic{chapter}.\arabic{section}.\arabic{table}}
\def\thefootnote{*\arabic{footnote}}

%%%%%%%%%%%%%%%%%%%%%%%%%%%%%%%%%%%%%%%%%%%%%%%%%%%%%%%%
%%%%%%%%        Dennou-Style Definition         %%%%%%%%

%% 改段落時の空行設定
\Dparskip      % 改段落時に一行空行を入れる
%\Dnoparskip    % 改段落時に一行空行を入れない

%% 改段落時のインデント設定
\Dparindent    % 改段落時にインデントする
%\Dnoparindent  % 改段落時にインデントしない

%%%%%%%%%%%%%%%%%%%%%%%%%%%%%%%%%%%%%%%%%%%%%%%%%%%%%%%%%%
%% Macro defined by author
\def\univec#1{ \hat{ \Dvect{\rm #1}} }

%%%%%%%%%%%%%%%%%%%%%%%%%%%%%%%%%%%%%%%%%%%%%%%%%%%%%%%%
%%%%%%%%             Text Start                 %%%%%%%%
\begin{document}
\chapter{数値解法} 
\section{}   
\section{鉛直離散化}
\markright{\arabic{chapter}.\arabic{section}  数値解法 } %  節の題名を書き込むこと

\section{水平離散化}

%\newpage
%\input{pedlosky20130912_appendix}
\end{document}

%%
%%%%%%%%              Text End                  %%%%%%%%
%%%%%%%%%%%%%%%%%%%%%%%%%%%%%%%%%%%%%%%%%%%%%%%%%%%%%%%%


%%%%%%%%%%%%%%%%%%%%%%%%%%%%%%%%%%%%%%%%%%%%%%%%%%%%%%%%%
\chapter{付録} %%%%%%%%%%%%%%%%%%
\section{静力学プリミティブ方程式系のエネルギー論}
ここでは, 静力学プリミティブ方程式系\eqref{eq:HydroBoussinesqPrimitive}に対するエネルギー保存則を記述する.
非線形の海水の状態方程式を用いる場合に適切な有効位置エネルギーの方程式の導出は,  
例えば\cite{vallis2006atmospheric,young2010dynamic,roquet2013dynamical,tailleux2013available}等で議論されているが,  
ここでは\cite{tailleux2013available}に従う. 

\subsection{エネルギー方程式}
最初に, 参照密度$\rho_r(z,t)$と相対的に定義される, 浮力
%%
\begin{align}
  b_R(\Theta,S,z,t) = - g\dfrac{\rho(\Theta,S,z) - \rho_r(z,t)}{\rho_0}
\end{align}
%%
を導入しておく.

運動エネルギーに対する方程式を導く.
\eqref{eq:HydroBoussinesqPrimitive_MomUV}と$\rho \Dvect{u}$の内積をとった結果と,
\eqref{eq:HydroBoussinesqPrimitive_MomW}と$\rho w$の積を加え,
さらに連続の式を使って整理すれば, 
%%
\begin{align}
    \DP{E_K}{t}
 =
  &-  \nabla_h \cdot \left[\left( E_K + g\eta + \dfrac{p}{\rho_0}  \right) \Dvect{u} - A_h \nabla_h E_K \right] \nonumber \\
  &-  \DP{}{z} \left[\left( E_K + \dfrac{p}{\rho_0} \right) w - A_v \DP{E_K}{z} \right]                         \nonumber \\
  &+  b_R w
   - \varepsilon
\label{eq:BoussinesqEnergetics_KE}
\end{align}
%%
を得る.
ここで, $E_K=\Dvect{u}^2/2$は(水平速度の)運動エネルギーである.
また, $\varepsilon$は粘性散逸を表す項であり, 今の系では$\varepsilon$は, 
%%
\begin{align}
 \varepsilon
 =   A_h \nabla_h \Dvect{u} : \nabla_h \Dvect{u}
   + A_v \left(\DP{w}{z}\right)^2
\end{align}
%%
と書かれる.

次に, 有効位置エネルギーに対する方程式を導く.
\cite{tailleux2013available}は, 浮力$b_R$に対する仕事として有効位置エネルギー密度を
%%
\begin{align}
 E_{APE} = - \int_{z_r}^z b_R(\Theta,S,z',t) \Dd{z'}
\label{eq:APEDensity_def}
\end{align}
%%
と定義した.
ここで, $z_r$は現場密度が参照密度と等しくなる高度であり, 
数学的には, 
%%
\begin{align}
  \rho(\Theta,S,z_r) = \rho_r(z_r,t)
\end{align}
%%
を満たす$z_r$である.
\eqref{eq:APEDensity_def}の物質微分をとり整理すれば,
$E_{APE}$の時間発展式
%%
\begin{align}
 \DP{E_{APE}}{t}
 =
 &- \nabla_h \cdot (E_{APE} \; \Dvect{u}) - \DP{}{z} (E_{APE} \; w) \\
 &- b_R w
 + \left[G_\Theta \DD{\Theta}{t} + G_S \DD{S}{t} + G_t\right]
\label{eq:BoussinesqEnergetics_APEDensity}
\end{align}
%%
を得る.
ただし, 定義により$b_R(\Theta,S,z_r,t)=0$であることを用いた.
また, $G_\Theta, G_S$は,
%%
\begin{align}
 G_\Theta &= - \int_{z_r}^z \DP{b_R}{\Theta}(\Theta,S,z',t) \Dd{z'}
           = \dfrac{g_0}{\rho_0} \int_{z_r}^z \DP{\rho}{\Theta}(\Theta,S,z') \Dd{z'}, \\
  G_S &= - \int_{z_r}^z \DP{b_R}{S}(\Theta,S,z',t) \Dd{z'}
       = \dfrac{g_0}{\rho_0} \int_{z_r}^z \DP{\rho}{S}(\Theta,S,z') \Dd{z'}, \\
\end{align}
であり, 熱や塩分のソース・シンクに対する熱力学的効率を定義する.
$G_t$は,
%%
\begin{align}
 G_t = - \int_{z_r}^z \DP{b_R}{t}(\Theta,S,z',t)\Dd{z'}
     = - \dfrac{g}{\rho_0} \int_{z_r}^z \DP{\rho_r}{t}(z',t) \Dd{z'}
\end{align}
%%
と書かれ,
参照密度分布が時間に依存しなければゼロである.
予期されたように, 浮力フラックス$b_R w$が\eqref{eq:BoussinesqEnergetics_KE}, \eqref{eq:BoussinesqEnergetics_APEDensity}の両方に
反符号として現れることに注意されたい. 

\eqref{eq:BoussinesqEnergetics_KE},\eqref{eq:BoussinesqEnergetics_APEDensity}を加えて整理すれば,
(非線形の海水の状態方程式を伴う)ブジネスクプリミティブ方程式系のエネルギー保存則
%%
\begin{align}
 \DP{}{t} \left(E_K + E_{APE}\right)
 =
 &- \nabla_h \cdot \left[
        \left( E_K + E_{APE} + \dfrac{p}{\rho_0} + g\eta \right) \Dvect{u}
      - A_h \nabla_h E_K
    \right]               \nonumber \\
 &- \DP{}{z} \left[
        \left(E_K + E_{APE} + \dfrac{p}{\rho_0} \right)w
      - A_v \DP{E_K}{z}
    \right]              \nonumber \\
 &+ \underbrace{ \left[
      G_\Theta \DD{\Theta}{t}
    + G_S \DD{S}{t}
    + G_t
    \right] }_{G_a}
  - \varepsilon
\end{align}
%%
を得る.
したがって, 非粘性かつ$G_a$がゼロである場合(熱や塩分のソース・シンクがなく, また参照密度が時間に依存しない場合)には,
$E_K,E_{APE}$の和は保存する.

\subsection{鉛直積分したエネルギー方程式}
以下では, 前節で求めたエネルギー方程式を鉛直積分し, さらに海面・海底の境界条件を考慮たし形式を導く.
ここで得られた方程式は, 南北エネルギー収支や全球エネルギー収支を議論する際に必要となる.
以下では, 簡潔に記述するために, rigid-lid 近似を適用した場合のみを扱うことにする. 

始めに運動エネルギーに対する方程式\eqref{eq:BoussinesqEnergetics_APEDensity}の鉛直積分を考えよう.
\eqref{eq:BoussinesqEnergetics_APEDensity}を$z=-H(\Dvect{x}_h)$から$z=0$まで積分すれば,
%%
\begin{align}
 \DP{\overline{E_K}^{(z)}}{t}
 =
 & - \overline{ \nabla_h \cdot \Dvect{F}_{h,E_K} }^{(z)}   \nonumber \\
 & - \left(F_{z,E_K}|_{z=0} - F_{z,E_K}|_{z=-H} \right)    \nonumber \\
 & + \overline{b_R w}^{(z)} - \overline{\varepsilon}^{(z)}
\label{eq:BoussinesqEnergetics_KE_VIntegral}
\end{align}
%%
を得る.
ここで, $f(\Dvect{x}_h,z)の$鉛直積分に関して, 
%%
\begin{align}
 \overline{f}^{(z)} = \int_{-H}^0 f \Dd{z}
\end{align}
%%
なる記号を導入した.
また, $(\Dvect{F}_{h,E_K},F_{z,E_K})$は運動エネルギー・圧力・粘性フラックスの和であり,
%%
\begin{align}
 \Dvect{F}_{h,E_K} &= \left( E_K + \dfrac{p_s + p}{\rho_0} \right) \Dvect{u} - A_h \nabla_h E_K, \\
 F_{z,E_K} &= (E_K + \dfrac{p}{\rho_0}) w - A_v \DP{E_K}{z}
\end{align}
%%
と書かれる.
今, 海面において$w=0$かつ風応力$\Dvect{\tau}_s$が課されるとし,
また海底において滑りなし条件が課されるとすれば, \eqref{eq:BoussinesqEnergetics_KE_VIntegral}は,
%%
\begin{align}
 \DP{\overline{E_K}^{(z)}}{t}
 =
 & - \overline{ \nabla_h \cdot \Dvect{F}_{h,E_K} }^{(z)} \nonumber \\
 & + \Dvect{u}|_{z=0} \cdot\Dvect{\tau}_s                \nonumber \\
 & + \overline{b_R w}^{(z)} - \overline{\varepsilon}^{(z)}
\label{eq:BoussinesqEnergetics_KE_VIntegral_applyBC}
\end{align}
%%
となる.

次に, 有効位置エネルギーに対する方程式\eqref{eq:BoussinesqEnergetics_APEDensity}の鉛直積分を考える.
今, 温位や塩分の保存則は,
%%
\begin{align}
  \DD{\Theta}{t} &= \nabla_h \cdot (K_h \nabla_h \Theta) + \DP{}{z}\left(K_z \DP{\Theta}{z} \right), \\
  \DD{S}{t} &= \nabla_h \cdot (K_h \nabla_h S) + \DP{}{z}\left(K_z \DP{S}{z} \right)
\end{align}
%%
の形式で記述されるとする. 
運動エネルギーのときと同様に\eqref{eq:BoussinesqEnergetics_APEDensity}の鉛直積分をとれば,
%%
\begin{align}
 \DP{\overline{E_{APE}}^{(z)}}{t} 
 =
 & - \overline{ \nabla_h \cdot \Dvect{F}_{h,E_{APE}} }^{(z)}                   \nonumber \\
 & - \left(F_{z,E_{APE}}|_{z=0} - F_{z,E_{APE}}|_{z=-H}  \right)                \nonumber \\
 & - \overline{b_R w}^{(z)}
   - \overline{d}^{(z)}
\label{eq:BoussinesqEnergetics_APEDensity_VIntegral} 
\end{align}
%%
を得る.
ただし, $(\Dvect{F}_{h,E_{APE}},F_{z,E_{APE}})$は, 有効位置エネルギーと拡散フラックスの和
%%
\begin{align}
  \Dvect{F}_{h,E_{APE}} &= E_{APE} \Dvect{u} - K_h (G_\Theta\nabla_h\Theta + G_S\nabla_h S), \\
  F_{z,E_{APE}} &= E_{APE} w - K_v (G_\Theta\DP{\Theta}{z} + G_S\DP{S}{z})
\end{align}
%%
である.
$d$は拡散過程による散逸項であり,
%%
\begin{align}
 d =
     &\left(K_h \nabla_h \Theta \cdot \nabla_h G_\Theta  + K_v \DP{\Theta}{z}\DP{G_\Theta}{z} \right) \\
   + &\left(K_h \nabla_h S \cdot \nabla_h G_S  + K_v \DP{S}{z}\DP{G_S}{z} \right)
\end{align}
%%
と書かれる.
今, 海面において熱フラックスと淡水フラックスが,
%%
\begin{align*}
 K_v \DP{\Theta}{z} = \dfrac{Q}{\rho_0 c_p}, \;\;
 K_v \DP{S}{z} = S_0 (E - P)
\end{align*}
%%
と与えられ, また海底において断熱条件が課されるならば(海面で$w=0$, 海底で滑りなし条件を同時に課せば),
\eqref{eq:BoussinesqEnergetics_APEDensity_VIntegral}は, 
%%
\begin{align}
 \DP{\overline{E_{APE}}^{(z)}}{t} 
 =
 & - \overline{ \nabla_h \cdot \Dvect{F}_{h,E_{APE}} }^{(z)}                   \nonumber \\
 & + G_\Theta|_{z=0} \cdot \dfrac{Q}{\rho_0 c_p} + G_S|_{z=0} \cdot S_0(E - P)                     \nonumber \\
 & - \overline{b_R w}^{(z)}
   - \overline{d}^{(z)}
\label{eq:BoussinesqEnergetics_APEDensity_VIntegral_applyBC} 
\end{align}
%%
となる. 




%%%%%%%%%%%%%%%%%%%%%%%%%%%%%%%%%%%%%%%%%%%%%%%%%%%%%%%%%
%% 参考文献
\bibliographystyle{plainnat}
\bibliography{./../../model/sogcm/misc/reflist/Dennou-OGCM_reflist}

\end{document}