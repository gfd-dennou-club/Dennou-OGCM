%
% Dennou-SOGCM
%
%   2013/11/12  海洋モデル 定式化
%               河合 佑太
%
% style  Setting             %%%%%%%%
% フォント: 12point (最大), 片面印刷
\documentclass[a4j,12pt,openbib,oneside]{jreport}

%%%%%%%%%%%%%%%%%%%%%%%%%%%%%%%%%%%%%%%%%%%%%%%%%%%%%%%%
%%%%%%%%             Package Include            %%%%%%%%

\usepackage{ascmac}
\usepackage{tabularx}
\usepackage{graphicx}
\usepackage{amssymb}
\usepackage{amsmath}
\usepackage{mathrsfs}
\usepackage{mathtools}
\usepackage{Dennou6}		% 電脳スタイル ver 6
%%%%%%%%%%%%%%%%%%%%%%%%%%%%%%%%%%%%%%%%%%%%%%%%%%%%%%%%
%%%%%%%%            PageStyle Setting           %%%%%%%%
\pagestyle{DAmyheadings}

%%%%%%%%%%%%%%%%%%%%%%%%%%%%%%%%%%%%%%%%%%%%%%%%%%%%%%%%
%%%%%%%%        Title and Auther Setting        %%%%%%%%
%%
%%  [ ] はヘッダに書き出される.
%%  { } は表題 (\maketitle) に書き出される.

\Dtitle{Dennou-SOGCM}   % 変更不可
\Dauthor{河合佑太}        % 担当者の名前
\Ddate{2013/11/14}        % 変更日時 (毎回変更すること)
\Dfile{formulatiom.tex}

%%%%%%%%%%%%%%%%%%%%%%%%%%%%%%%%%%%%%%%%%%%%%%%%%%%%%%%%
%%%%%%%%   Set Counter (chapter, section etc. ) %%%%%%%%
\setcounter{chapter}{1}    % 章番号
\setcounter{section}{1}    % 節番号
\setcounter{subsection}{1}    % 節番号
\setcounter{equation}{1}   % 式番号
\setcounter{page}{1}     % 必ず開始ページは明記する
\setcounter{figure}{0}     % 図番号
\setcounter{table}{0}      % 表番号
%\setcounter{footnote}{0}


%%%%%%%%%%%%%%%%%%%%%%%%%%%%%%%%%%%%%%%%%%%%%%%%%%%%%%%%
%%%%%%%%        Counter Output Format           %%%%%%%%
\def\thechapter{\arabic{chapter}}
\def\thesection{\arabic{chapter}.\arabic{section}}
\def\thesubsection{\arabic{chapter}.\arabic{section}.\alph{subsection}}
\def\theequation{\arabic{chapter}.\arabic{section}.\arabic{equation}}
\def\thepage{\arabic{page}}
\def\thefigure{\arabic{chapter}.\arabic{section}.\arabic{figure}}
\def\thetable{\arabic{chapter}.\arabic{section}.\arabic{table}}
\def\thefootnote{*\arabic{footnote}}

%%%%%%%%%%%%%%%%%%%%%%%%%%%%%%%%%%%%%%%%%%%%%%%%%%%%%%%%
%%%%%%%%        Dennou-Style Definition         %%%%%%%%

%% 改段落時の空行設定
\Dparskip      % 改段落時に一行空行を入れる
%\Dnoparskip    % 改段落時に一行空行を入れない

%% 改段落時のインデント設定
\Dparindent    % 改段落時にインデントする
%\Dnoparindent  % 改段落時にインデントしない

%%%%%%%%%%%%%%%%%%%%%%%%%%%%%%%%%%%%%%%%%%%%%%%%%%%%%%%%%%
%% Macro defined by author
\def\univec#1{ \hat{ \Dvect{\rm #1}} }

%%%%%%%%%%%%%%%%%%%%%%%%%%%%%%%%%%%%%%%%%%%%%%%%%%%%%%%
%%%%%%%             Text Start                 %%%%%%%%
\begin{document}
\chapter{定式化}    % 章の始めからの場合はこのコマンドを使用する
\section{基礎方程式と境界条件} % 節の始めからの場合はこのコマンドを使用する
\markright{\arabic{chapter}.\arabic{section}  定式化 } %  節の題名

\subsection{基礎方程式}
基礎方程式は, 静水圧近似とブジネスク近似を施したプリミティブ方程式である. 
本節では, まずデカルト座標系における基礎方程式を記述することにする. 
%%
\begin{align}
\shortintertext{水平方向の運動方程式}
  \DP{\Dvect{v}}{t} + \Dvect{v} \cdot \nabla_H \Dvect{v} + w\DP{\Dvect{v}}{z} + f \univec{k} \times \Dvect{v} 
 =& - \nabla_H \Phi + \mathscr{F}_{\Dvect{v}} + \mathscr{D}_{\Dvect{v}}, \label{hmotion_Eq} \\
%%
\shortintertext{温位の式}
   \DP{\Theta}{t} + \Dvect{v}\cdot\nabla_H\Theta + w\DP{\Theta}{z} 
 =& \mathscr{F}_\Theta + \mathscr{D}_\Theta, \label{ptemp_Eq} \\
%%
\shortintertext{塩分の式}
   \DP{S}{t} + \Dvect{v}\cdot\nabla_H S + w\DP{S}{z} 
 =& \mathscr{F}_S + \mathscr{D}_S, \label{salt_Eq} \\
%%
\shortintertext{状態方程式}
  \rho =& \rho(\Theta, S, z), \label{state_Eq} \\
%%
\shortintertext{静水圧平衡の式}
  \DP{\Phi}{z} =& - \dfrac{\rho'}{\rho_0} g, \label{hydrostatic_Eq} \\
%%
\shortintertext{連続の式}
  \nabla_H \cdot \Dvect{v} + \DP{w}{z} =& 0 \label{continious_Eq}. 
\end{align}
%%
ここで, $\Dvect{v}$は水平速度ベクトル, $w$は鉛直速度, 
$\rho_0$は参照密度, $\rho$は現場密度(density in situ), $\rho'(=\rho-\rho_0)$は密度の変動成分, 
$\Theta$は温位, 
$S$は塩分, $p$は圧力, $\Phi(=p/\rho_0)$は動圧である. 
また, $\{ \mathscr{F}_{\Dvect{v}}, \mathscr{F}_\theta, \mathscr{F}_S \}$は強制項, 
$\{ \mathscr{D}_{\Dvect{v}}, \mathscr{D}_\theta, \mathscr{D}_S \}$は拡散項である. 
$g$は重力加速度, $\nabla_H$は水平微分演算子, $\univec{k}$は鉛直単位ベクトルである. 

ブジネスク近似を用いるときには, 運動方程式における密度の変化は, 
鉛直方向の運動方程式における浮力の寄与を除いて無視される. 
さらに静水圧平衡近似を適用する場合には, 
(\ref{hydrostatic_Eq})で示されるように, 鉛直圧力勾配は浮力項とバランスすることもまた仮定する. 
最後に, 連続の式として非圧縮流体に対する連続の式(\ref{continious_Eq})を用いる. 

\subsection{鉛直方向の境界条件}
次に, 系の上端である海面の自由表面($z=\eta(x,y,t)$)と, 下端である海底($z=-H(x,y)$)において,  
方程式(\ref{hmotion_Eq})-(\ref{salt_Eq})に課される境界条件を示す. 
ここで, $\eta=\eta(x,y,t)$は自由表面の水位, $H=H(x,y)$は運動のないときの深さである. 

剛体の海底($z=-H$)において, 底面を横切る流れが存在しないために, 
%%
\begin{equation*}
  w= - \Dvect{v}\cdot\nabla H
\end{equation*}
%%
を課す. 
また, 自由表面($z=\eta$)では, 運動学的境界条件
%%
\begin{equation}
 w = \DD{\eta}{t} - (P-E)
\end{equation}
%%
を課す. 
ここで, $P-E$は降水量から蒸発量を差し引いた淡水フラックスである. 
非圧縮流体に対する連続の式(\ref{continious_Eq})を流体層の厚さに渡って積分し,  
上端・下端において上の鉛直速度の境界条件を用いれば, 
自由表面に対する方程式
%%
\begin{equation}
 \DP{\eta}{t} + \nabla \cdot \left[ (\eta+H) <\Dvect{v}> \right] = P-E
\end{equation}
%%
を得る. 
ただし, $<\Dvect{v}>$は深さ平均した水平速度ベクトルであり, 
その深さ平均とは
%%
\begin{equation}
 <(\;\;)> \equiv \int_{z=-H}^{z=\eta} (\;\;) dz
\end{equation}
%%
と定義される. 

次に上端下端における力学的境界条件は, ???.

最後に, 上端下端における熱や塩分に対する境界条件は, ???.

\subsubsection*{剛体表面(Rigid-lid)近似}

\subsection{水平方向の境界条件}

\section{鉛直座標変換}
海洋の数値モデリングでは, 自由表面や海底地形などの効果を取り扱うために,
さまざまな鉛直座標が用いられる. 
本モデルでもまたこれらの効果を取り入れるために, 高度座標系($z$座標系)を鉛直座標変換する. 
その準備としてまず基礎方程式の鉛直座標を一般座標に変換することから始める. 

\subsection{一般鉛直座標変換}
基礎方程式を高度座標系$z$から一般鉛直座標$\xi$に変換するために,  
%%
\begin{equation}
 z= z(x,y,\xi,t), \;\;\; \xi = \xi(x,y,z,t)  
\end{equation}
%%
なる座標変換を行う. 

いくつかの座標変換の規則を用いることにより, 
$\xi$系における基礎方程式
%%
\begin{align}
  \DP{\Dvect{v}}{t} + \Dvect{v} \cdot \nabla_\xi \Dvect{v} + \dot{\xi} \DP{\Dvect{v}}{\xi} + f \univec{k} \times \Dvect{v} 
 =& - \nabla_\xi \Phi + \dfrac{\rho'}{\rho_0} \nabla_\xi (gz) 
  + \mathscr{F}_{\Dvect{v}} + \mathscr{D}_{\Dvect{v}}, \label{hmotion_Eq_genCoord} \\
%%
   \DP{\Theta}{t} + \Dvect{v}\cdot\nabla_\xi\Theta + \dot{\xi} \DP{\Theta}{\xi} 
 =& \mathscr{F}_\Theta + \mathscr{D}_\Theta, \label{ptemp_Eq_genCoord} \\
%%
   \DP{S}{t} + \Dvect{v}\cdot\nabla_\xi S + \dot{\xi}\DP{S}{\xi} 
 =& \mathscr{F}_S + \mathscr{D}_S, \label{salt_Eq_genCoord} \\
%%
  \DP{\Phi}{\xi} =& - \dfrac{\rho'}{\rho_0} \DP{(gz)}{\xi}, \label{hydrostatic_Eq_genCoord} \\
%%
  \DP{z_\xi}{t} + \nabla_\xi \cdot (z_\xi \Dvect{v}) + \DP{}{z}(z_\xi \dot{\xi}) =& 0 \label{continious_Eq_genCoord}. 
\end{align}
%%
が得られる. 
ここで, 
%%
\begin{align*}
 z_\xi = \DP{z}{\xi}, \;\;\;\;\;
 \dot{\xi} = \DD{\xi}{t}
\end{align*}
%%
と定義する. 
また, $\nabla_\xi$は等$\xi$面における水平微分演算子である. 

\subsection{$\sigma$座標系}
本モデルでは, 自由表面と地形の効果を取り入れるために, 
次に定義される鉛直座標系($\sigma$系)を用いる. 
また, 鉛直離散化には chevyshev 多項式によるスペクトル変換法を用いるので, 
その際に必要な鉛直座標のリスケールもまた行う. 

今, $\sigma$系と$z$系の間の座標変換は, 
%%
\begin{equation}
 \sigma(x,y,z,t) = \dfrac{z - \eta(x,y,t)}{\eta(x,y,t) + H(x,y)} 
\end{equation}
%%
と定義する. 
このとき, モデル上端は$\sigma=0$, 下端は$\sigma=-1$となる. 
また, $\sigma$系の基礎方程式は, 
%%
\begin{align}
  \DP{\Dvect{v}}{t} + \Dvect{v} \cdot \nabla_\sigma \Dvect{v} + \dot{\sigma} \DP{\Dvect{v}}{\xi} + f \univec{k} \times \Dvect{v} 
 =& - \nabla_\sigma \Phi + \dfrac{\rho'}{\rho_0} \nabla_\sigma (gz) 
  + \mathscr{F}_{\Dvect{v}} + \mathscr{D}_{\Dvect{v}}, \label{hmotion_Eq_SigCoord} \\
%%
   \DP{\Theta}{t} + \Dvect{v}\cdot\nabla_\sigma\Theta + \dot{\sigma} \DP{\Theta}{\sigma} 
 =& \mathscr{F}_\Theta + \mathscr{D}_\Theta, \label{ptemp_Eq_SigCoord} \\
%%
   \DP{S}{t} + \Dvect{v}\cdot\nabla_\sigma S + \dot{\sigma}\DP{S}{\sigma} 
 =& \mathscr{F}_S + \mathscr{D}_S, \label{salt_Eq_SigCoord} \\
%%
  \DP{\Phi}{\sigma} =& - \dfrac{\rho'}{\rho_0} gD, \label{hydrostatic_Eq_SigCoord} \\
%%
  \DP{\eta}{t} + \nabla_\sigma \cdot (D \Dvect{v}) + \DP{}{\sigma}(D \dot{\sigma}) =& 0 \label{continious_Eq_SigCoord}. 
\end{align}
%%
と書かれる. 
ここで, $D(=\eta+H)$は自由表面の水位も含めた流体層の厚さである. 

さらに, $z$系における鉛直方向の境界条件は, $\sigma$系において次のように
書きかえられる. 
海底および海面における運動学的境界条件は, 
それぞれ
%%
\begin{equation}
 \dot{\sigma} = 0 \;\;\;\;\;\;\;\;\;\; {\rm at} \;\; \sigma=-1,
\end{equation}
\begin{equation}
 \dot{\sigma} = (P-E)/D \;\;\;\;\;     {\rm at} \;\; \sigma=0
\end{equation}
となる. 
一方, 力学的境界条件は, ???.
また, 熱と塩分に対する境界条件は, ???.

%\newpage
%\input{pedlosky20130912_appendix}
\end{document}

%%
%%%%%%%%              Text End                  %%%%%%%%
%%%%%%%%%%%%%%%%%%%%%%%%%%%%%%%%%%%%%%%%%%%%%%%%%%%%%%%%
