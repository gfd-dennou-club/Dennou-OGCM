%
% Dennou-SOGCM
%
%   2015/06/15  海洋モデル 定式化
%               河合 佑太
%
% style  Setting             %%%%%%%%
% フォント: 12point (最大), 片面印刷
\documentclass[a4j,12pt,openbib,oneside]{jreport}

%%%%%%%%%%%%%%%%%%%%%%%%%%%%%%%%%%%%%%%%%%%%%%%%%%%%%%%%
%%%%%%%%             Package Include            %%%%%%%%

\usepackage{ascmac}
\usepackage{tabularx}
\usepackage{graphicx}
\usepackage{amssymb}
\usepackage{amsmath}
\usepackage{mathrsfs}
\usepackage{mathtools}
\usepackage{Dennou6}		% 電脳スタイル ver 6
\usepackage[round]{natbib}


%%%%%%%%%%%%%%%%%%%%%%%%%%%%%%%%%%%%%%%%%%%%%%%%%%%%%%%%
%%%%%%%%            PageStyle Setting           %%%%%%%%
\pagestyle{DAmyheadings}

%%%%%%%%%%%%%%%%%%%%%%%%%%%%%%%%%%%%%%%%%%%%%%%%%%%%%%%%
%%%%%%%%        Title and Auther Setting        %%%%%%%%
%%
%%  [ ] はヘッダに書き出される.
%%  { } は表題 (\maketitle) に書き出される.

\Dtitle{Dennou-SOGCM}   % 変更不可
\Dauthor{河合佑太}        % 担当者の名前
\Ddate{2015/06/16}        % 変更日時 (毎回変更すること)
\Dfile{formulatiom.tex}

%%%%%%%%%%%%%%%%%%%%%%%%%%%%%%%%%%%%%%%%%%%%%%%%%%%%%%%%
%%%%%%%%   Set Counter (chapter, section etc. ) %%%%%%%%
\setcounter{chapter}{0}    % 章番号
\setcounter{section}{1}    % 節番号
\setcounter{subsection}{1}    % 節番号
\setcounter{equation}{1}   % 式番号
\setcounter{page}{1}     % 必ず開始ページは明記する
\setcounter{figure}{0}     % 図番号
\setcounter{table}{0}      % 表番号
%\setcounter{footnote}{0}


%%%%%%%%%%%%%%%%%%%%%%%%%%%%%%%%%%%%%%%%%%%%%%%%%%%%%%%%
%%%%%%%%        Counter Output Format           %%%%%%%%
\def\thechapter{\arabic{chapter}}
\def\thesection{\arabic{chapter}.\arabic{section}}
\def\thesubsection{\arabic{chapter}.\arabic{section}.\alph{subsection}}
\def\theequation{\arabic{chapter}.\arabic{section}.\arabic{equation}}
\def\thepage{\arabic{page}}
\def\thefigure{\arabic{chapter}.\arabic{section}.\arabic{figure}}
\def\thetable{\arabic{chapter}.\arabic{section}.\arabic{table}}
\def\thefootnote{*\arabic{footnote}}

%%%%%%%%%%%%%%%%%%%%%%%%%%%%%%%%%%%%%%%%%%%%%%%%%%%%%%%%
%%%%%%%%        Dennou-Style Definition         %%%%%%%%

%% 改段落時の空行設定
\Dparskip      % 改段落時に一行空行を入れる
%\Dnoparskip    % 改段落時に一行空行を入れない

%% 改段落時のインデント設定
\Dparindent    % 改段落時にインデントする
%\Dnoparindent  % 改段落時にインデントしない

%%%%%%%%%%%%%%%%%%%%%%%%%%%%%%%%%%%%%%%%%%%%%%%%%%%%%%%%%%
%% Macro defined by author
\def\univec#1{ \hat{ \Dvect{\rm #1}} }
\newcommand{\Deg}{\(^{\circ} \) \kern-.4em}

%%%%%%%%%%%%%%%%%%%%%%%%%%%%%%%%%%%%%%%%%%%%%%%%%%%%%%%
%%%%%%%             Text Start                 %%%%%%%%
\begin{document}

\chapter{海氷熱力学モデルの定式化}

\section{系の設定}
%%
%%
\begin{figure}[b]]
 \includegraphics[width=10cm]{seaice_therm_model_config.eps}
 \caption{系の設定}
 \label{fig:configuration}
\end{figure} 
%%
多くの海氷熱力学モデル\citep{maykut1971some,semtner1976model,winton2000reformulated}は,
海氷を図\ref{fig:configuration}のような雪層と氷層から成る鉛直一次元のカラムとして考える. 
また, 海氷の上端では大気, 下端では海洋に接しており, 
そこで海氷は大気や海洋と熱のやりとりをする.

雪層や氷層内の鉛直温度分布は, 各層において熱伝導方程式を解くによって決定される.
この熱伝導方程式に課す境界条件は, 大気からの熱フラックスや海洋から熱フラックス
である. 
%%
\section{支配方程式系}
以下に, 一般的な海氷熱力学モデルにおける支配方程式系を示す.
これらの方程式は, 雪層・氷層の温度(それぞれを$T_{\rm snow}=T_{\rm snow}(z,t)$, $T_{\rm ice}=T_{\rm ice}(z,t)$とおく)
の鉛直分布や雪層・氷層の厚さ(それぞれを$h_{\rm snow}=h_{\rm snow}(t)$, $h_{\rm ice}=h_{\rm ice}(t)$)を決定する. 
なお, 以下では, 特に断らない限り温度の単位として摂氏($^\circ$ C)を用いる. 
%%
%%%%%%%%%%%%%%%%%%%%%%%%%%%%%%%%%%%%%%%%%%%%%%%%%%
\subsection*{雪層に対する熱伝導方程式}
%%
\begin{equation}
 \rho_{\rm snow} \; c_{\rm snow} \DP{T_{\rm snow}}{t}
  = k_{\rm snow} \DP[2]{T_{\rm snow}}{z}
    + \kappa_s I(0) \exp{[\kappa_{\rm snow} z]}
\end{equation}
%%
ここで, $\rho_{\rm snow}$は雪の密度, $c_{\rm snow}=c_{\rm snow}$は雪の比熱,
$k_{\rm snow}$は雪の熱伝導係数, $\kappa_{\rm snow}$は雪の透過係数, 
$I(0)$は雪層表面を貫く短波放射である.

\subsection*{雪層に対する質量保存則}
%%
\begin{equation}
  \rho_{\rm snow} \DD{h_{\rm snow}}{t} = M_{\rm snow} + F_{\rm snowfall}
\end{equation}
%%
ここで, $M_{\rm snow}$, $F_{\rm snowfall}$はそれぞれ, 単位時間あたりの雪層の融解量と積雪量である. 

\subsection*{氷層に対する熱伝導方程式}
%%
\begin{equation}
 \rho_{\rm ice} \; c_{\rm ice} \DP{T_{\rm ice}}{t}
  = k_{\rm ice} \DP[2]{T_{\rm ice}}{z}
    + \kappa_s I(-h_s) \exp{[\kappa_{\rm ice} (z + h_s)]}
\end{equation}
%%
ここで, $\rho_{\rm ice}$は海氷の密度, $c_{\rm ice}=c_{\rm ice}$は海氷の比熱,
$k_{\rm ice}$は海氷の熱伝導係数, $\kappa_{\rm ice}$は海氷の透過係数である.
また, $I(-h_s)$は氷層表面を貫く短波放射である.

\subsection*{氷層に対する質量保存則}
%%
\begin{equation}
  \rho_{\rm ice} \DD{h_{\rm ice}}{t} = M_{\rm ice}
\end{equation}
%%
ここで, $M_{\rm ice}$は単位時間あたりの氷層の融解量である.

\section{境界条件}

\subsubsection*{海氷表面($z=0$)における境界条件(表面熱収支)}
雪層の存在する場合, 海氷表面において次の熱収支バランスを満たすとする. 
%%
\begin{equation}
 \begin{split}
    (1 - \alpha_s) F_r - I(0) + F_L &- \sigma [T_{\rm snow}(0) + 273.15]^4 + F_s + F_e
    + k_{\rm snow} \left(\DP{T_{\rm snow}}{z}\right)_{z=0} \\
  &= \begin{cases} 
      0 \;\;\; (T_{\rm snow} < 0 \; {\rm ^\circ C}) \\
      H_{\rm snow} \;\;\; (T_{\rm snow} = 0 \; {\rm ^\circ C} )
     \end{cases}
 \end{split}
\end{equation}
%%
ここで, $F_r, F_L, F_s, F_e$はそれぞれ海氷表面における入射短波放射フラックス, 入射長波放射フラックス,
顕熱フラックス, 潜熱フラックスである(これらのフラックスの符号は, 鉛直上向きを正にとる).
また, $\alpha_s$は雪のアルベド, $\sigma$はステファン・ボルツマン定数である. 
$H_{\rm snow}$は, 雪層表面の融解に伴う単位時間あたりの融解熱を表す. 

一方, 雪層が存在せず氷層表面が大気にさらされている場合($h_s=0$の場合)は,
海氷表面において次の熱収支バランスを満たすとする. 
%%
%%
\begin{equation}
 \begin{split}
    (1 - \alpha_i) F_r - I(0) &- \sigma [T_{\rm ice}(0) + 273.15]^4 + F_s + F_e
    + k_{\rm snow} \left(\DP{T_{\rm ice}}{z}\right)_{z=0} \\
  &= \begin{cases} 
      0 \;\;\; (T_{\rm ice} < T_{f,{\rm si}}) \\
      H_{s,{\rm ice}} \;\;\; (T_{\rm ice} = T_{f,{\rm si}} )
     \end{cases}
 \end{split}
\end{equation}
%%
ここで, $\alpha_s$は雪のアルベド, $T_{f,{\rm si}}$は海氷の氷点, 
$H_{s,{\rm ice}}$は氷層表面の融解に伴う単位時間あたりの融解熱を表す. 

\subsubsection*{雪層・氷層間の境界条件}
雪層と氷層の間で, 熱伝導フラックスは連続であることを課す.
すなわち, 
%%
\begin{equation}
   \kappa_{\rm snow} \left(\DP{T_{\rm snow}}{z} \right)_{z=-h_s}
=  \kappa_{\rm ice} \left(\DP{T_{\rm ice}}{z} \right)_{z=-h_s}. 
\end{equation}

\subsubsection*{海氷下端の境界条件}
%%
海氷下端($z=-(h_s + h_i)\equiv z_b$)の氷の温度は, 海水の氷点$T_{f,{\rm sw}}$に固定する.
すなわち, 
%%
\begin{equation}
  T_{\rm ice} (z_b)  = T_{f,{\rm sw}}.
\end{equation}

\subsubsection*{氷層・海洋間の境界条件}
%%
海氷下端では, 
\begin{equation}
  F_b + k_{\rm ice} \left(\DP{T_{\rm ice}}{z} \right)_{z=z_b} = H_{b,{\rm ice}}
\end{equation}
%%
の熱収支バランスを満たすとする.
ここで, $F_b$は海洋熱フラックス, $H_{b,{\rm ice}}$は海氷下端の海氷生成・融解に伴う融解熱である.

\section{海氷の比熱}
海氷の比熱は, 次の近似式\citep{untersteiner1961mass,ono1967specific}を用いて求める%
\footnote{
海氷の比熱の近似式の導出は, 付録 A を参照されたい.
}. 
%%
\begin{equation}
  c_{\rm ice}(T,S) = c_o + \dfrac{L \mu S}{T^2}. 
\end{equation}
%%
ここで, $c_o$は純粋な氷の比熱, $S$は塩分量(単位はパーミル)である. 
また, $\mu (= -T_{f,{\rm ice}}/S)$は経験的な定数である(詳細は付録 A を参照). 
%%
\chapter*{付録 A: 海氷の比熱の式の導出}
ここでは, 海氷熱力学モデルで用いられる海氷の比熱の近似式
%%
\begin{equation*}
 c_{\rm si} = c_o + \dfrac{L \mu S}{T^2}
  \tag{A.1}
\end{equation*}
%%
の導出を\citep{ono1967specific}に従って行う.
ただし, $L$は 0$\circ$C における純氷の融解の潜熱,
$\mu$は経験的な定数(以下の導出を参照), $S$は海氷の塩分量である. 
本文と同様に以下でも, 温度の単位として摂氏を用いることに注意されたい. 

今, 海氷は純氷と濃縮された塩類溶液(ブライン)から成るとし, 気泡などの存在無視すれば, 
海氷 1 g の温度を$\Delta T$ $^\circ$C 上昇させるための熱量$C_{si}\Delta T$は,
%%
\begin{equation*}
 C_{si} \Delta T = C_i m_i \Delta T + C_b m_b \Delta T - \lambda_T \DD{m_i}{T} \Delta T
 \tag{A.2}
\end{equation*}
%%
と書ける.
ここで, $m_i, m_b$はそれぞれ海氷 1 g 中の純氷およびブラインの質量である.
また, $C_i$は純氷の比熱, $C_b$はブラインの比熱である.
$\lambda_T$は$T$ $^\circ$C における氷の融解の潜熱であり,
%%
\begin{equation*}
 \lambda_T = \lambda_0 + (C_w - C_i)T
 \tag{A.3}
\end{equation*}
%%
表される.
すなわち, $\lambda_T$は, $T$$^\circ$C の純氷 1 g をまず 0$^\circ$C まで暖め,
0 $^\circ$C で溶かし, 溶けた水の温度をまた $T$$^\circ$C まで下げるのに必要な熱量に等しい.
比熱は 1 g の物質の温度を 1$^\circ$C 上昇させるのに必要な熱量で定義されるので,
(A.2) から海水の比熱は,
%%
\begin{equation*}
 C_{si} = C_i m_i + C_b m_b - \lambda_T \DD{m_i}{T}
 \tag{A.4}
\end{equation*}
%%
と書ける.

海氷 1 g 中のブライン$m_b$ gは, $m_w$ g の水に$m_s$ g の塩類%
\footnote{
$m_s$は海氷中の塩類の質量分率であり, 塩分量$S$(単位はパーミル)とは,
%%
$$
 S = m_s \times 1000 
$$
%%
の関係がある.  
}
が溶けてできているとする.
このとき,
%
\begin{equation*}
 m_i + m_b = m_i + m_w + m_s = 1
 \tag{A.5}
\end{equation*}
%%
である.
比熱の中に現れるブライン中の水の質量$m_w$を$m_s$を使って書くために,
塩分の質量比
%%
\begin{equation*}
 \gamma = \dfrac{m_s}{m_w}
 \tag{A.6}
\end{equation*}
%%
を導入すれば, $m_w$は,
%%
\begin{equation*}
 m_w = \dfrac{m_s}{\gamma}
 \tag{A.7}  
\end{equation*}
%%
と書ける.
同様にして, 純粋の質量$m_i$は,
%%
\begin{equation*}
 m_i = 1 - m_s - \dfrac{m_s}{\gamma}
 \tag{A.8}  
\end{equation*}
%%
と書ける.

ブラインの比熱$C_b$は, 塩分量が増加すると小さくなり,
温度が下がるとわずかに大きくなる. したがって, ブラインの比熱を
%%
\begin{equation*}
 C_b = C_w  - \beta(S,T)
 \tag{A.9}  
\end{equation*}
%%
の形式で書くことができる.

(A.4) に (A.5),(A.7),(A.8),(A.9) を代入すれば,
%%
\begin{equation*}
 C_{\rm si} =
  C_i - \lambda_0 \DD{m_i}{T}
  + (C_w - C_i)\left(\dfrac{m_s}{\gamma} - T\DD{m_i}{T} \right)
  + (C_w - C_i - \beta)m_s - \beta\dfrac{m_s}{\gamma}
  \tag{A.10}
\end{equation*}
%%
を得る.
さらに,
$m_s$が$T$に依存しないことに注意すれば, 純氷の変化量は,
%%
$$
 \DD{m_i}{T} = - \DD{m_w}{T} = - \DD{}{T}\left(\dfrac{m_s}{\gamma}\right)
 = \dfrac{m_s}{\gamma^2}\DD{\gamma}{T}
 $$
 と$m_s$と$\gamma$を用いて書き表されるので, (A.10) は,
 %%
 \begin{equation*}
 C_{\rm si} =
  C_i - \lambda_0 \dfrac{m_s}{\gamma^2}\DD{\gamma}{T}
  + (C_w - C_i) \dfrac{m_s}{\gamma} \left(1 - \dfrac{T}{\gamma}\DD{\gamma}{T} \right)
  + (C_w - C_i - \beta)m_s - \beta\dfrac{m_s}{\gamma}
  \tag{A.10}  
 \end{equation*}
 %%
 となる.

 温度 $T ^\circ$C を結氷温度とするような平衡濃度にあるブラインの質量比$\gamma$と
 $T$には,
 %%
 \begin{equation*}
  T = - \alpha \gamma
  \tag{A.11}
 \end{equation*}
 %%
 のような線形の関係が成り立つことが知られている.
 -8.2 $^\circ$C 以上の温度範囲を考えるとき, 
 比例定数$\alpha$は 54.11 に取られる.
 (A.11)を(A.10)に代入すれば,
 %%
 \begin{equation*}
 C_{\rm si} =
  C_i + \lambda_0 \alpha \dfrac{m_s}{T^2}
  + (C_w - C_i - \beta)m_s + \alpha \beta\dfrac{m_s}{T}
  \tag{A.10}  
 \end{equation*}
 %%
 を得る.
 今考える温度範囲では, 右辺 3,4 項目の寄与は 2 項目に比べて小さい\citep{ono1967specific}ので
 無視すれば, 海氷の比熱は,
 %%
 \begin{equation*}
 C_{\rm si} =
  C_i + \lambda_0 \mu \dfrac{S}{T^2}
  \tag{A.10}  
 \end{equation*}
 %%
 と書ける.
 ただし, $m_s$を塩分量$S$で書き直し, $\mu = \alpha \times 1000$と置き直した. 

%% 参考文献
\bibliographystyle{plainnat}
\bibliography{Dennou-OGCM_reflist}

%%%%%%%%%%%%%%%%%%%%%%%%%%%%%%%%

\end{document}