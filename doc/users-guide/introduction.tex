%!TEX root = ./dcpom_description.tex
\section{この文書について}
地球流体電脳倶楽部では, 海洋大循環を考慮した惑星気候の探索を目指して, 
海洋大循環モデルと海氷モデルを開発しており, 
Dennou-Club Planetary Ocean Model (DCPOM) として公開している%
\footnote{
\url{https://www.gfd-dennou.org/arch/index.htm}からソースコード等を取得できる. 
}. 
本書は, DCPOM に含まれる海洋モデルと海氷モデルの利用者に向けた取り扱い説明書であり, 
DCPOM version \version \; に対応した説明を行う. 
モデルの支配方程式系やその離散化手法の詳細については, 
\verb|doc/description/dcpom_description.pdf|を参照されたい.  

本書の構成は次の通りである. 
第\ref{intro}章において, DCPOM の概要を示す. 
第\ref{install}章において, インストール方法を説明する. 
第\ref{tutorial}章において, DCPOM に含まれる海洋海氷モデルの基本的な使い方を, 
理想実験のチュートリアルを通して説明する. 
モデルの設定の変更方法などは, 第\ref{setting_detail}章で記述する. 


\section{DCPOM の概要}

\subsection*{dogcm}
本書で主に対象とするのは, \verb|${TOP_DIR}/model/dogcm|ディレクトリ中にある, 海洋大循環モデルと海氷モデルのコードである. 
この海洋大循環モデルで表現される過程や, そこで用いられる数値解法の概要は以下である。
\begin{itemize}
  \item 力学コア: 
  \begin{itemize}
    \item 浅い流体近似・静水圧近似をしたブジネスク方程式系
    \item 海水の状態方程式(温位, 塩分, 深さに依存)
    \begin{itemize}
      \item 線形関数 
      \item 二次関数 \citep{vallis2006AOFD}
      \item \cite{jackett1995minimal}に記述されている状態方程式
    \end{itemize}  
  \end{itemize}
  \item サブグリッド・スケールの物理過程のパラメタリゼーション
    \begin{itemize}
      \item メソスケールの渦による混合
      \begin{itemize}  
        \item 等密度面に沿った拡散スキーム \citep{redi1982oceanic} 
        \item skew flux に基づく GM スキーム \citep{gent1990isopycnal,griffies1998gent}
      \end{itemize} 
      \item 対流による鉛直混合
      \begin{itemize}  
        \item 対流調節スキーム \citep{marotzke1991influence,rahmstorf1993fast}
      \end{itemize}   
    \end{itemize} 
  \item 数値解法
  \begin{itemize}
    \item 水平離散化: 球面調和関数展開に基づくスペクトル法
    \item 鉛直離散化: 有限体積法
    \item 時間離散化
    \begin{itemize}
      \item 順圧モードと傾圧モードの分離 \citep{shchepetkin2005regional}
      \item 移流項: LF-AM3 スキーム \citep{shchepetkin2005regional}
  	  \item 鉛直拡散項, 伝播速度の速い波と関係した項: リープフロッグ法
  	  \item 水平拡散項: 前進オイラー法または後退オイラー法
    \end{itemize}                  
  \end{itemize}
\end{itemize}
また, 海氷モデルで表現される過程や, そこで用いられる数値解法の概要は以下である。
\begin{itemize}
  \item 熱力学過程: \citep{winton2000reformulated}に基づく三層モデル
  \item 水平輸送のパラメタリゼーション: 海氷厚さの水平拡散で表現
  \item 数値解法
  \begin{itemize}
    \item 水平離散化: 有限体積法
    \item 鉛直離散化: 有限体積法
    \item 時間離散化
    \begin{itemize}  
      \item 移流項: LF-AM3 スキーム \citep{shchepetkin2005regional}
  	  \item 鉛直拡散項: リープフロッグ法
  	  \item 水平拡散項: 前進オイラー法
    \end{itemize}                  
  \end{itemize}
\end{itemize}

\subsection*{その他の海洋モデル}

なお, \verb|${TOP_DIR}/model|内には, 上記のモデルのディレクトリ(dogcm)以外にも, 
「globalSWModel\_DG, globalSWModel\_FVM」等のディレクトリが存在する. 
これらは, 以下のように岸が取り扱うことを念頭にした海洋モデルの試作品であるが, 
これらのモデルの詳細については, 本書にはまだ記述できていない.  



